\documentclass[landscape]{ximera}
%\documentclass[landscape]{article}
%%% Begin Laad packages

\makeatletter
\@ifclassloaded{xourse}{%
    \typeout{Start loading preamble.tex (in a XOURSE)}%
    \def\isXourse{true}   % automatically defined; pre 112022 it had to be set 'manually' in a xourse
}{%
    \typeout{Start loading preamble.tex (NOT in a XOURSE)}%
}
\makeatother

\pgfplotsset{compat=1.16}

\usepackage{currfile}

% 201908/202301: PAS OP: babel en doclicense lijken problemen te veroorzaken in .jax bestand
% (wegens syntax error met toegevoegde \newcommands ...)
\pdfOnly{
    \usepackage[type={CC},modifier={by-nc-sa},version={4.0}]{doclicense}
    \usepackage[dutch]{babel}
}



\usepackage[utf8]{inputenc}
\usepackage{morewrites}   % nav zomercursus (answer...?)
\usepackage{multirow}
\usepackage{multicol}
\usepackage{tikzsymbols}
\usepackage{ifthen}
%\usepackage{animate} BREAKS HTML STRUCTURE USED BY XIMERA
\usepackage{relsize}

\usepackage{eurosym}    % \euro  (€ werkt niet in xake ...?)

% Nuttig als ook interactieve beamer slides worden voorzien:
\providecommand{\p}{} % default nothing ; potentially usefull for slides: redefine as \pause
%providecommand{\p}{\pause}

\usepackage{caption} % captionof
%\usepackage{pdflscape}    % landscape environment

% Met "\newcommand\showtodonotes{}" kan je todonotes tonen (in pdf/online)
% 201908: online werkt het niet (goed)
\providecommand\showtodonotes{disable}
\providecommand\todo[1]{\typeout{TODO #1}}
%\usepackage[\showtodonotes]{todonotes}
%\usepackage{todonotes}

%
% Poging tot aanpassen layout
%
\usepackage{tcolorbox}
\tcbuselibrary{theorems}

%%% Einde laad packages

%%% Begin Ximera specifieke zaken

\graphicspath{
	{../../}
	{../}
	{./}
  	{../../pictures/}
   	{../pictures/}
   	{./pictures/}
	{./explog/}    % M05 in groeimodellen       
}

%%% Einde Ximera specifieke zaken

%
% define softer blue/red/green, use KU Leuven base colors for blue (and dark orange for red ?)
%
% todo: rather redefine blue/red/green ...?
%\definecolor{xmblue}{rgb}{0.01, 0.31, 0.59}
%\definecolor{xmred}{rgb}{0.89, 0.02, 0.17}
\definecolor{xmdarkblue}{rgb}{0.122, 0.671, 0.835}   % KU Leuven Blauw
\definecolor{xmblue}{rgb}{0.114, 0.553, 0.69}        % KU Leuven Blauw
\definecolor{xmgreen}{rgb}{0.13, 0.55, 0.13}         % No KULeuven variant for green found ...

\definecolor{xmaccent}{rgb}{0.867, 0.541, 0.18}      % KU Leuven Accent (orange ...)
\definecolor{kuaccent}{rgb}{0.867, 0.541, 0.18}      % KU Leuven Accent (orange ...)

\colorlet{xmred}{xmaccent!50!black}                  % Darker version of KU Leuven Accent

\providecommand{\blue}[1]{{\color{blue}#1}}    
\providecommand{\red}[1]{{\color{red}#1}}

\renewcommand\CancelColor{\color{xmaccent!50!black}}

% werkt in math en text mode om MATH met oranje (of grijze...)  achtergond te tonen (ook \important{\text{blabla}} lijkt te werken)
%\newcommand{\important}[1]{\ensuremath{\colorbox{xmaccent!50!white}{$#1$}}}   % werkt niet in Mathjax
%\newcommand{\important}[1]{\ensuremath{\colorbox{lightgray}{$#1$}}}
\newcommand{\important}[1]{\ensuremath{\colorbox{orange}{$#1$}}}   % TODO: kleur aanpassen voor mathjax; wordt overschreven infra!


% Uitzonderlijk kan met \pdfnl in de PDF een newline worden geforceerd, die online niet nodig/nuttig is omdat daar de regellengte hoe dan ook niet gekend is.
\ifdefined\HCode%
\providecommand{\pdfnl}{}%
\else%
\providecommand{\pdfnl}{%
  \\%
}%
\fi

% Uitzonderlijk kan met \handoutnl in de handout-PDF een newline worden geforceerd, die noch online noch in de PDF-met-antwoorden nuttig is.
\ifdefined\HCode
\providecommand{\handoutnl}{}
\else
\providecommand{\handoutnl}{%
\ifhandout%
  \nl%
\fi%
}
\fi



% \cellcolor IGNORED by tex4ht ?
% \begin{center} seems not to wordk
    % (missing margin-left: auto;   on tabular-inside-center ???)
%\newcommand{\importantcell}[1]{\ensuremath{\cellcolor{lightgray}#1}}  %  in tabular; usablility to be checked
\providecommand{\importantcell}[1]{\ensuremath{#1}}     % no mathjax2 support for colloring array cells

\pdfOnly{
  \renewcommand{\important}[1]{\ensuremath{\colorbox{kuaccent!50!white}{$#1$}}}
  \renewcommand{\importantcell}[1]{\ensuremath{\cellcolor{kuaccent!40!white}#1}}   
}

%%% Tikz styles


\pgfplotsset{compat=1.16}

\usetikzlibrary{trees,positioning,arrows,fit,shapes,math,calc,decorations.markings,through,intersections,patterns,matrix}

\usetikzlibrary{decorations.pathreplacing,backgrounds}    % 5/2023: from experimental


\usetikzlibrary{angles,quotes}

\usepgfplotslibrary{fillbetween} % bepaalde_integraal
\usepgfplotslibrary{polar}    % oa voor poolcoordinaten.tex

\pgfplotsset{ownstyle/.style={axis lines = center, axis equal image, xlabel = $x$, ylabel = $y$, enlargelimits}} 

\pgfplotsset{
	plot/.style={no marks,samples=50}
}

\newcommand{\xmPlotsColor}{
	\pgfplotsset{
		plot1/.style={darkgray,no marks,samples=100},
		plot2/.style={lightgray,no marks,samples=100},
		plotresult/.style={blue,no marks,samples=100},
		plotblue/.style={blue,no marks,samples=100},
		plotred/.style={red,no marks,samples=100},
		plotgreen/.style={green,no marks,samples=100},
		plotpurple/.style={purple,no marks,samples=100}
	}
}
\newcommand{\xmPlotsBlackWhite}{
	\pgfplotsset{
		plot1/.style={black,loosely dashed,no marks,samples=100},
		plot2/.style={black,loosely dotted,no marks,samples=100},
		plotresult/.style={black,no marks,samples=100},
		plotblue/.style={black,no marks,samples=100},
		plotred/.style={black,dotted,no marks,samples=100},
		plotgreen/.style={black,dashed,no marks,samples=100},
		plotpurple/.style={black,dashdotted,no marks,samples=100}
	}
}


\newcommand{\xmPlotsColorAndStyle}{
	\pgfplotsset{
		plot1/.style={darkgray,no marks,samples=100},
		plot2/.style={lightgray,no marks,samples=100},
		plotresult/.style={blue,no marks,samples=100},
		plotblue/.style={xmblue,no marks,samples=100},
		plotred/.style={xmred,dashed,thick,no marks,samples=100},
		plotgreen/.style={xmgreen,dotted,very thick,no marks,samples=100},
		plotpurple/.style={purple,no marks,samples=100}
	}
}


%\iftikzexport
\xmPlotsColorAndStyle
%\else
%\xmPlotsBlackWhite
%\fi
%%%


%
% Om venndiagrammen te arceren ...
%
\makeatletter
\pgfdeclarepatternformonly[\hatchdistance,\hatchthickness]{north east hatch}% name
{\pgfqpoint{-1pt}{-1pt}}% below left
{\pgfqpoint{\hatchdistance}{\hatchdistance}}% above right
{\pgfpoint{\hatchdistance-1pt}{\hatchdistance-1pt}}%
{
	\pgfsetcolor{\tikz@pattern@color}
	\pgfsetlinewidth{\hatchthickness}
	\pgfpathmoveto{\pgfqpoint{0pt}{0pt}}
	\pgfpathlineto{\pgfqpoint{\hatchdistance}{\hatchdistance}}
	\pgfusepath{stroke}
}
\pgfdeclarepatternformonly[\hatchdistance,\hatchthickness]{north west hatch}% name
{\pgfqpoint{-\hatchthickness}{-\hatchthickness}}% below left
{\pgfqpoint{\hatchdistance+\hatchthickness}{\hatchdistance+\hatchthickness}}% above right
{\pgfpoint{\hatchdistance}{\hatchdistance}}%
{
	\pgfsetcolor{\tikz@pattern@color}
	\pgfsetlinewidth{\hatchthickness}
	\pgfpathmoveto{\pgfqpoint{\hatchdistance+\hatchthickness}{-\hatchthickness}}
	\pgfpathlineto{\pgfqpoint{-\hatchthickness}{\hatchdistance+\hatchthickness}}
	\pgfusepath{stroke}
}
%\makeatother

\tikzset{
    hatch distance/.store in=\hatchdistance,
    hatch distance=10pt,
    hatch thickness/.store in=\hatchthickness,
   	hatch thickness=2pt
}

\colorlet{circle edge}{black}
\colorlet{circle area}{blue!20}


\tikzset{
    filled/.style={fill=green!30, draw=circle edge, thick},
    arceerl/.style={pattern=north east hatch, pattern color=blue!50, draw=circle edge},
    arceerr/.style={pattern=north west hatch, pattern color=yellow!50, draw=circle edge},
    outline/.style={draw=circle edge, thick}
}




%%% Updaten commando's
\def\hoofding #1#2#3{\maketitle}     % OBSOLETE ??

% we willen (bijna) altijd \geqslant ipv \geq ...!
\newcommand{\geqnoslant}{\geq}
\renewcommand{\geq}{\geqslant}
\newcommand{\leqnoslant}{\leq}
\renewcommand{\leq}{\leqslant}

% Todo: (201908) waarom komt er (soms) underlined voor emph ...?
\renewcommand{\emph}[1]{\textit{#1}}

% API commando's

\newcommand{\ds}{\displaystyle}
\newcommand{\ts}{\textstyle}  % tegenhanger van \ds   (Ximera zet PER  DEFAULT \ds!)

% uit Zomercursus-macro's: 
\newcommand{\bron}[1]{\begin{scriptsize} \emph{#1} \end{scriptsize}}     % deprecated ...?


%definities nieuwe commando's - afkortingen veel gebruikte symbolen
\newcommand{\R}{\ensuremath{\mathbb{R}}}
\newcommand{\Rnul}{\ensuremath{\mathbb{R}_0}}
\newcommand{\Reen}{\ensuremath{\mathbb{R}\setminus\{1\}}}
\newcommand{\Rnuleen}{\ensuremath{\mathbb{R}\setminus\{0,1\}}}
\newcommand{\Rplus}{\ensuremath{\mathbb{R}^+}}
\newcommand{\Rmin}{\ensuremath{\mathbb{R}^-}}
\newcommand{\Rnulplus}{\ensuremath{\mathbb{R}_0^+}}
\newcommand{\Rnulmin}{\ensuremath{\mathbb{R}_0^-}}
\newcommand{\Rnuleenplus}{\ensuremath{\mathbb{R}^+\setminus\{0,1\}}}
\newcommand{\N}{\ensuremath{\mathbb{N}}}
\newcommand{\Nnul}{\ensuremath{\mathbb{N}_0}}
\newcommand{\Z}{\ensuremath{\mathbb{Z}}}
\newcommand{\Znul}{\ensuremath{\mathbb{Z}_0}}
\newcommand{\Zplus}{\ensuremath{\mathbb{Z}^+}}
\newcommand{\Zmin}{\ensuremath{\mathbb{Z}^-}}
\newcommand{\Znulplus}{\ensuremath{\mathbb{Z}_0^+}}
\newcommand{\Znulmin}{\ensuremath{\mathbb{Z}_0^-}}
\newcommand{\C}{\ensuremath{\mathbb{C}}}
\newcommand{\Cnul}{\ensuremath{\mathbb{C}_0}}
\newcommand{\Cplus}{\ensuremath{\mathbb{C}^+}}
\newcommand{\Cmin}{\ensuremath{\mathbb{C}^-}}
\newcommand{\Cnulplus}{\ensuremath{\mathbb{C}_0^+}}
\newcommand{\Cnulmin}{\ensuremath{\mathbb{C}_0^-}}
\newcommand{\Q}{\ensuremath{\mathbb{Q}}}
\newcommand{\Qnul}{\ensuremath{\mathbb{Q}_0}}
\newcommand{\Qplus}{\ensuremath{\mathbb{Q}^+}}
\newcommand{\Qmin}{\ensuremath{\mathbb{Q}^-}}
\newcommand{\Qnulplus}{\ensuremath{\mathbb{Q}_0^+}}
\newcommand{\Qnulmin}{\ensuremath{\mathbb{Q}_0^-}}

\newcommand{\perdef}{\overset{\mathrm{def}}{=}}
\newcommand{\pernot}{\overset{\mathrm{notatie}}{=}}
\newcommand\perinderdaad{\overset{!}{=}}     % voorlopig gebruikt in limietenrekenregels
\newcommand\perhaps{\overset{?}{=}}          % voorlopig gebruikt in limietenrekenregels

\newcommand{\degree}{^\circ}


\DeclareMathOperator{\dom}{dom}     % domein
\DeclareMathOperator{\codom}{codom} % codomein
\DeclareMathOperator{\bld}{bld}     % beeld
\DeclareMathOperator{\graf}{graf}   % grafiek
\DeclareMathOperator{\rico}{rico}   % richtingcoëfficient
\DeclareMathOperator{\co}{co}       % coordinaat
\DeclareMathOperator{\gr}{gr}       % graad

\newcommand{\func}[5]{\ensuremath{#1: #2 \rightarrow #3: #4 \mapsto #5}} % Easy to write a function


% Operators
\DeclareMathOperator{\bgsin}{bgsin}
\DeclareMathOperator{\bgcos}{bgcos}
\DeclareMathOperator{\bgtan}{bgtan}
\DeclareMathOperator{\bgcot}{bgcot}
\DeclareMathOperator{\bgsinh}{bgsinh}
\DeclareMathOperator{\bgcosh}{bgcosh}
\DeclareMathOperator{\bgtanh}{bgtanh}
\DeclareMathOperator{\bgcoth}{bgcoth}

% Oude \Bgsin etc deprecated: gebruik \bgsin, en herdefinieer dat als je Bgsin wil!
%\DeclareMathOperator{\cosec}{cosec}    % not used? gebruik \csc en herdefinieer

% operatoren voor differentialen: to be verified; 1/2020: inconsequent gebruik bij afgeleiden/integralen
\renewcommand{\d}{\mathrm{d}}
\newcommand{\dx}{\d x}
\newcommand{\dd}[1]{\frac{\mathrm{d}}{\mathrm{d}#1}}
\newcommand{\ddx}{\dd{x}}

% om in voorbeelden/oefeningen de notatie voor afgeleiden te kunnen kiezen
% Usage: \afg{(2\sin(x))}  (en wordt d/dx, of accent, of D )
\newcommand{\afg}[1]{{#1}'}
%\renewcommand{\afg}[1]{\frac{\mathrm{d}#1}{\mathrm{d}x}}   % include in relevant exercises ...
%\renewcommand{\afg}[1]{D{#1}}

%
% \xmxxx commands: Extra KU Leuven functionaliteit van, boven of naast Ximera
%   ( Conventie 8/2019: xm+nederlandse omschrijving, maar is niet consequent gevolgd, en misschien ook niet erg handig !)
%
% (Met een minimale ximera.cls en preamble.tex zou een bruikbare .pdf moeten kunnen worden gemaakt van eender welke ximera)
%
% Usage: \xmtitle[Mijn korte abstract]{Mijn titel}{Mijn abstract}
% Eerste command na \begin{document}:
%  -> definieert de \title
%  -> definieert de abstract
%  -> doet \maketitle ( dus: print de hoofding als 'chapter' of 'sectie')
% Optionele parameter geeft eenn kort abstract (die met de globale setting \xmshortabstract{} al dan niet kan worden geprint.
% De optionele korte abstract kan worden gebruikt voor pseudo-grappige abtsarts, dus dus globaal al dan niet kunnen worden gebuikt...
% Globale settings:
%  de (optionele) 'korte abstract' wordt enkele getoond als \xmshortabstract is gezet
\providecommand\xmshortabstract{} % default: print (only!) short abstract if present
\providecommand\theabstract{} % otherwise complaint Undefined control sequence.  <recently read> \theabstract  ????
\newcommand{\xmtitle}[3][]{
	\title{#2}
	% \begin{abstract}
	% 			\ifdefined\xmshortabstract
	% 			\ifstrempty{#1}{%
	% 						#3
	% 			}{%
	% 						#1
	% 			}%
	% 			\else
	% 			#3
	% 			\fi
	% \end{abstract}
	\maketitle
}

% 
% Kleine grapjes: moeten zonder verder gevolg kunnen worden verwijderd
%
%\newcommand{\xmopje}[1]{{\small#1{\reversemarginpar\marginpar{\Smiley}}}}   % probleem in floats!!
\newtoggle{showxmopje}
\toggletrue{showxmopje}

\newcommand{\xmopje}[1]{%
   \iftoggle{showxmopje}{#1}{}%
}


% -> geef een abstracte-formule-met-rechts-een-concreet-voorbeeld
% VB:  \formulevb{a^2+b^2=c^2}{3^2+4^2=5^2}
%
\ifdefined\HCode
\NewEnviron{xmdiv}[1]{\HCode{\Hnewline<div class="#1">\Hnewline}\BODY{\HCode{\Hnewline</div>\Hnewline}}}
\else
\NewEnviron{xmdiv}[1]{\BODY}
\fi

\providecommand{\formulevb}[2]{
	{\centering

    \begin{xmdiv}{xmformulevb}    % zie css voor online layout !!!
	\begin{tabular}{lcl}
		\important{#1}
		&  &
		Vb: $#2$
		\end{tabular}
	\end{xmdiv}

	}
}

\ifdefined\HCode
\providecommand{\xmcolorbox}[2]{
	\HCode{\Hnewline<div class="xmcolorbox">\Hnewline}#2\HCode{\Hnewline</div>\Hnewline}
}
\else
\providecommand{\xmcolorbox}[2]{
  \cellcolor{#1}#2
}
\fi


\ifdefined\HCode
\providecommand{\xmopmerking}[1]{
 \HCode{\Hnewline<div class="xmopmerking">\Hnewline}#1\HCode{\Hnewline</div>\Hnewline}
}
\else
\providecommand{\xmopmerking}[1]{
	{\footnotesize #1}
}
\fi
% \providecommand{\voorbeeld}[1]{
% 	\colorbox{blue!10}{$#1$}
% }



% Hernoem Proof naar Bewijs, nodig voor HTML versie
\renewcommand*{\proofname}{Bewijs}

% Om opgave van oefening (wordt niet geprint bij oplossingenblad)
% (to be tested test)
\NewEnviron{statement}{\BODY}

% Environment 'oplossing' en 'uitkomst'
% voor resp. volledige 'uitwerking' dan wel 'enkel eindresultaat'
% geimplementeerd via feedback, omdat er in de ximera-server adhoc feedback-code is toegevoegd
%% Niet tonen indien handout
%% Te gebruiken om volledige oplossingen/uitwerkingen van oefeningen te tonen
%% \begin{oplossing}        De optelling is commutatief \end{oplossing}  : verschijnt online enkel 'op vraag'
%% \begin{oplossing}[toon]  De optelling is commutatief \end{oplossing}  : verschijnt steeds onmiddellijk online (bv te gebruiken bij voorbeelden) 

\ifhandout%
    \NewEnviron{oplossing}[1][onzichtbaar]%
    {%
    \ifthenelse{\equal{\detokenize{#1}}{\detokenize{toon}}}
    {
    \def\PH@Command{#1}% Use PH@Command to hold the content and be a target for "\expandafter" to expand once.

    \begin{trivlist}% Begin the trivlist to use formating of the "Feedback" label.
    \item[\hskip \labelsep\small\slshape\bfseries Oplossing% Format the "Feedback" label. Don't forget the space.
    %(\texttt{\detokenize\expandafter{\PH@Command}}):% Format (and detokenize) the condition for feedback to trigger
    \hspace{2ex}]\small%\slshape% Insert some space before the actual feedback given.
    \BODY
    \end{trivlist}
    }
    {  % \begin{feedback}[solution]   \BODY     \end{feedback}  }
    }
    }    
\else
% ONLY for HTML; xmoplossing is styled with css, and is not, and need not be a LaTeX environment
% THUS: it does NOT use feedback anymore ...
%    \NewEnviron{oplossing}{\begin{expandable}{xmoplossing}{\nlen{Toon uitwerking}{Show solution}}{\BODY}\end{expandable}}
    \newenvironment{oplossing}[1][onzichtbaar]
   {%
       \begin{expandable}{xmoplossing}{}
   }
   {%
   	   \end{expandable}
   } 
%     \newenvironment{oplossing}[1][onzichtbaar]
%    {%
%        \begin{feedback}[solution]   	
%    }
%    {%
%    	   \end{feedback}
%    } 
\fi

\ifhandout%
    \NewEnviron{uitkomst}[1][onzichtbaar]%
    {%
    \ifthenelse{\equal{\detokenize{#1}}{\detokenize{toon}}}
    {
    \def\PH@Command{#1}% Use PH@Command to hold the content and be a target for "\expandafter" to expand once.

    \begin{trivlist}% Begin the trivlist to use formating of the "Feedback" label.
    \item[\hskip \labelsep\small\slshape\bfseries Uitkomst:% Format the "Feedback" label. Don't forget the space.
    %(\texttt{\detokenize\expandafter{\PH@Command}}):% Format (and detokenize) the condition for feedback to trigger
    \hspace{2ex}]\small%\slshape% Insert some space before the actual feedback given.
    \BODY
    \end{trivlist}
    }
    {  % \begin{feedback}[solution]   \BODY     \end{feedback}  }
    }
    }    
\else
\ifdefined\HCode
   \newenvironment{uitkomst}[1][onzichtbaar]
    {%
        \begin{expandable}{xmuitkomst}{}%
    }
    {%
    	\end{expandable}%
    } 
\else
  % Do NOT print 'uitkomst' in non-handout
  %  (presumably, there is also an 'oplossing' ??)
  \newenvironment{uitkomst}[1][onzichtbaar]{}{}
\fi
\fi

%
% Uitweidingen zijn extra's die niet redelijkerwijze tot de leerstof behoren
% Uitbreidingen zijn extra's die wel redelijkerwijze tot de leerstof van bv meer geavanceerde versies kunnen behoren (B-programma/Wiskundestudenten/...?)
% Nog niet voorzien: design voor verschillende versies (A/B programma, BIO, voorkennis/ ...)
% Voor 'uitweidingen' is er een environment die online per default is ingeklapt, en in pdf al dan niet kan worden geincluded  (via \xmnouitweiding) 
%
% in een xourse, per default GEEN uitweidingen, tenzij \xmuitweiding expliciet ergens is gezet ...
\ifdefined\isXourse
   \ifdefined\xmuitweiding
   \else
       \def\xmnouitweiding{true}
   \fi
\fi

\ifdefined\xmnouitweiding
\newcounter{xmuitweiding}  % anders error undefined ...  
\excludecomment{xmuitweiding}
\else
\newtheoremstyle{dotless}{}{}{}{}{}{}{ }{}
\theoremstyle{dotless}
\newtheorem*{xmuitweidingnofrills}{}   % nofrills = no accordion; gebruikt dus de dotless theoremstyle!

\newcounter{xmuitweiding}
\newenvironment{xmuitweiding}[1][ ]%
{% 
	\refstepcounter{xmuitweiding}%
    \begin{expandable}{xmuitweiding}{Uitweiding \arabic{xmuitweiding}: #1}%
	\begin{xmuitweidingnofrills}%
}
{%
    \end{xmuitweidingnofrills}%
    \end{expandable}%
}   
% \newenvironment{xmuitweiding}[1][ ]%
% {% 
% 	\refstepcounter{xmuitweiding}
% 	\begin{accordion}\begin{accordion-item}[Uitweiding \arabic{xmuitweiding}: #1]%
% 			\begin{xmuitweidingnofrills}%
% 			}
% 			{\end{xmuitweidingnofrills}\end{accordion-item}\end{accordion}}   
\fi


\newenvironment{xmexpandable}[1][]{
	\begin{accordion}\begin{accordion-item}[#1]%
		}{\end{accordion-item}\end{accordion}}


% Command that gives a selection box online, but just prints the right answer in pdf
\newcommand{\xmonlineChoice}[1]{\pdfOnly{\wordchoicegiventrue}\wordChoice{#1}\pdfOnly{\wordchoicegivenfalse}}   % deprecated, gebruik onlineChoice ...
\newcommand{\onlineChoice}[1]{\pdfOnly{\wordchoicegiventrue}\wordChoice{#1}\pdfOnly{\wordchoicegivenfalse}}

\newcommand{\TJa}{\nlentext{ Ja }{ Yes }}
\newcommand{\TNee}{\nlentext{ Nee }{ No }}
\newcommand{\TJuist}{\nlentext{ Juist }{ True }}
\newcommand{\TFout}{\nlentext{ Fout }{ False }}

\newcommand{\choiceTrue}{{\wordChoice{\choice[correct]{\TJuist}\choice{\TFout}}}}
\newcommand{\choiceFalse}{{\wordChoice{\choice{\TJuist}\choice[correct]{\TFout}}}}

\newcommand{\choiceYes}{{\wordChoice{\choice[correct]{\TJa}\choice{\TNee}}}}
\newcommand{\choiceNo}{{\wordChoice{\choice{\TJa}\choice[correct]{\TNee}}}}

\newcommand{\choiceEen}{{\wordChoice{\choice[correct]{een }\choice{geen }}}}
\newcommand{\choiceGeen}{{\wordChoice{\choice{een }\choice[correct]{geen }}}}

% Optional nicer formatting for wordChoice in PDF

\let\inlinechoiceorig\inlinechoice

%\makeatletter
%\providecommand{\choiceminimumverticalsize}{\vphantom{$\frac{\sqrt{2}}{2}$}}   % minimum height of boxes (cfr infra)
\providecommand{\choiceminimumverticalsize}{\vphantom{$\tfrac{2}{2}$}}   % minimum height of boxes (cfr infra)

\newcommand{\inlinechoicesquares}[2][]{%
		\setkeys{choice}{#1}%
		\ifthenelse{\boolean{\choice@correct}}%
		{%
            \ifhandout%
               \fbox{\choiceminimumverticalsize #2}\allowbreak\ignorespaces%
            \else%
               \fcolorbox{blue}{blue!20}{\choiceminimumverticalsize #2\checkmark}\allowbreak\ignorespaces\setkeys{choice}{correct=false}\ignorespaces%
            \fi%
		}%
		{% else
			\fbox{\choiceminimumverticalsize #2}\allowbreak\ignorespaces%  HACK: wat kleiner, zodat fits on line ... 	
		}%
}

\newcommand{\inlinechoicesquareX}[2][]{%
		\setkeys{choice}{#1}%
		\ifthenelse{\boolean{\choice@correct}}%
		{%
            \ifhandout%
               \fbox{\choiceminimumverticalsize #2}\allowbreak\ignorespaces\setkeys{choice}{correct=false}\ignorespaces%
            \else%
               \fcolorbox{blue}{blue!20}{\choiceminimumverticalsize #2\checkmark}\allowbreak\ignorespaces\setkeys{choice}{correct=false}\ignorespaces%
            \fi%
		}%
		{% else
        \ifhandout%
			\fbox{\choiceminimumverticalsize #2}\allowbreak\ignorespaces%  HACK: wat kleiner, zodat fits on line ... 	
        \fi
		}%
}


\newcommand{\inlinechoicepuntjes}[2][]{%
		\setkeys{choice}{#1}%
		\ifthenelse{\boolean{\choice@correct}}%
		{%
            \ifhandout%
               \dots\ldots\ignorespaces\setkeys{choice}{correct=false}\ignorespaces
            \else%
               \fcolorbox{blue}{blue!20}{\choiceminimumverticalsize #2}\allowbreak\ignorespaces\setkeys{choice}{correct=false}\ignorespaces%
            \fi%
		}%
		{% else
			%\fbox{\choiceminimumverticalsize #2}\allowbreak\ignorespaces%  HACK: wat kleiner, zodat fits on line ... 	
		}%
}

% print niets, maar definieer globale variable \myanswer
%  (gebruikt om oplossingsbladen te printen) 
\newcommand{\inlinechoicedefanswer}[2][]{%
		\setkeys{choice}{#1}%
		\ifthenelse{\boolean{\choice@correct}}%
		{%
               \gdef\myanswer{#2}\setkeys{choice}{correct=false}

		}%
		{% else
			%\fbox{\choiceminimumverticalsize #2}\allowbreak\ignorespaces%  HACK: wat kleiner, zodat fits on line ... 	
		}%
}



%\makeatother

\newcommand{\setchoicedefanswer}{
\ifdefined\HCode
\else
%    \renewenvironment{multipleChoice@}[1][]{}{} % remove trailing ')'
    \let\inlinechoice\inlinechoicedefanswer
\fi
}

\newcommand{\setchoicepuntjes}{
\ifdefined\HCode
\else
    \renewenvironment{multipleChoice@}[1][]{}{} % remove trailing ')'
    \let\inlinechoice\inlinechoicepuntjes
\fi
}
\newcommand{\setchoicesquares}{
\ifdefined\HCode
\else
    \renewenvironment{multipleChoice@}[1][]{}{} % remove trailing ')'
    \let\inlinechoice\inlinechoicesquares
\fi
}
%
\newcommand{\setchoicesquareX}{
\ifdefined\HCode
\else
    \renewenvironment{multipleChoice@}[1][]{}{} % remove trailing ')'
    \let\inlinechoice\inlinechoicesquareX
\fi
}
%
\newcommand{\setchoicelist}{
\ifdefined\HCode
\else
    \renewenvironment{multipleChoice@}[1][]{}{)}% re-add trailing ')'
    \let\inlinechoice\inlinechoiceorig
\fi
}

\setchoicesquareX  % by default list-of-squares with onlineChoice in PDF

% Omdat multicols niet werkt in html: enkel in pdf  (in html zijn langere pagina's misschien ook minder storend)
\newenvironment{xmmulticols}[1][2]{
 \pdfOnly{\begin{multicols}{#1}}%
}{ \pdfOnly{\end{multicols}}}

%
% Te gebruiken in plaats van \section\subsection
%  (in een printstyle kan dan het level worden aangepast
%    naargelang \chapter vs \section style )
% 3/2021: DO NOT USE \xmsubsection !
\newcommand\xmsection\subsection
\newcommand\xmsubsection\subsubsection

% Aanpassen printversie
%  (hier gedefinieerd, zodat ze in xourse kunnen worden gezet/overschreven)
\providebool{parttoc}
\providebool{printpartfrontpage}
\providebool{printactivitytitle}
\providebool{printactivityqrcode}
\providebool{printactivityurl}
\providebool{printcontinuouspagenumbers}

% The following three commands are hardcoded in xake, you can't create other commands like these, without adding them to xake as well
%  ( gebruikt in xourses om juiste soort titelpagina te krijgen voor verschillende ximera's )
\newcommand{\activitychapter}[1]{
	\typeout{ACTIVITYCHAPTER #1}   % logging
	\chapterstyle
	\activity{#1}
}
\newcommand{\activitysection}[1]{
	\typeout{ACTIVITYSECTION #1}   % logging
	\sectionstyle
	\activity{#1}
}
% Partices worden als activity getoond om de grote blokken te krijgen online
\newcommand{\practicesection}[1]{
	\typeout{PRACTICESECTION #1}   % logging
	\sectionstyle
	\activity{#1}
}


% Commando om de printstyle toe te voegen in ximera's. Zorgt ervoor dat er geen problemen zijn als je de xourses compileert
% hack om onhandige relative paden in TeX te omzeilen
% should work both in xourse and ximera (pre-112022 only in ximera; thus obsoletes adhoc setup in xourses)
% loads global.sty if present (cfr global.css for online settings!)
% use global.sty to overwrite settings in printstyle.sty ...
\newcommand{\addPrintStyle}[1]{
\iftikzexport\else   % only in PDF
  \makeatletter
  \ifx\@onlypreamble\@notprerr\else   % ONLY if in tex-preamble   (and e.g. not when included from xourse)
    \typeout{Loading printstyle}   % logging
    \usepackage{#1/printstyle} % mag enkel geinclude worden als je die apart compileert
    \IfFileExists{#1/global.sty}{
        \typeout{Loading printstyle-folder #1/global.sty}   % logging
        \usepackage{#1/global}
        }{
        \typeout{Info: No extra #1/global.sty}   % logging
    }   % load global.sty if present
    \IfFileExists{global.sty}{
        \typeout{Loading local-folder global.sty (or TEXINPUTPATH..)}   % logging
        \usepackage{global}
    }{
        \typeout{Info: No folder/global.sty}   % logging
    }   % load global.sty if present
    \IfFileExists{\currfilebase.sty}
    {
        \typeout{Loading \currfilebase.sty}
        \input{\currfilebase.sty}
    }{
        \typeout{Info: No local \currfilebase.sty}
    }
    \fi
  \makeatother
\fi
}

%
%  
% references: Ximera heeft adhoc logica	 om online labels te doen werken over verschillende files heen
% met \hyperref kan de getoonde tekst toch worden opgegeven, in plaats van af te hangen van de label-text
\ifdefined\HCode
% Link to standard \labels, but give your own description
% Usage:  Volg \hyperref[my_very_verbose_label]{deze link} voor wat tijdverlies
%   (01/2020: Ximera-server aangepast om bij class reference-keeptext de link-text NIET te vervangen door de label-text !!!) 
\renewcommand{\hyperref}[2][]{\HCode{<a class="reference reference-keeptext" href="\##1">}#2\HCode{</a>}}
%
%  Link to specific targets  (not tested ?)
\renewcommand{\hypertarget}[1]{\HCode{<a class="ximera-label" id="#1"></a>}}
\renewcommand{\hyperlink}[2]{\HCode{<a class="reference reference-keeptext" href="\##1">}#2\HCode{</a>}}
\fi


\renewcommand{\figurename}{Figuur}
\renewcommand{\tablename}{Tabel}

%
% Gedoe om verschillende versies van xourse/ximera te maken afhankelijk van settings
%
% default: versie met antwoorden
% handout: versie voor de studenten, zonder antwoorden/oplossingen
% full: met alles erop en eraan, dus geschikt voor auteurs en/of lesgevers  (bevat in de pdf ook de 'online-only' stukken!)
%
%
% verder kunnen ook opties/variabele worden gezet voor hints/auteurs/uitweidingen/ etc
%
% 'Full' versie
\newtoggle{showonline}
\ifdefined\HCode   % zet default showOnline
    \toggletrue{showonline} 
\else
    \togglefalse{showonline}
\fi

% Full versie   % deprecated: see infra
\newcommand{\printFull}{
    \hintstrue
    \handoutfalse
    \toggletrue{showonline} 
}

\ifdefined\shouldPrintFull   % deprecated: see infra
    \printFull
\fi

%% \onlineOnly kan jammer genoeg niet, omdat het al betsaat als neveneffect van \begin{onlineOnly} ...
\newcommand{\onlyOnline}[1]{\ifdefined\HCode#1\fi}

% Overschrijf onlineOnly  (zoals gedefinieerd in ximera.cls)
\ifhandout   % in handout: gebruik de oorspronkelijke ximera.cls implementatie  (is dit wel nodig/nuttig?)
\else
    \iftoggle{showonline}{%
        \ifdefined\HCode
          \RenewEnviron{onlineOnly}{\bgroup\BODY\egroup}   % showOnline, en we zijn  online, dus toon de tekst
        \else
          \RenewEnviron{onlineOnly}{\bgroup\color{red!50!black}\BODY\egroup}   % showOnline, maar we zijn toch niet online: kleur de tekst rood 
        \fi
    }{%
      \RenewEnviron{onlineOnly}{\setbox0\vbox\bgroup\BODY\egroup}% geen showOnline
    }
\fi

% hack om na hoofding van definition/proposition/... als dan niet op een nieuwe lijn te starten
% soms is dat goed en mooi, en soms niet; en in HTML is het nu (2/2020) anders dan in pdf
% vandaar suggestie om 
%     \begin{definition}[Nieuw concept] \nl
% te gebruiken als je zeker een newline wil na de hoofdig en titel
% (in het bijzonder itemize zonder \nl is 'lelijk' ...)
\ifdefined\HCode
\newcommand{\nl}{}
\else
\newcommand{\nl}{\ \par} % newline (achter heading van definition etc.)
\fi


% \nl enkel in handoutmode (ihb voor \wordChoice, die dan typisch veeeel langer wordt)
\ifdefined\HCode
\providecommand{\handoutnl}{}
\else
\providecommand{\handoutnl}{%
\ifhandout%
  \nl%
\fi%
}
\fi

% Could potentially replace \pdfOnline/\begin{onlineOnly} : 
% Usage= \ifonline{Hallo surfer}{Hallo PDFlezer}
\providecommand{\ifonline}[2]%
{
\begin{onlineOnly}#1\end{onlineOnly}%
\pdfOnly{#2}
}%


%
% Maak optionele 'basic' en 'extended' versies van een activity
%  met environment basicOnly, basicSkip en extendedOnly
%
%  (
%   Dit werkt ENKEL in de PDF; de online versies tonen (minstens voorklopig) steeds 
%   het default geval met printbasicversion en printextendversion beide FALSE
%  )
%
\providebool{printbasicversion}
\providebool{printextendedversion}   % not properly implemented
\providebool{printfullversion}       % presumably print everything (debug/auteur)
%
% only set these in xourses, and BEFORE loading this preamble
%
%\newif\ifshowbasic     \showbasictrue        % use this line in xourse to show 'basic' sections
%\newif\ifshowextended  \showextendedtrue     % use this line in xourse to show 'extended' sections
%
%
%\ifbool{showbasic}
%      { \NewEnviron{basicOnly}{\BODY} }    % if yes: just print contents
%      { \NewEnviron{basicOnly}{}      }    % if no:  completely ignore contents
%
%\ifbool{showbasic}
%      { \NewEnviron{basicSkip}{}      }
%      { \NewEnviron{basicSkip}{\BODY} }
%

\ifbool{printextendedversion}
      { \NewEnviron{extendedOnly}{\BODY} }
      { \NewEnviron{extendedOnly}{}      }
      


\ifdefined\HCode    % in html: always print
      \newenvironment*{basicOnly}{}{}    % if yes: just print contents
      \newenvironment*{basicSkip}{}{}    % if yes: just print contents
\else

\ifbool{printbasicversion}
      {\newenvironment*{basicOnly}{}{}}    % if yes: just print contents
      {\NewEnviron{basicOnly}{}      }    % if no:  completely ignore contents

\ifbool{printbasicversion}
      {\NewEnviron{basicSkip}{}      }
      {\newenvironment*{basicSkip}{}{}}

\fi

\usepackage{float}
\usepackage[rightbars,color]{changebar}

% Full versie
\ifbool{printfullversion}{
    \hintstrue
    \handoutfalse
    \toggletrue{showonline}
    \printbasicversionfalse
    \cbcolor{red}
    \renewenvironment*{basicOnly}{\cbstart}{\cbend}
    \renewenvironment*{basicSkip}{\cbstart}{\cbend}
    \def\xmtoonprintopties{FULL}   % will be printed in footer
}
{}
      
%
% Evalueer \ifhints IN de environment
%  
%
%\RenewEnviron{hint}
%{
%\ifhandout
%\ifhints\else\setbox0\vbox\fi%everything in een emty box
%\bgroup 
%\stepcounter{hintLevel}
%\BODY
%\egroup\ignorespacesafterend
%\addtocounter{hintLevel}{-1}
%\else
%\ifhints
%\begin{trivlist}\item[\hskip \labelsep\small\slshape\bfseries Hint:\hspace{2ex}]
%\small\slshape
%\stepcounter{hintLevel}
%\BODY
%\end{trivlist}
%\addtocounter{hintLevel}{-1}
%\fi
%\fi
%}

% Onafhankelijk van \ifhandout ...? TO BE VERIFIED
\RenewEnviron{hint}
{
\ifhints
\begin{trivlist}\item[\hskip \labelsep\small\bfseries Hint:\hspace{2ex}]
\small%\slshape
\stepcounter{hintLevel}
\BODY
\end{trivlist}
\addtocounter{hintLevel}{-1}
\else
\iftikzexport   % anders worden de tikz tekeningen in hints niet gegenereerd ?
\setbox0\vbox\bgroup
\stepcounter{hintLevel}
\BODY
\egroup\ignorespacesafterend
\addtocounter{hintLevel}{-1}
\fi % ifhandout
\fi %ifhints
}

%
% \tab sets typewriter-tabs (e.g. to format questions)
% (Has no effect in HTML :-( ))
%
\usepackage{tabto}
\ifdefined\HCode
  \renewcommand{\tab}{\quad}    % otherwise dummy .png's are generated ...?
\fi


% Also redefined in  preamble to get correct styling 
% for tikz images for (\tikzexport)
%

\theoremstyle{definition} % Bold titels
\makeatletter
\let\proposition\relax
\let\c@proposition\relax
\let\endproposition\relax
\makeatother
\newtheorem{proposition}{Eigenschap}


%\instructornotesfalse

% logic with \ifhandoutin ximera.cls unclear;so overwrite ...
\makeatletter
\@ifundefined{ifinstructornotes}{%
  \newif\ifinstructornotes
  \instructornotesfalse
  \newenvironment{instructorNotes}{}{}
}{}
\makeatother
\ifinstructornotes
\else
\renewenvironment{instructorNotes}%
{%
    \setbox0\vbox\bgroup
}
{%
    \egroup
}
\fi

% \RedeclareMathOperator
% from https://tex.stackexchange.com/questions/175251/how-to-redefine-a-command-using-declaremathoperator
\makeatletter
\newcommand\RedeclareMathOperator{%
    \@ifstar{\def\rmo@s{m}\rmo@redeclare}{\def\rmo@s{o}\rmo@redeclare}%
}
% this is taken from \renew@command
\newcommand\rmo@redeclare[2]{%
    \begingroup \escapechar\m@ne\xdef\@gtempa{{\string#1}}\endgroup
    \expandafter\@ifundefined\@gtempa
    {\@latex@error{\noexpand#1undefined}\@ehc}%
    \relax
    \expandafter\rmo@declmathop\rmo@s{#1}{#2}}
% This is just \@declmathop without \@ifdefinable
\newcommand\rmo@declmathop[3]{%
    \DeclareRobustCommand{#2}{\qopname\newmcodes@#1{#3}}%
}
\@onlypreamble\RedeclareMathOperator
\makeatother


%
% Engelse vertaling, vooral in mathmode
%
% 1. Algemene procedure
%
\ifdefined\isEn
 \newcommand{\nlen}[2]{#2}
 \newcommand{\nlentext}[2]{\text{#2}}
 \newcommand{\nlentextbf}[2]{\textbf{#2}}
\else
 \newcommand{\nlen}[2]{#1}
 \newcommand{\nlentext}[2]{\text{#1}}
 \newcommand{\nlentextbf}[2]{\textbf{#1}}
\fi

%
% 2. Lijst van erg veel gebruikte uitdrukkingen
%

% Ja/Nee/Fout/Juits etc
%\newcommand{\TJa}{\nlentext{ Ja }{ and }}
%\newcommand{\TNee}{\nlentext{ Nee }{ No }}
%\newcommand{\TJuist}{\nlentext{ Juist }{ Correct }
%\newcommand{\TFout}{\nlentext{ Fout }{ Wrong }
\newcommand{\TWaar}{\nlentext{ Waar }{ True }}
\newcommand{\TOnwaar}{\nlentext{ Vals }{ False }}
% Korte bindwoorden en, of, dus, ...
\newcommand{\Ten}{\nlentext{ en }{ and }}
\newcommand{\Tof}{\nlentext{ of }{ or }}
\newcommand{\Tdus}{\nlentext{ dus }{ so }}
\newcommand{\Tendus}{\nlentext{ en dus }{ and thus }}
\newcommand{\Tvooralle}{\nlentext{ voor alle }{ for all }}
\newcommand{\Took}{\nlentext{ ook }{ also }}
\newcommand{\Tals}{\nlentext{ als }{ when }} %of if?
\newcommand{\Twant}{\nlentext{ want }{ as }}
\newcommand{\Tmaal}{\nlentext{ maal }{ times }}
\newcommand{\Toptellen}{\nlentext{ optellen }{ add }}
\newcommand{\Tde}{\nlentext{ de }{ the }}
\newcommand{\Thet}{\nlentext{ het }{ the }}
\newcommand{\Tis}{\nlentext{ is }{ is }} %zodat is in text staat in mathmode (geen italics)
\newcommand{\Tmet}{\nlentext{ met }{ where }} % in situaties e.g met p < n --> where p < n
\newcommand{\Tnooit}{\nlentext{ nooit }{ never }}
\newcommand{\Tmaar}{\nlentext{ maar }{ but }}
\newcommand{\Tniet}{\nlentext{ niet }{ not }}
\newcommand{\Tuit}{\nlentext{ uit }{ from }}
\newcommand{\Ttov}{\nlentext{ t.o.v. }{ w.r.t. }}
\newcommand{\Tzodat}{\nlentext{ zodat }{ such that }}
\newcommand{\Tdeth}{\nlentext{de }{th }}
\newcommand{\Tomdat}{\nlentext{omdat }{because }} 


%
% Overschrijf addhoc commando's
%
\ifdefined\isEn
\renewcommand{\pernot}{\overset{\mathrm{notation}}{=}}
\RedeclareMathOperator{\bld}{im}     % beeld
\RedeclareMathOperator{\graf}{graph}   % grafiek
\RedeclareMathOperator{\rico}{slope}   % richtingcoëfficient
\RedeclareMathOperator{\co}{co}       % coordinaat
\RedeclareMathOperator{\gr}{deg}       % graad

% Operators
\RedeclareMathOperator{\bgsin}{arcsin}
\RedeclareMathOperator{\bgcos}{arccos}
\RedeclareMathOperator{\bgtan}{arctan}
\RedeclareMathOperator{\bgcot}{arccot}
\RedeclareMathOperator{\bgsinh}{arcsinh}
\RedeclareMathOperator{\bgcosh}{arccosh}
\RedeclareMathOperator{\bgtanh}{arctanh}
\RedeclareMathOperator{\bgcoth}{arccoth}

\fi

\renewcommand{\Im}[1]{\text{Im}#1}
\renewcommand{\Re}[1]{\text{Re}#1}


% Problem-inside-div  (for css styling ...)
\newcommand{\xmdivEnvironmentStart}[3]{%
\ifdefined\HCode%
   \HCode{\Hnewline<div class="#2">}%
\fi%
\problemEnvironmentStart{#1}{#3}%
}


\newcommand{\xmdivEnvironmentEnd}{%
\problemEnvironmentEnd%
\ifdefined\HCode%
    \HCode{\Hnewline</div>}%
\fi%
}


\newenvironment{quickquestion*}[1][2in]%
{%Env start code
\xmdivEnvironmentStart{#1}{quickquestion}{Quick Question}%
}
{%Env end code
\xmdivEnvironmentEnd%
}
\newenvironment{quickquestion}[1][2in]%
{%Env start code
\xmdivEnvironmentStart{#1}{quickquestion}{Quick Question}%
}
{%Env end code
\xmdivEnvironmentEnd%
}

\newenvironment{denkvraag*}[1][2in]%
{%Env start code
\xmdivEnvironmentStart{#1}{denkvraag}{Denkvraag}%
}
{%Env end code
\xmdivEnvironmentEnd
}

\newenvironment{denkvraag}[1][2in]%
{%Env start code
\xmdivEnvironmentStart{#1}{denkvraag}{Denkvraag}%
}
{%Env end code
\xmdivEnvironmentEnd
}

\input{../preambles/poster.tex}
%\input{../preambles/linalg}
\addPrintStyle{..}

\begin{document}
    \author{Zomercursus KU Leuven}
    \xmtitle{Voorkennis complexe getallen}{}
    \label{steekkaart:complexe_getallen_voorkennis}

\tikzset{>=latex}	
\renewcommand{\important}[1]{\ensuremath{\fcolorbox{kuaccent!50!white}{kuaccent!50!white}{$#1$}}}   % HACK: NO border ...

\begin{tcbposter}
    
\xmpostertitle{Voorkennis voor complexe getallen}{span=6}

\posterbox[adjusted title=Tweedegraadsvergelijking met re\"ele co\"effici\"enten]{
	name=l1,column=1,span=6,below=title
}{	

De oplossingen van de vergelijking $ax^2+bx+c=0$ met $\blue{a,b,c \in \R}$ en $D = b^2-4ac$ zijn:

\begin{tabular}{lll}
als $D> 0$: 
		& $x_1 = \frac{-b + \sqrt{D}}{2a}$ en $x_2 = \frac{-b - \sqrt{D}}{2a}$
		& \textbf{twee oplossingen}\\[2mm]
als $D= 0$: 
		& $x_1 = x_2 =\frac{-b}{2a}$ 
		& \textbf{één oplossing} \\[2mm] % (tweevoudig nulpunt)\\
als $D< 0$: 
		& & \textbf{geen oplossingen}\\	
\end{tabular}
}

\posterbox[adjusted title={Merkwaardige producten}]{
    name=r1,column=7,span=6,below=title,
}{
    $ (a+b)^2 = a^2 + 2ab + b^2 $ \qquad en \qquad 
    $ a^2 - b^2 = (a+b)(a-b) $
}

\posterbox[adjusted title={Rekenen met machten en wortels}]{
	name=r2,column=7,span=6,below=r1,
}{
	$$
    x^{a+b} = x^a \cdot x^b      \quad 
    \left(x^a\right)^b = x^{ab}  \qquad
    \sqrt{ab} = \sqrt{a}\sqrt{b} \quad \Ten
    \left(\sqrt{x^a}\right)^b = \sqrt{x^{ab}} = \left(\sqrt{x}\right)^{ab} = x^{\frac{ab}{2} } 
    $$
}
\posterbox[adjusted title={Wortels verdrijven uit de noemer}]{
	name=t1,column=7,span=6,below=l1,
}{
    Je kan noemers wortelvrij maken door te vermenigvuldigen met de \textit{toegevoegde tweeterm}:
    \begin{align*}
    \frac{1+\sqrt{2}}{1+\sqrt{3}} & = \frac{1+\sqrt{2}}{1+\sqrt{3}}\cdot\frac{1-\sqrt{3}}{1-\sqrt{3}} 
    =  \frac{(1+\sqrt{2})(1-\sqrt{3})}{(1+\sqrt{3})1-\sqrt{3}} 
    = \frac{1+6 +\sqrt{2}-\sqrt{3}}{1-3} \\
    & = -\frac{7+\sqrt{2}-\sqrt{3}}{2}=-\frac{7}{2} + \frac{\sqrt{2}+\sqrt{3}}{2}
    \end{align*}
}




\posterbox[adjusted title=De stelling van Pythagoras]{
	name=r1,column=7,span=6,below=t1
}{
%    Een driehoek $ABC$ is rechthoekig in hoek $C$ %  als en slechts als
%    \hfill $\iff$ \hfill $|AC|^2 + |CB|^2 = |AB|^2$.
%    \\ \\
\begin{minipage}[t]{0.75\textwidth}
    \centering
    Een driehoek met zijden $a,b,c$ is
    rechthoekig in de hoek tegenover $c$ \\
    $\iff$ 
    $\important{ a^2+b^2=c^2}$
\end{minipage}
\hfill
\begin{tikzpicture}[baseline={([yshift={-\ht\strutbox}]current bounding box.north)}, 
                    line cap=round,line join=round,>=triangle 45,
                    x=1cm,y=1cm,
                    scale=0.5]
     \draw[thick] (0,0) -- node[below]{$a$} (4,0) 
                        -- node[right]{$b$} (4,1.5)
                        -- node[above left]{$c$} (0,0) 
                        --cycle
    ;
    \draw (4,0) +(-0.4,0.2) -- +(-0.2,0.2) -- +(-0.2,0.4);
\end{tikzpicture}
\qquad
}


\posterbox[adjusted title=Goniometrie]{	
    name=l2,column=1,span=6,below=l1
}{
Een hoek $\alpha$ komt overeen met een uniek \textbf{beeldpunt} $P_\alpha$ op de \textbf{goniometrische cirkel}.
    
\begin{image}[0.8\textwidth]
	\begin{tikzpicture}[scale=2.5,baseline=(current bounding box.center)]%,cap=round,transform canvas={scale=0.5}]
	
	\tikzmath{\hoek = 35; \myc = cos(\hoek); \mys = sin(\hoek); 
		\hoekb = 20;}
	
	
	% Goniometrische cirkel
	\draw (0,0) circle (1cm);
	\draw[->] (-1.2,0) -- (1.5,0);% node[right] {$x$};
	\draw[->] (0,-1.2) -- (0,1.2);% node[above] {$y$};
	\draw  (0, 0) node [below left]  {$(0,0)$};
	\draw (1,0) node [below right] {$(0,1)$};
    % geen maateenheden op deze definitie; wel \phantoms om de alignering mertt de cirkel rechts te behouden    
%	\draw  ( 1, 0) node [above right] {$\color{blue}\alpha=0$};
	\draw  ( 0, -1) node [below left]  {\phantom{$\color{blue}\alpha=\frac{3\pi}{2}$}};	
	\path  ( 0, 1) node [above left]  {\phantom{$\color{blue}\alpha=\frac{\pi}{2}$}};		
%	\draw  ( -1,0) node [below left]  {$\color{blue}\alpha=\pi$};
	%
	\draw[color=blue,ultra thick] (1.3,0) -- (0:0)  -- (\hoek:1.3); 
	\draw[color=blue, ->] (0.3,0) arc (0:\hoek:0.3cm) node [midway,right] {$\alpha$};   
	%
	\draw[color=black] (\hoek:1) node[name=P,circle, fill=black, radius=1pt,scale=0.8] {} node [yshift=2pt,above] {\Large$P_\alpha$} ;  
	%
	%\draw[dashed] ({cos(\hoek)},0) node[circle, fill=black, radius=1pt,scale=0.5] {} node[below left] {$\cos\alpha$} -- (P);
	%\draw[dashed] (0,{sin(\hoek)}) node[circle, fill=black, radius=1pt,scale=0.5] {} node[left] {$\sin\alpha$} -- (P);
	%
	%\draw [thick, red,decorate,decoration={brace,amplitude=10pt,mirror},yshift=-5pt](0,0) -- ({cos(\hoek)},0) node[black,midway,yshift=-0.6cm] {\footnotesize $\cos\alpha$};
	%
	%\draw [thick, red,decorate,decoration={brace,amplitude=10pt},xshift=-10pt](0,0) -- (0,{sin(\hoek)}) node[black,midway,left,xshift=-8pt] {\footnotesize $\sin\alpha$};
	
	\end{tikzpicture}
	\quad\quad
	\begin{tikzpicture}[scale=2.5,baseline=(current bounding box.center)]%,cap=round,transform canvas={scale=0.5}]
	
	\tikzmath{\hoek = 20; \myc = cos(\hoek); \mys = sin(\hoek); 
		\hoekb = 20;}
	
	% Goniometrische cirkel
	\draw (0,0) circle (1cm);
	\draw[->] (-1.2,0) -- (1.2,0);% node[right] {$x$};
	\draw[->] (0,-1.2) -- (0,1.2);% node[above] {$y$};
	%	\draw  (0, 0) node [below left]  {$(0,0)$};
	\draw  ( 1, 0) node [below right] {$\color{blue}\alpha=0$};
	\draw  ( 0, -1) node [below left]  {$\color{blue}\alpha=\frac{3\pi}{2}$};	
	\draw  ( 0, 1) node [above left]  {$\color{blue}\alpha=\frac{\pi}{2}$};		
	\draw  ( -1,0) node [below left]  {$\color{blue}\alpha=\pi$};
	\draw[blue,dashed]  (0,0) -- (30:1) node [above right]  {$\color{blue}\alpha=\frac{\pi}{6}$};
	\draw[blue,dashed]  (0,0) -- (45:1) node [above right]  {$\color{blue}\alpha=\frac{\pi}{4}$};
	\draw[blue,dashed]  (0,0) -- (-45:1) node [below right,align=left]  {$\color{blue}\alpha=-\frac{\pi}{4}$\\$\alpha=\frac{7\pi}{4}$};
	\draw[blue,dashed]  (0,0) -- (60:1) node [above right]  {$\color{blue}\alpha=\frac{\pi}{3}$};
	\draw[blue,dashed]  (0,0) -- (135:1) node [above left]  {$\color{blue}\alpha=\frac{3\pi}{4}$};		
	%
	
	\draw[black]  ( 0.5, 0.5) node[fill=lightgray] {\Huge I};	
	\draw[black]  (-0.5, 0.5) node[fill=lightgray] {\Huge II};	
	\draw[black]  (-0.5,-0.5) node[fill=lightgray] {\Huge III};	
	\draw[black]  ( 0.5,-0.5) node[fill=lightgray] {\Huge IV};	
	
	\end{tikzpicture}
  
  
 % rechthoekige driehoek, met tangens en cotangens
\begin{tikzpicture}[scale=2.5,baseline=(current bounding box.center)]%,cap=round,transform canvas={scale=0.5}]
 
\tikzmath{\hoek = 35; \myc = cos(\hoek); \mys = sin(\hoek);
    \hoekb = 20;}
% Goniometrische cirkel
\draw (0,0) circle (1cm);
\draw[->] (-1.2,0) -- (1.2,0);% node[right] {$x$};
\draw[->] (0,-1.2) -- (0,1.2);% node[above] {$y$};
%   \draw  (0, 0) node [below left]  {$(0,0)$};
\draw  ( 1, 0) node [below right] {$ 1$};
\draw  (-1, 0) node [below left]  {$-1$};  
\draw  ( 0, 1) node [above left]  {$ 1$};      
\draw  ( 0,-1) node [below left]  {$-1$};
 
\draw  ( 1,-1) -- (1,1.2);  %node [right]  {$\tan\alpha$};
%\draw  ( -1,1)  node [above]  {$\cot\alpha$} -- (1.5,1) ;


%% =>>> volgende regel is horizontaal voor cot weg
%\draw  ( -1,1)  -- (1.5,1) ;
 
 
\draw (0:0)  -- (\hoek:1.4);
%\draw[color=blue,ultra thick] ({cos(\hoek)},0) -- (0:0)  -- (\hoek:1);
 
\draw[color=blue, ->] (0.3,0) arc (0:\hoek:0.3cm) node [midway,right] {$\alpha$};  
 
\draw[color=black] (\hoek:1) node[name=P,circle, fill=black, radius=1pt,scale=0.8] {} node [yshift=4pt,left] {$(x,y)$} ; 
 
\draw[blue, ultra thick] (0,0) -- ({cos(\hoek)},0);
\draw[blue, ultra thick] ({cos(\hoek)},0) node[circle, fill=black, radius=1pt,scale=0.5] {} node[below] {\color{black}$x$} -- (P);
\draw[dashed] (0,{sin(\hoek)}) node[circle, fill=black, radius=1pt,scale=0.5] {} node[left] {$y$} -- (P);
 
\draw [thick, blue,decorate,decoration={brace,amplitude=10pt,mirror},yshift=-5pt](0,0) -- ({cos(\hoek)},0) node[black,midway,yshift=-0.6cm] {\footnotesize $\cos\alpha$};
 
\draw [thick, blue,decorate,decoration={brace,amplitude=10pt},xshift=-4pt](0,0) -- (0,{sin(\hoek)}) node[black,midway,right,xshift=-1.1cm] {\footnotesize $\sin\alpha$};
 
\draw [very thick, red] (1,0) -- (1,{tan(\hoek)}) node[circle, fill=black, radius=1pt,scale=0.5] {};
\draw [thick, red,decorate,decoration={brace,amplitude=10pt,mirror},xshift=2pt](1,0) -- (1,{tan(\hoek)}) node[black,midway,right,xshift=8pt] {\footnotesize $\tan\alpha$};
 
%\draw [very thick, red] (0,1) --  ({cot(\hoek)},1) node[circle, fill=black, radius=1pt,scale=0.5] {};
%\draw [thick, red,decorate,decoration={brace,amplitude=10pt},yshift=2pt](0,1) -- ({cot(\hoek)},1) node[black,midway,above,yshift=8pt] {\footnotesize $\cot\alpha$};
 
 
\end{tikzpicture}
\end{image}

\begin{tabular}{ll}
De \textbf{cosinus} en \textbf{sinus} van $\alpha$ zijn per definitie de coördinaten van  $P_\alpha$:&
\important{P_\alpha = (\cos\alpha,\sin\alpha)} \\
De \textbf{tangens} van $\alpha$ is de verhouding van sinus tot cosinus: &
\important{\tan\alpha = \frac{\sin \alpha}{\cos \alpha}}.
\end{tabular}
% wat meetkundig tevens overeenkomt met het lijnsegment dat raakt aan de cirkel (zie figuur). \\ 

Voor scherpe hoeken volgt dan uit de stelling van Pythagoras:
% de meer gekende formules in een rechthoekige driehoek:
\newcommand{\Loverstaandezijde}{\nlentext{Overstaande zijde}{opposite side}}
\newcommand{\Laanliggendezijde}{\nlentext{Aanliggende zijde}{adjacent side}}
\newcommand{\Lschuinezijde}{\nlentext{Schuine zijde}{hypotenuse}}
$$
\begin{array}{rccl}
	 \important{\sin\alpha = \dfrac{a}{c}}  = \dfrac{\Loverstaandezijde}{\Lschuinezijde}      & \nlentext{SOS}{SOH} && \important{a = \sin\alpha \cdot c} \\
	 \important{\cos\alpha = \dfrac{b}{c}}  = \dfrac{\Laanliggendezijde}{\Lschuinezijde}      & \nlentext{CAS}{CAH} && \important{b = \cos\alpha \cdot c} \\
	 \important{\tan\alpha = \dfrac{a}{b}}  = \dfrac{\Loverstaandezijde}{\Laanliggendezijde}  & \text{TOA}          && \important{a = \tan\alpha \cdot b} \\
	% \important{\cot\alpha = \dfrac{b}{a} } & = \dfrac{1}{\tan\alpha} = \dfrac{\cos\alpha}{\sin\alpha} && b = \cot\alpha \cdot a & \Lcotangens 
\end{array}
$$

\textbf{Hoofdformule van de goniometrie}:
$\important{\cos^2\alpha + \sin^2 \alpha = 1}$.

}

\posterbox[adjusted title={Het reële vlak, met vectoren en afstand}]{
	name=r2,column=7,span=6,below=r1
}{

\begin{tikzpicture}[line cap=round,line join=round,>=triangle 45,x=0.7cm,y=0.7cm]
    \draw[->,thick] (-0.5,0) -- (8,0)node[below left]{$x$};
    \draw[->,thick] (0,-0.5) -- (0,5)node[below left]{$y$};
    \draw[thick,blue,fill=black] (2.5,4)  circle(2pt) node[above]{$Q$}
                         -- (2.5,1.5)circle(2pt) node[below left]{$R$} node(Y)[midway,left ]{$|y_2-y_1|$}
                         -- (6.5,1.5)circle(2pt) node[right]{$P$}      node(X)[midway,below]{$|x_2-x_1|$}
                         -- (2.5,4);
    \draw[line width=1pt,dash pattern=on 4pt off 3pt] (2.5,0) node[below]{$x_2$}|-(0,1.5)node[left]{$y_1$};
    \draw[line width=1pt,dash pattern=on 4pt off 3pt] (6.5,0) node[below]{$x_1$}--(6.5,1.5);
    \draw[line width=1pt,dash pattern=on 4pt off 3pt] (2.5,4)--(0,4)node[left]{$y_2$};
    \draw (2.7,1.9)|-(2.9,1.7);
    \draw[->] (Y)--(Y|-2.5,4);
    \draw[->] (Y)--(Y|-2.5,1.5);
    \draw[->] (X)--(X-|6.5,1.5);
    \draw[->] (X)--(X-|2.5,1.5);
    \draw (6,2.5) node[right] {\Large
$\small
\begin{array}{rl}
|PQ|^2 & = |PR|^2 + |RQ|^2 \\
       & = |x_2 - x_1|^2 + |y_2 - y_1|^2 \\
       & = (x_2 - x_1)^2 + (y_2 - y_1)^2
\end{array}    
$
};
\end{tikzpicture}

% \end{minipage}

De \textbf{afstand} tussen $P(x_1,y_1)$ en $Q(x_2,y_2)$ is
$ 
d(P,Q) \perdef |PQ| \perdef \important{\sqrt{ (x_2-x_1)^2 + (y_2 - y_1)^2}}
$.

}
\posterbox[adjusted title={Poolcoördinaten}]{
	name=l3,column=1,span=6,below=l2
}{

$P(a,b)$ heeft als poolcoördinaten $r=\sqrt{a2 ^+b^2}$ en $\theta=$ de hoek naar $(a,b)$.

}



\end{tcbposter}


% \begin{tcbposter}
% 	\xmpostertitle{Complexe getallen: bewerkingen}{span=4}
	
	
% 	\posterbox[adjusted title=Optellen / som]{
% 		name=l1,column=1,span=4,below=title
% 	}{
	
% 	%\textbf{Optellen / Som} 
	
% 	\begin{tikzpicture}[scale=1.5]
		
% 		\draw[thin,->] (-0.2,0)  -- (3.5,0) node[above] {$Re$};
% 		\draw[thin,->] (0,-0.2)  -- (0,2.3) node[right] {$Im$};
		
% 		\draw[blue,ultra thick, ->] (0,0) --  (3,2  ) coordinate (z1) node[above left]{$z_1=a+bi$};	
% 		\draw[blue,ultra thick, ->] (0,0) --  (2,0.5) coordinate (z2) node[below right]{$z_2=c+di$};	
		
% 		% \draw[color=red,ultra thick, ->] (0,0) -- (3,2);
% 		\draw[color=red,ultra thick, ->] (0,0) -- ($(z1)+(z2)$) node[above]{$z_1+z_2$};
% 		\draw[color=red] (0,3) node[right]{\important{z_1+z_2=(a+c)+(b+d)i}};
% 		\draw[dashed] (z2) -- ($(z1)+(z2)$) ;
% 		\draw[dashed,ultra thick, color=blue,->] (z1) -- ($(z1)+(z2)$);	
% 		%\draw (3,1.6) node[right] {\textbf{som}};	
		
% 	\end{tikzpicture}
	
% 	\begin{tabular}{ll}
% 	\textbf{Slagzin}: & Som van de reële delen, \\
% 	                  & som van de imaginaire delen.\\
% 	\textbf{Grafisch}:& Pijlen na elkaar zetten. \\
% 	\textbf{Polair}:  & Omzetten naar cartesiaans.
% 	%% \textbf{Polair}:& $re^{i\alpha}+se^{i\beta} = ???$ (geen eenvoudige formule).

% 	\end{tabular}
	
% 	}
% 	\posterbox[adjusted title=Aftrekken / verschil]{
% 		name=l2,column=1,span=4,below=l1    % could be r1, but must be replaced !
% 	}{
		
% 	%\textbf{Aftrekken}
	
	
% 	\begin{tikzpicture}[scale=1.5]
		
% 		\draw[thin,->] (-0.2,0)  -- (3.5,0) node[above] {$Re$};
% 		\draw[thin,->] (0,-0.2)  -- (0,2.3) node[right] {$Im$};
		
% 		\draw[blue,ultra thick, ->] (0,0) -- (3,2  ) coordinate (z1) node[right]{$z_1=a+bi$};
% 		\draw[blue,ultra thick, ->] (0,0) -- (2,0.5) coordinate (z2) node[below right]{$z_2=c+di$};

% %		\draw[dashed] (1,1.5) -- (3,2) ;
% 		\draw[color=red,ultra thick, ->] (0,0) -- ($(z1)-(z2)$) node[above] {$z_1-z_2$};
% 		\draw[dashed, color=blue,ultra thick, ->] (z1) -- node[above left] {$-z_2$} ($(z1)-(z2)$);
% 		\draw[color=red] (0,3) node[right]{\important{z_1-z_2=(a-c)+(b-d)i}};
% 		%\draw[color=red,ultra thick, ->] (2,0.5) -- ($(z1)-(z2)$);	
% 		%\draw[color=teal,ultra thick, ->] (0,0) -- (1,1.5);	
% 		%\draw (2.7,1) node[right] {\textbf{verschil}};	
		
% 		% \draw[color=blue,<-] (1,1.5) -- (3,2) ;
% %		
% 		\draw[color=teal,thin,dashed,->] (z2)  -- ($(z2)+(1.5,0)$) node[above] {$Re$};
% 		\draw[color=teal,thin,dashed,->] (z2)  -- ($(z2)+(0,1.9)$) node[right] {$Im$};
% 		\draw[color=red,thick,dashed,->] (z2)  -- (z1) node[right] {};
		
		
% 	\end{tikzpicture}
	
% 	\begin{tabular}{ll}
% 		\textbf{Slagzin}: & Verschil van de reële delen, \\
% 						  & verschil van de imaginaire delen.\\
% 		\textbf{Grafisch}:& Eindpunten met mekaar verbinden. \\
% 		\textbf{Polair}:  & Omzetten naar cartesiaans.
% 	\end{tabular}
	
% }
	
% \posterbox[adjusted title=Vermenigvuldigen / product]{
% 	name=r1,column=5,span=4,below=title
% }{	
	
	
% %	\begin{minipage}[t]{12cm}
% %		$(a+bi)(c+di)=(ac-bd)+(ad+bc)i$\\
% %		$ r(\cos\alpha+i\sin\alpha) \cdot s(\cos  \beta+i\sin\beta)=rs(\cos(\alpha+\beta)+i\sin(\alpha+\beta))
% %		$\\
% %		$
% %		re^{i\alpha}se^{i\beta}=rse^{i(\alpha+\beta)}
% %		$
% %	\end{minipage}

	
		
% 	\begin{tikzpicture}[scale=1.5]%,cap=round,transform canvas={scale=0.5}]
		
% 		\draw[thin,->] (-0.2,0)  -- (2.5,0) node[above] {$Re$};
% 		\draw[thin,->] (0,-0.2)  -- (0,2.2) node[right] {$Im$};
		
		
% 		\draw[blue,ultra thick, ->] (0,0) -- (1,0.3)node[right] {$z_1=re^{i \alpha}$}; 
% 		\draw[blue,ultra thick, ->] (0,0) -- (2,1.5)node[right] {$z_2=se^{i \beta}$}; 
% 		\draw[color=red, ultra thick, ->] (0,0) -- node[left] {$rs$} (1.7,2.1) node[right] {$z_1z_2$};
% 		\draw[color=red] (5,2.3) node[left] {\important{z_1z_2=rse^{i (\alpha+\beta)}}}; 	
% 		% \draw (1.7,1.7) node[right] {\textbf{produkt}};
		
% 		\draw[->] (0.6,0) arc (0:atan(0.3):0.6) node [midway,right] {$\alpha$}; 
% 		\draw[color=teal,->] (0.766,0.230) arc (atan(0.3):atan(2.1/1.7):0.76) node [midway,right] {$\beta$}; 		
% 	\end{tikzpicture}
	
% 	\begin{tabular}{ll}
% 		\textbf{Slagzin}: & Product van de moduli, \\
% 						  & som van de argumenten.\\
% 		\textbf{Grafisch}:& modulus=schaalfactor, \\ 
% 		                  & argument=rotatiehoek (fasedraaiing) \\
% 		\textbf{Cartesiaans}: & \important{(a+bi)(c+di)=(ac-bd)+(ad+bc)i}  \\
% 		                      & dus: uitwerken met $i^2=-1$
% 	\end{tabular}

	
	
% %	\hspace{1cm}\textbf{polair/exponentieel}
	
% }

% \posterbox[adjusted title=Delen / quotiënt]{
% 	name=r2,column=5,span=4,below=r1
% }{	

% %	\begin{minipage}[t]{12cm}
% %		$\displaystyle{
% %			\frac{a+bi}{c+di}=\frac{(a+bi)(c-di)}{(c+di)(c-di)}=\frac{ac+bd}{c^2+d^2}+\frac{bc-ad}{c^2+d^2}i
% %		}$\\
% %		$\displaystyle{
% %			\frac{r(\cos(\alpha)+\sin(\alpha)i)}{s(\cos(\beta)+\sin(\beta)i)}=\frac{r}{s}(\cos(\alpha-\beta)+\sin(\alpha-\beta)i)
% %		}$\\
% %		$\displaystyle{
% %			\frac{r e^{i\alpha}}{s e^{i\beta}} = \frac{r}{s} e^{i(\alpha - \beta)}
% %		}$
% %	\end{minipage}


% 	\begin{tikzpicture}[scale=1.5]%,cap=round,transform canvas={scale=0.5}]
	
% 	\draw[color=red,ultra thick, ->] (0,0) -- (1,0.3) node[right] {$\dfrac{z}{z_2}$};
% 	\draw[red]  (5,2.3) node[left] {\important{\frac{z_1}{z_2}=\frac{r}{s}e^{i (\alpha-\beta)}}}; 
% 	\draw[blue,ultra thick, ->] (0,0) -- (2,1.5) node[right] {$z_2=se^{i \beta}$}; 
% 	\draw[blue,ultra thick, ->] (0,0) -- (1.7,2.1) node[right] {$z_1=re^{i \alpha}$};

% %	\draw (1.7,1.7) node[right] ;
% %	
% 	\draw[->] (0.4,0) arc (0:atan(2.1/1.7):0.4) node [midway,right] {$\alpha$}; 
% 	\draw[color=teal,<-] (0.766,0.230) arc (atan(0.3):atan(2.1/1.7):0.76) node [midway,right] {$-\beta$}; 	

% 	\draw[thin,->] (-0.2,0)  -- (2.5,0) node[below] {$Re$};
% 	\draw[thin,->] (0,-0.2)  -- (0,2.5) node[right] {$Im$};	
% 	\end{tikzpicture}

% 	\begin{tabular}{@{}l@{}l@{}}
% 		\textbf{Slagzin}: & Quotiënt van de moduli, \\
% 						  & verschil van de argumenten.\\
% 		\textbf{Grafisch}:& modulus=schaalfactor, \\ 
% 		                  & argument=rotatiehoek (fasedraaiing) \\
% 		\textbf{Cartesiaans}: & \important{\frac{a+bi}{c+di}=\dfrac{(a+bi)\blue{(c-di)}}{(c+di)\blue{(c-di)}}=\dfrac{ac+bd}{c^2+d^2}+\frac{bc-ad}{c^2+d^2}i} \\
% 		                      & dus: teller en noemer maal \\
% 							  & complex toegevoegde van noemer.
% 	\end{tabular}

% % \hspace{1cm}\textbf{polair/exponentieel}
% }


% \posterbox[adjusted title=Machten]{
% 		name=x1,column=9,span=4,below=title
% }{	
	

% %	\begin{minipage}[t]{12cm}
% %	$r(\cos(\theta)+\sin(\theta)i)^n=r^n(\cos(n\theta)+\sin(n\theta)i)$
% %	
% %	$(re^{i\theta})^n=r^n e^{in\theta}$
% %	\end{minipage}
	
% 	\begin{tikzpicture}[scale=1.5]%,cap=round,transform canvas={scale=0.5}]
	

% 	\draw [blue, ->, ultra thick,rotate=10] (0,0) --(1.3,0) node [right] {$z=re^{i\theta}$};
	
% 	\foreach \i in {2,3,...,6} 
% 		\draw [->, ultra thick,color=red,rotate=\i*10] (0,0) -- node [color=teal, above, rotate=\i*10,pos=0.9] {\small $r^{\i}$} (1.3^\i,0) node [right, black] {$z^{\i} = r^{\i}e^{i\i\theta}$};

% 		% \draw [->, ultra thick,color=teal,rotate=20] (0,0) --(1.69,0) node [right] {$z^2$};
% 		% \draw [->, ultra thick,color=teal,rotate=30] (0,0) --(2.197,0) node [right] {$z^3$};
% 		% \draw [->, ultra thick,color=teal,rotate=40] (0,0) --(2.856,0) node [right] {$z^4$};
% 		% \draw [->, ultra thick,color=teal,rotate=50] (0,0) --(3.713,0) node [right] {$z^5$};
% 		% \draw [->, ultra thick,color=teal,rotate=60] (0,0) --(4.827,0) node [right] {$z^6$};
% 	\draw[thin,->] (-0.2,0)  -- (2.5,0) node[below] {$Re$};
% 	\draw[thin,->] (0,-0.2)  -- (0,3) node[right] {$Im$};	
	
% 	\draw[red] (0,4.5) node[right] {\important{z^n=(re^{i\theta})^n=r^n e^{in\theta}}};
% \end{tikzpicture}


% \begin{tabular}{ll}
% 	\textbf{Slagzin}: & Modulus tot de \(n\)-de macht, \\ 
% 					  & argument maal \(n\) \\
% 	\textbf{Grafisch}:& modulus=schaalfactor, \\ 
% 					  & argument=rotatiehoek (fasedraaiing) \\
% 	\textbf{Cartesiaans}:& uitwerken (Binomium van Newton) \\ 
% 	                     & of omzetten naar polair \\ 
% \end{tabular}
% }

% \posterbox[adjusted title=Wortels]{
% 		name=x2,column=9,span=4,below=x1
% }{	

% 		$z=re^{i\theta}$ heeft precies $n$ n-demachtswortels:\\
% 		 $x_k = \sqrt[n]{r}e^{i\frac{\theta + k 2 \pi}{n}}$,  met $k=0,1,\ldots,n-1$

% % \url{https://www.geogebra.org/m/By0RL6MI}

% \centering

% \scalebox{1}{
% 	\begin{tikzpicture}[scale=1.5,baseline=(current bounding box.center)]
	
% 	\tikzmath{\hoek = 20; \myc = cos(\hoek); \mys = sin(\hoek); 
% 		\hoekb = 20;}
	
	
% 	% Goniometrische cirkel
% 	\draw (0,0) circle (1cm);
% 	\draw[gray,->] (-1.2,0) -- (1.2,0);% node[right] {$x$};
% 	\draw[gray,->] (0,-1.2) -- (0,1.2);% node[above] {$y$};
% 	%	\draw  (0, 0) node [below left]  {$(0,0)$};
% 	\draw  (0,0) -- ( 0:1 ) node [align=right,above right] {$\blue{e^{0}} = 1$};
% 	\draw  (0,0) -- ( 72:1 ) node [align=right,above right] {$\blue{e^{i2\pi/5}}$};
% 	\draw  (0,0) -- ( 144:1 ) node [align=right,above left] {$\blue{e^{i4\pi/5}}$};
% 	\draw  (0,0) -- ( -144:1 ) node [align=right,below left] {$\blue{e^{i6\pi/5}}$};
% 	\draw  (0,0) -- ( -72:1 ) node [align=right,below right] {$\blue{e^{i8\pi/5}}$};
% 	\node at (0:1) [circle, fill=black, scale=0.6]{};  
% 	\node at (72:1) [circle, fill=black, scale=0.6]{};  
% 	\node at (144:1) [circle, fill=black, scale=0.6]{};  
% 	\node at (-144:1) [circle, fill=black, scale=0.6]{};  
% 	\node at (-72:1) [circle, fill=black, scale=0.6]{};  
% 	\end{tikzpicture}
% }

% Er zijn vijf 5-de wortels van $1$, namelijk $\important{e^{i2k\pi/5}}, k=0,1,2,3,4$.
% }

% \end{tcbposter}

\end{document}
