\documentclass{ximera}
%%% Begin Laad packages

\makeatletter
\@ifclassloaded{xourse}{%
    \typeout{Start loading preamble.tex (in a XOURSE)}%
    \def\isXourse{true}   % automatically defined; pre 112022 it had to be set 'manually' in a xourse
}{%
    \typeout{Start loading preamble.tex (NOT in a XOURSE)}%
}
\makeatother

\pgfplotsset{compat=1.16}

\usepackage{currfile}

% 201908/202301: PAS OP: babel en doclicense lijken problemen te veroorzaken in .jax bestand
% (wegens syntax error met toegevoegde \newcommands ...)
\pdfOnly{
    \usepackage[type={CC},modifier={by-nc-sa},version={4.0}]{doclicense}
    \usepackage[dutch]{babel}
}



\usepackage[utf8]{inputenc}
\usepackage{morewrites}   % nav zomercursus (answer...?)
\usepackage{multirow}
\usepackage{multicol}
\usepackage{tikzsymbols}
\usepackage{ifthen}
%\usepackage{animate} BREAKS HTML STRUCTURE USED BY XIMERA
\usepackage{relsize}

\usepackage{eurosym}    % \euro  (€ werkt niet in xake ...?)

% Nuttig als ook interactieve beamer slides worden voorzien:
\providecommand{\p}{} % default nothing ; potentially usefull for slides: redefine as \pause
%providecommand{\p}{\pause}

\usepackage{caption} % captionof
%\usepackage{pdflscape}    % landscape environment

% Met "\newcommand\showtodonotes{}" kan je todonotes tonen (in pdf/online)
% 201908: online werkt het niet (goed)
\providecommand\showtodonotes{disable}
\providecommand\todo[1]{\typeout{TODO #1}}
%\usepackage[\showtodonotes]{todonotes}
%\usepackage{todonotes}

%
% Poging tot aanpassen layout
%
\usepackage{tcolorbox}
\tcbuselibrary{theorems}

%%% Einde laad packages

%%% Begin Ximera specifieke zaken

\graphicspath{
	{../../}
	{../}
	{./}
  	{../../pictures/}
   	{../pictures/}
   	{./pictures/}
	{./explog/}    % M05 in groeimodellen       
}

%%% Einde Ximera specifieke zaken

%
% define softer blue/red/green, use KU Leuven base colors for blue (and dark orange for red ?)
%
% todo: rather redefine blue/red/green ...?
%\definecolor{xmblue}{rgb}{0.01, 0.31, 0.59}
%\definecolor{xmred}{rgb}{0.89, 0.02, 0.17}
\definecolor{xmdarkblue}{rgb}{0.122, 0.671, 0.835}   % KU Leuven Blauw
\definecolor{xmblue}{rgb}{0.114, 0.553, 0.69}        % KU Leuven Blauw
\definecolor{xmgreen}{rgb}{0.13, 0.55, 0.13}         % No KULeuven variant for green found ...

\definecolor{xmaccent}{rgb}{0.867, 0.541, 0.18}      % KU Leuven Accent (orange ...)
\definecolor{kuaccent}{rgb}{0.867, 0.541, 0.18}      % KU Leuven Accent (orange ...)

\colorlet{xmred}{xmaccent!50!black}                  % Darker version of KU Leuven Accent

\providecommand{\blue}[1]{{\color{blue}#1}}    
\providecommand{\red}[1]{{\color{red}#1}}

\renewcommand\CancelColor{\color{xmaccent!50!black}}

% werkt in math en text mode om MATH met oranje (of grijze...)  achtergond te tonen (ook \important{\text{blabla}} lijkt te werken)
%\newcommand{\important}[1]{\ensuremath{\colorbox{xmaccent!50!white}{$#1$}}}   % werkt niet in Mathjax
%\newcommand{\important}[1]{\ensuremath{\colorbox{lightgray}{$#1$}}}
\newcommand{\important}[1]{\ensuremath{\colorbox{orange}{$#1$}}}   % TODO: kleur aanpassen voor mathjax; wordt overschreven infra!


% Uitzonderlijk kan met \pdfnl in de PDF een newline worden geforceerd, die online niet nodig/nuttig is omdat daar de regellengte hoe dan ook niet gekend is.
\ifdefined\HCode%
\providecommand{\pdfnl}{}%
\else%
\providecommand{\pdfnl}{%
  \\%
}%
\fi

% Uitzonderlijk kan met \handoutnl in de handout-PDF een newline worden geforceerd, die noch online noch in de PDF-met-antwoorden nuttig is.
\ifdefined\HCode
\providecommand{\handoutnl}{}
\else
\providecommand{\handoutnl}{%
\ifhandout%
  \nl%
\fi%
}
\fi



% \cellcolor IGNORED by tex4ht ?
% \begin{center} seems not to wordk
    % (missing margin-left: auto;   on tabular-inside-center ???)
%\newcommand{\importantcell}[1]{\ensuremath{\cellcolor{lightgray}#1}}  %  in tabular; usablility to be checked
\providecommand{\importantcell}[1]{\ensuremath{#1}}     % no mathjax2 support for colloring array cells

\pdfOnly{
  \renewcommand{\important}[1]{\ensuremath{\colorbox{kuaccent!50!white}{$#1$}}}
  \renewcommand{\importantcell}[1]{\ensuremath{\cellcolor{kuaccent!40!white}#1}}   
}

%%% Tikz styles


\pgfplotsset{compat=1.16}

\usetikzlibrary{trees,positioning,arrows,fit,shapes,math,calc,decorations.markings,through,intersections,patterns,matrix}

\usetikzlibrary{decorations.pathreplacing,backgrounds}    % 5/2023: from experimental


\usetikzlibrary{angles,quotes}

\usepgfplotslibrary{fillbetween} % bepaalde_integraal
\usepgfplotslibrary{polar}    % oa voor poolcoordinaten.tex

\pgfplotsset{ownstyle/.style={axis lines = center, axis equal image, xlabel = $x$, ylabel = $y$, enlargelimits}} 

\pgfplotsset{
	plot/.style={no marks,samples=50}
}

\newcommand{\xmPlotsColor}{
	\pgfplotsset{
		plot1/.style={darkgray,no marks,samples=100},
		plot2/.style={lightgray,no marks,samples=100},
		plotresult/.style={blue,no marks,samples=100},
		plotblue/.style={blue,no marks,samples=100},
		plotred/.style={red,no marks,samples=100},
		plotgreen/.style={green,no marks,samples=100},
		plotpurple/.style={purple,no marks,samples=100}
	}
}
\newcommand{\xmPlotsBlackWhite}{
	\pgfplotsset{
		plot1/.style={black,loosely dashed,no marks,samples=100},
		plot2/.style={black,loosely dotted,no marks,samples=100},
		plotresult/.style={black,no marks,samples=100},
		plotblue/.style={black,no marks,samples=100},
		plotred/.style={black,dotted,no marks,samples=100},
		plotgreen/.style={black,dashed,no marks,samples=100},
		plotpurple/.style={black,dashdotted,no marks,samples=100}
	}
}


\newcommand{\xmPlotsColorAndStyle}{
	\pgfplotsset{
		plot1/.style={darkgray,no marks,samples=100},
		plot2/.style={lightgray,no marks,samples=100},
		plotresult/.style={blue,no marks,samples=100},
		plotblue/.style={xmblue,no marks,samples=100},
		plotred/.style={xmred,dashed,thick,no marks,samples=100},
		plotgreen/.style={xmgreen,dotted,very thick,no marks,samples=100},
		plotpurple/.style={purple,no marks,samples=100}
	}
}


%\iftikzexport
\xmPlotsColorAndStyle
%\else
%\xmPlotsBlackWhite
%\fi
%%%


%
% Om venndiagrammen te arceren ...
%
\makeatletter
\pgfdeclarepatternformonly[\hatchdistance,\hatchthickness]{north east hatch}% name
{\pgfqpoint{-1pt}{-1pt}}% below left
{\pgfqpoint{\hatchdistance}{\hatchdistance}}% above right
{\pgfpoint{\hatchdistance-1pt}{\hatchdistance-1pt}}%
{
	\pgfsetcolor{\tikz@pattern@color}
	\pgfsetlinewidth{\hatchthickness}
	\pgfpathmoveto{\pgfqpoint{0pt}{0pt}}
	\pgfpathlineto{\pgfqpoint{\hatchdistance}{\hatchdistance}}
	\pgfusepath{stroke}
}
\pgfdeclarepatternformonly[\hatchdistance,\hatchthickness]{north west hatch}% name
{\pgfqpoint{-\hatchthickness}{-\hatchthickness}}% below left
{\pgfqpoint{\hatchdistance+\hatchthickness}{\hatchdistance+\hatchthickness}}% above right
{\pgfpoint{\hatchdistance}{\hatchdistance}}%
{
	\pgfsetcolor{\tikz@pattern@color}
	\pgfsetlinewidth{\hatchthickness}
	\pgfpathmoveto{\pgfqpoint{\hatchdistance+\hatchthickness}{-\hatchthickness}}
	\pgfpathlineto{\pgfqpoint{-\hatchthickness}{\hatchdistance+\hatchthickness}}
	\pgfusepath{stroke}
}
%\makeatother

\tikzset{
    hatch distance/.store in=\hatchdistance,
    hatch distance=10pt,
    hatch thickness/.store in=\hatchthickness,
   	hatch thickness=2pt
}

\colorlet{circle edge}{black}
\colorlet{circle area}{blue!20}


\tikzset{
    filled/.style={fill=green!30, draw=circle edge, thick},
    arceerl/.style={pattern=north east hatch, pattern color=blue!50, draw=circle edge},
    arceerr/.style={pattern=north west hatch, pattern color=yellow!50, draw=circle edge},
    outline/.style={draw=circle edge, thick}
}




%%% Updaten commando's
\def\hoofding #1#2#3{\maketitle}     % OBSOLETE ??

% we willen (bijna) altijd \geqslant ipv \geq ...!
\newcommand{\geqnoslant}{\geq}
\renewcommand{\geq}{\geqslant}
\newcommand{\leqnoslant}{\leq}
\renewcommand{\leq}{\leqslant}

% Todo: (201908) waarom komt er (soms) underlined voor emph ...?
\renewcommand{\emph}[1]{\textit{#1}}

% API commando's

\newcommand{\ds}{\displaystyle}
\newcommand{\ts}{\textstyle}  % tegenhanger van \ds   (Ximera zet PER  DEFAULT \ds!)

% uit Zomercursus-macro's: 
\newcommand{\bron}[1]{\begin{scriptsize} \emph{#1} \end{scriptsize}}     % deprecated ...?


%definities nieuwe commando's - afkortingen veel gebruikte symbolen
\newcommand{\R}{\ensuremath{\mathbb{R}}}
\newcommand{\Rnul}{\ensuremath{\mathbb{R}_0}}
\newcommand{\Reen}{\ensuremath{\mathbb{R}\setminus\{1\}}}
\newcommand{\Rnuleen}{\ensuremath{\mathbb{R}\setminus\{0,1\}}}
\newcommand{\Rplus}{\ensuremath{\mathbb{R}^+}}
\newcommand{\Rmin}{\ensuremath{\mathbb{R}^-}}
\newcommand{\Rnulplus}{\ensuremath{\mathbb{R}_0^+}}
\newcommand{\Rnulmin}{\ensuremath{\mathbb{R}_0^-}}
\newcommand{\Rnuleenplus}{\ensuremath{\mathbb{R}^+\setminus\{0,1\}}}
\newcommand{\N}{\ensuremath{\mathbb{N}}}
\newcommand{\Nnul}{\ensuremath{\mathbb{N}_0}}
\newcommand{\Z}{\ensuremath{\mathbb{Z}}}
\newcommand{\Znul}{\ensuremath{\mathbb{Z}_0}}
\newcommand{\Zplus}{\ensuremath{\mathbb{Z}^+}}
\newcommand{\Zmin}{\ensuremath{\mathbb{Z}^-}}
\newcommand{\Znulplus}{\ensuremath{\mathbb{Z}_0^+}}
\newcommand{\Znulmin}{\ensuremath{\mathbb{Z}_0^-}}
\newcommand{\C}{\ensuremath{\mathbb{C}}}
\newcommand{\Cnul}{\ensuremath{\mathbb{C}_0}}
\newcommand{\Cplus}{\ensuremath{\mathbb{C}^+}}
\newcommand{\Cmin}{\ensuremath{\mathbb{C}^-}}
\newcommand{\Cnulplus}{\ensuremath{\mathbb{C}_0^+}}
\newcommand{\Cnulmin}{\ensuremath{\mathbb{C}_0^-}}
\newcommand{\Q}{\ensuremath{\mathbb{Q}}}
\newcommand{\Qnul}{\ensuremath{\mathbb{Q}_0}}
\newcommand{\Qplus}{\ensuremath{\mathbb{Q}^+}}
\newcommand{\Qmin}{\ensuremath{\mathbb{Q}^-}}
\newcommand{\Qnulplus}{\ensuremath{\mathbb{Q}_0^+}}
\newcommand{\Qnulmin}{\ensuremath{\mathbb{Q}_0^-}}

\newcommand{\perdef}{\overset{\mathrm{def}}{=}}
\newcommand{\pernot}{\overset{\mathrm{notatie}}{=}}
\newcommand\perinderdaad{\overset{!}{=}}     % voorlopig gebruikt in limietenrekenregels
\newcommand\perhaps{\overset{?}{=}}          % voorlopig gebruikt in limietenrekenregels

\newcommand{\degree}{^\circ}


\DeclareMathOperator{\dom}{dom}     % domein
\DeclareMathOperator{\codom}{codom} % codomein
\DeclareMathOperator{\bld}{bld}     % beeld
\DeclareMathOperator{\graf}{graf}   % grafiek
\DeclareMathOperator{\rico}{rico}   % richtingcoëfficient
\DeclareMathOperator{\co}{co}       % coordinaat
\DeclareMathOperator{\gr}{gr}       % graad

\newcommand{\func}[5]{\ensuremath{#1: #2 \rightarrow #3: #4 \mapsto #5}} % Easy to write a function


% Operators
\DeclareMathOperator{\bgsin}{bgsin}
\DeclareMathOperator{\bgcos}{bgcos}
\DeclareMathOperator{\bgtan}{bgtan}
\DeclareMathOperator{\bgcot}{bgcot}
\DeclareMathOperator{\bgsinh}{bgsinh}
\DeclareMathOperator{\bgcosh}{bgcosh}
\DeclareMathOperator{\bgtanh}{bgtanh}
\DeclareMathOperator{\bgcoth}{bgcoth}

% Oude \Bgsin etc deprecated: gebruik \bgsin, en herdefinieer dat als je Bgsin wil!
%\DeclareMathOperator{\cosec}{cosec}    % not used? gebruik \csc en herdefinieer

% operatoren voor differentialen: to be verified; 1/2020: inconsequent gebruik bij afgeleiden/integralen
\renewcommand{\d}{\mathrm{d}}
\newcommand{\dx}{\d x}
\newcommand{\dd}[1]{\frac{\mathrm{d}}{\mathrm{d}#1}}
\newcommand{\ddx}{\dd{x}}

% om in voorbeelden/oefeningen de notatie voor afgeleiden te kunnen kiezen
% Usage: \afg{(2\sin(x))}  (en wordt d/dx, of accent, of D )
\newcommand{\afg}[1]{{#1}'}
%\renewcommand{\afg}[1]{\frac{\mathrm{d}#1}{\mathrm{d}x}}   % include in relevant exercises ...
%\renewcommand{\afg}[1]{D{#1}}

%
% \xmxxx commands: Extra KU Leuven functionaliteit van, boven of naast Ximera
%   ( Conventie 8/2019: xm+nederlandse omschrijving, maar is niet consequent gevolgd, en misschien ook niet erg handig !)
%
% (Met een minimale ximera.cls en preamble.tex zou een bruikbare .pdf moeten kunnen worden gemaakt van eender welke ximera)
%
% Usage: \xmtitle[Mijn korte abstract]{Mijn titel}{Mijn abstract}
% Eerste command na \begin{document}:
%  -> definieert de \title
%  -> definieert de abstract
%  -> doet \maketitle ( dus: print de hoofding als 'chapter' of 'sectie')
% Optionele parameter geeft eenn kort abstract (die met de globale setting \xmshortabstract{} al dan niet kan worden geprint.
% De optionele korte abstract kan worden gebruikt voor pseudo-grappige abtsarts, dus dus globaal al dan niet kunnen worden gebuikt...
% Globale settings:
%  de (optionele) 'korte abstract' wordt enkele getoond als \xmshortabstract is gezet
\providecommand\xmshortabstract{} % default: print (only!) short abstract if present
\providecommand\theabstract{} % otherwise complaint Undefined control sequence.  <recently read> \theabstract  ????
\newcommand{\xmtitle}[3][]{
	\title{#2}
	% \begin{abstract}
	% 			\ifdefined\xmshortabstract
	% 			\ifstrempty{#1}{%
	% 						#3
	% 			}{%
	% 						#1
	% 			}%
	% 			\else
	% 			#3
	% 			\fi
	% \end{abstract}
	\maketitle
}

% 
% Kleine grapjes: moeten zonder verder gevolg kunnen worden verwijderd
%
%\newcommand{\xmopje}[1]{{\small#1{\reversemarginpar\marginpar{\Smiley}}}}   % probleem in floats!!
\newtoggle{showxmopje}
\toggletrue{showxmopje}

\newcommand{\xmopje}[1]{%
   \iftoggle{showxmopje}{#1}{}%
}


% -> geef een abstracte-formule-met-rechts-een-concreet-voorbeeld
% VB:  \formulevb{a^2+b^2=c^2}{3^2+4^2=5^2}
%
\ifdefined\HCode
\NewEnviron{xmdiv}[1]{\HCode{\Hnewline<div class="#1">\Hnewline}\BODY{\HCode{\Hnewline</div>\Hnewline}}}
\else
\NewEnviron{xmdiv}[1]{\BODY}
\fi

\providecommand{\formulevb}[2]{
	{\centering

    \begin{xmdiv}{xmformulevb}    % zie css voor online layout !!!
	\begin{tabular}{lcl}
		\important{#1}
		&  &
		Vb: $#2$
		\end{tabular}
	\end{xmdiv}

	}
}

\ifdefined\HCode
\providecommand{\xmcolorbox}[2]{
	\HCode{\Hnewline<div class="xmcolorbox">\Hnewline}#2\HCode{\Hnewline</div>\Hnewline}
}
\else
\providecommand{\xmcolorbox}[2]{
  \cellcolor{#1}#2
}
\fi


\ifdefined\HCode
\providecommand{\xmopmerking}[1]{
 \HCode{\Hnewline<div class="xmopmerking">\Hnewline}#1\HCode{\Hnewline</div>\Hnewline}
}
\else
\providecommand{\xmopmerking}[1]{
	{\footnotesize #1}
}
\fi
% \providecommand{\voorbeeld}[1]{
% 	\colorbox{blue!10}{$#1$}
% }



% Hernoem Proof naar Bewijs, nodig voor HTML versie
\renewcommand*{\proofname}{Bewijs}

% Om opgave van oefening (wordt niet geprint bij oplossingenblad)
% (to be tested test)
\NewEnviron{statement}{\BODY}

% Environment 'oplossing' en 'uitkomst'
% voor resp. volledige 'uitwerking' dan wel 'enkel eindresultaat'
% geimplementeerd via feedback, omdat er in de ximera-server adhoc feedback-code is toegevoegd
%% Niet tonen indien handout
%% Te gebruiken om volledige oplossingen/uitwerkingen van oefeningen te tonen
%% \begin{oplossing}        De optelling is commutatief \end{oplossing}  : verschijnt online enkel 'op vraag'
%% \begin{oplossing}[toon]  De optelling is commutatief \end{oplossing}  : verschijnt steeds onmiddellijk online (bv te gebruiken bij voorbeelden) 

\ifhandout%
    \NewEnviron{oplossing}[1][onzichtbaar]%
    {%
    \ifthenelse{\equal{\detokenize{#1}}{\detokenize{toon}}}
    {
    \def\PH@Command{#1}% Use PH@Command to hold the content and be a target for "\expandafter" to expand once.

    \begin{trivlist}% Begin the trivlist to use formating of the "Feedback" label.
    \item[\hskip \labelsep\small\slshape\bfseries Oplossing% Format the "Feedback" label. Don't forget the space.
    %(\texttt{\detokenize\expandafter{\PH@Command}}):% Format (and detokenize) the condition for feedback to trigger
    \hspace{2ex}]\small%\slshape% Insert some space before the actual feedback given.
    \BODY
    \end{trivlist}
    }
    {  % \begin{feedback}[solution]   \BODY     \end{feedback}  }
    }
    }    
\else
% ONLY for HTML; xmoplossing is styled with css, and is not, and need not be a LaTeX environment
% THUS: it does NOT use feedback anymore ...
%    \NewEnviron{oplossing}{\begin{expandable}{xmoplossing}{\nlen{Toon uitwerking}{Show solution}}{\BODY}\end{expandable}}
    \newenvironment{oplossing}[1][onzichtbaar]
   {%
       \begin{expandable}{xmoplossing}{}
   }
   {%
   	   \end{expandable}
   } 
%     \newenvironment{oplossing}[1][onzichtbaar]
%    {%
%        \begin{feedback}[solution]   	
%    }
%    {%
%    	   \end{feedback}
%    } 
\fi

\ifhandout%
    \NewEnviron{uitkomst}[1][onzichtbaar]%
    {%
    \ifthenelse{\equal{\detokenize{#1}}{\detokenize{toon}}}
    {
    \def\PH@Command{#1}% Use PH@Command to hold the content and be a target for "\expandafter" to expand once.

    \begin{trivlist}% Begin the trivlist to use formating of the "Feedback" label.
    \item[\hskip \labelsep\small\slshape\bfseries Uitkomst:% Format the "Feedback" label. Don't forget the space.
    %(\texttt{\detokenize\expandafter{\PH@Command}}):% Format (and detokenize) the condition for feedback to trigger
    \hspace{2ex}]\small%\slshape% Insert some space before the actual feedback given.
    \BODY
    \end{trivlist}
    }
    {  % \begin{feedback}[solution]   \BODY     \end{feedback}  }
    }
    }    
\else
\ifdefined\HCode
   \newenvironment{uitkomst}[1][onzichtbaar]
    {%
        \begin{expandable}{xmuitkomst}{}%
    }
    {%
    	\end{expandable}%
    } 
\else
  % Do NOT print 'uitkomst' in non-handout
  %  (presumably, there is also an 'oplossing' ??)
  \newenvironment{uitkomst}[1][onzichtbaar]{}{}
\fi
\fi

%
% Uitweidingen zijn extra's die niet redelijkerwijze tot de leerstof behoren
% Uitbreidingen zijn extra's die wel redelijkerwijze tot de leerstof van bv meer geavanceerde versies kunnen behoren (B-programma/Wiskundestudenten/...?)
% Nog niet voorzien: design voor verschillende versies (A/B programma, BIO, voorkennis/ ...)
% Voor 'uitweidingen' is er een environment die online per default is ingeklapt, en in pdf al dan niet kan worden geincluded  (via \xmnouitweiding) 
%
% in een xourse, per default GEEN uitweidingen, tenzij \xmuitweiding expliciet ergens is gezet ...
\ifdefined\isXourse
   \ifdefined\xmuitweiding
   \else
       \def\xmnouitweiding{true}
   \fi
\fi

\ifdefined\xmnouitweiding
\newcounter{xmuitweiding}  % anders error undefined ...  
\excludecomment{xmuitweiding}
\else
\newtheoremstyle{dotless}{}{}{}{}{}{}{ }{}
\theoremstyle{dotless}
\newtheorem*{xmuitweidingnofrills}{}   % nofrills = no accordion; gebruikt dus de dotless theoremstyle!

\newcounter{xmuitweiding}
\newenvironment{xmuitweiding}[1][ ]%
{% 
	\refstepcounter{xmuitweiding}%
    \begin{expandable}{xmuitweiding}{Uitweiding \arabic{xmuitweiding}: #1}%
	\begin{xmuitweidingnofrills}%
}
{%
    \end{xmuitweidingnofrills}%
    \end{expandable}%
}   
% \newenvironment{xmuitweiding}[1][ ]%
% {% 
% 	\refstepcounter{xmuitweiding}
% 	\begin{accordion}\begin{accordion-item}[Uitweiding \arabic{xmuitweiding}: #1]%
% 			\begin{xmuitweidingnofrills}%
% 			}
% 			{\end{xmuitweidingnofrills}\end{accordion-item}\end{accordion}}   
\fi


\newenvironment{xmexpandable}[1][]{
	\begin{accordion}\begin{accordion-item}[#1]%
		}{\end{accordion-item}\end{accordion}}


% Command that gives a selection box online, but just prints the right answer in pdf
\newcommand{\xmonlineChoice}[1]{\pdfOnly{\wordchoicegiventrue}\wordChoice{#1}\pdfOnly{\wordchoicegivenfalse}}   % deprecated, gebruik onlineChoice ...
\newcommand{\onlineChoice}[1]{\pdfOnly{\wordchoicegiventrue}\wordChoice{#1}\pdfOnly{\wordchoicegivenfalse}}

\newcommand{\TJa}{\nlentext{ Ja }{ Yes }}
\newcommand{\TNee}{\nlentext{ Nee }{ No }}
\newcommand{\TJuist}{\nlentext{ Juist }{ True }}
\newcommand{\TFout}{\nlentext{ Fout }{ False }}

\newcommand{\choiceTrue}{{\wordChoice{\choice[correct]{\TJuist}\choice{\TFout}}}}
\newcommand{\choiceFalse}{{\wordChoice{\choice{\TJuist}\choice[correct]{\TFout}}}}

\newcommand{\choiceYes}{{\wordChoice{\choice[correct]{\TJa}\choice{\TNee}}}}
\newcommand{\choiceNo}{{\wordChoice{\choice{\TJa}\choice[correct]{\TNee}}}}

\newcommand{\choiceEen}{{\wordChoice{\choice[correct]{een }\choice{geen }}}}
\newcommand{\choiceGeen}{{\wordChoice{\choice{een }\choice[correct]{geen }}}}

% Optional nicer formatting for wordChoice in PDF

\let\inlinechoiceorig\inlinechoice

%\makeatletter
%\providecommand{\choiceminimumverticalsize}{\vphantom{$\frac{\sqrt{2}}{2}$}}   % minimum height of boxes (cfr infra)
\providecommand{\choiceminimumverticalsize}{\vphantom{$\tfrac{2}{2}$}}   % minimum height of boxes (cfr infra)

\newcommand{\inlinechoicesquares}[2][]{%
		\setkeys{choice}{#1}%
		\ifthenelse{\boolean{\choice@correct}}%
		{%
            \ifhandout%
               \fbox{\choiceminimumverticalsize #2}\allowbreak\ignorespaces%
            \else%
               \fcolorbox{blue}{blue!20}{\choiceminimumverticalsize #2\checkmark}\allowbreak\ignorespaces\setkeys{choice}{correct=false}\ignorespaces%
            \fi%
		}%
		{% else
			\fbox{\choiceminimumverticalsize #2}\allowbreak\ignorespaces%  HACK: wat kleiner, zodat fits on line ... 	
		}%
}

\newcommand{\inlinechoicesquareX}[2][]{%
		\setkeys{choice}{#1}%
		\ifthenelse{\boolean{\choice@correct}}%
		{%
            \ifhandout%
               \fbox{\choiceminimumverticalsize #2}\allowbreak\ignorespaces\setkeys{choice}{correct=false}\ignorespaces%
            \else%
               \fcolorbox{blue}{blue!20}{\choiceminimumverticalsize #2\checkmark}\allowbreak\ignorespaces\setkeys{choice}{correct=false}\ignorespaces%
            \fi%
		}%
		{% else
        \ifhandout%
			\fbox{\choiceminimumverticalsize #2}\allowbreak\ignorespaces%  HACK: wat kleiner, zodat fits on line ... 	
        \fi
		}%
}


\newcommand{\inlinechoicepuntjes}[2][]{%
		\setkeys{choice}{#1}%
		\ifthenelse{\boolean{\choice@correct}}%
		{%
            \ifhandout%
               \dots\ldots\ignorespaces\setkeys{choice}{correct=false}\ignorespaces
            \else%
               \fcolorbox{blue}{blue!20}{\choiceminimumverticalsize #2}\allowbreak\ignorespaces\setkeys{choice}{correct=false}\ignorespaces%
            \fi%
		}%
		{% else
			%\fbox{\choiceminimumverticalsize #2}\allowbreak\ignorespaces%  HACK: wat kleiner, zodat fits on line ... 	
		}%
}

% print niets, maar definieer globale variable \myanswer
%  (gebruikt om oplossingsbladen te printen) 
\newcommand{\inlinechoicedefanswer}[2][]{%
		\setkeys{choice}{#1}%
		\ifthenelse{\boolean{\choice@correct}}%
		{%
               \gdef\myanswer{#2}\setkeys{choice}{correct=false}

		}%
		{% else
			%\fbox{\choiceminimumverticalsize #2}\allowbreak\ignorespaces%  HACK: wat kleiner, zodat fits on line ... 	
		}%
}



%\makeatother

\newcommand{\setchoicedefanswer}{
\ifdefined\HCode
\else
%    \renewenvironment{multipleChoice@}[1][]{}{} % remove trailing ')'
    \let\inlinechoice\inlinechoicedefanswer
\fi
}

\newcommand{\setchoicepuntjes}{
\ifdefined\HCode
\else
    \renewenvironment{multipleChoice@}[1][]{}{} % remove trailing ')'
    \let\inlinechoice\inlinechoicepuntjes
\fi
}
\newcommand{\setchoicesquares}{
\ifdefined\HCode
\else
    \renewenvironment{multipleChoice@}[1][]{}{} % remove trailing ')'
    \let\inlinechoice\inlinechoicesquares
\fi
}
%
\newcommand{\setchoicesquareX}{
\ifdefined\HCode
\else
    \renewenvironment{multipleChoice@}[1][]{}{} % remove trailing ')'
    \let\inlinechoice\inlinechoicesquareX
\fi
}
%
\newcommand{\setchoicelist}{
\ifdefined\HCode
\else
    \renewenvironment{multipleChoice@}[1][]{}{)}% re-add trailing ')'
    \let\inlinechoice\inlinechoiceorig
\fi
}

\setchoicesquareX  % by default list-of-squares with onlineChoice in PDF

% Omdat multicols niet werkt in html: enkel in pdf  (in html zijn langere pagina's misschien ook minder storend)
\newenvironment{xmmulticols}[1][2]{
 \pdfOnly{\begin{multicols}{#1}}%
}{ \pdfOnly{\end{multicols}}}

%
% Te gebruiken in plaats van \section\subsection
%  (in een printstyle kan dan het level worden aangepast
%    naargelang \chapter vs \section style )
% 3/2021: DO NOT USE \xmsubsection !
\newcommand\xmsection\subsection
\newcommand\xmsubsection\subsubsection

% Aanpassen printversie
%  (hier gedefinieerd, zodat ze in xourse kunnen worden gezet/overschreven)
\providebool{parttoc}
\providebool{printpartfrontpage}
\providebool{printactivitytitle}
\providebool{printactivityqrcode}
\providebool{printactivityurl}
\providebool{printcontinuouspagenumbers}

% The following three commands are hardcoded in xake, you can't create other commands like these, without adding them to xake as well
%  ( gebruikt in xourses om juiste soort titelpagina te krijgen voor verschillende ximera's )
\newcommand{\activitychapter}[1]{
	\typeout{ACTIVITYCHAPTER #1}   % logging
	\chapterstyle
	\activity{#1}
}
\newcommand{\activitysection}[1]{
	\typeout{ACTIVITYSECTION #1}   % logging
	\sectionstyle
	\activity{#1}
}
% Partices worden als activity getoond om de grote blokken te krijgen online
\newcommand{\practicesection}[1]{
	\typeout{PRACTICESECTION #1}   % logging
	\sectionstyle
	\activity{#1}
}


% Commando om de printstyle toe te voegen in ximera's. Zorgt ervoor dat er geen problemen zijn als je de xourses compileert
% hack om onhandige relative paden in TeX te omzeilen
% should work both in xourse and ximera (pre-112022 only in ximera; thus obsoletes adhoc setup in xourses)
% loads global.sty if present (cfr global.css for online settings!)
% use global.sty to overwrite settings in printstyle.sty ...
\newcommand{\addPrintStyle}[1]{
\iftikzexport\else   % only in PDF
  \makeatletter
  \ifx\@onlypreamble\@notprerr\else   % ONLY if in tex-preamble   (and e.g. not when included from xourse)
    \typeout{Loading printstyle}   % logging
    \usepackage{#1/printstyle} % mag enkel geinclude worden als je die apart compileert
    \IfFileExists{#1/global.sty}{
        \typeout{Loading printstyle-folder #1/global.sty}   % logging
        \usepackage{#1/global}
        }{
        \typeout{Info: No extra #1/global.sty}   % logging
    }   % load global.sty if present
    \IfFileExists{global.sty}{
        \typeout{Loading local-folder global.sty (or TEXINPUTPATH..)}   % logging
        \usepackage{global}
    }{
        \typeout{Info: No folder/global.sty}   % logging
    }   % load global.sty if present
    \IfFileExists{\currfilebase.sty}
    {
        \typeout{Loading \currfilebase.sty}
        \input{\currfilebase.sty}
    }{
        \typeout{Info: No local \currfilebase.sty}
    }
    \fi
  \makeatother
\fi
}

%
%  
% references: Ximera heeft adhoc logica	 om online labels te doen werken over verschillende files heen
% met \hyperref kan de getoonde tekst toch worden opgegeven, in plaats van af te hangen van de label-text
\ifdefined\HCode
% Link to standard \labels, but give your own description
% Usage:  Volg \hyperref[my_very_verbose_label]{deze link} voor wat tijdverlies
%   (01/2020: Ximera-server aangepast om bij class reference-keeptext de link-text NIET te vervangen door de label-text !!!) 
\renewcommand{\hyperref}[2][]{\HCode{<a class="reference reference-keeptext" href="\##1">}#2\HCode{</a>}}
%
%  Link to specific targets  (not tested ?)
\renewcommand{\hypertarget}[1]{\HCode{<a class="ximera-label" id="#1"></a>}}
\renewcommand{\hyperlink}[2]{\HCode{<a class="reference reference-keeptext" href="\##1">}#2\HCode{</a>}}
\fi


\renewcommand{\figurename}{Figuur}
\renewcommand{\tablename}{Tabel}

%
% Gedoe om verschillende versies van xourse/ximera te maken afhankelijk van settings
%
% default: versie met antwoorden
% handout: versie voor de studenten, zonder antwoorden/oplossingen
% full: met alles erop en eraan, dus geschikt voor auteurs en/of lesgevers  (bevat in de pdf ook de 'online-only' stukken!)
%
%
% verder kunnen ook opties/variabele worden gezet voor hints/auteurs/uitweidingen/ etc
%
% 'Full' versie
\newtoggle{showonline}
\ifdefined\HCode   % zet default showOnline
    \toggletrue{showonline} 
\else
    \togglefalse{showonline}
\fi

% Full versie   % deprecated: see infra
\newcommand{\printFull}{
    \hintstrue
    \handoutfalse
    \toggletrue{showonline} 
}

\ifdefined\shouldPrintFull   % deprecated: see infra
    \printFull
\fi

%% \onlineOnly kan jammer genoeg niet, omdat het al betsaat als neveneffect van \begin{onlineOnly} ...
\newcommand{\onlyOnline}[1]{\ifdefined\HCode#1\fi}

% Overschrijf onlineOnly  (zoals gedefinieerd in ximera.cls)
\ifhandout   % in handout: gebruik de oorspronkelijke ximera.cls implementatie  (is dit wel nodig/nuttig?)
\else
    \iftoggle{showonline}{%
        \ifdefined\HCode
          \RenewEnviron{onlineOnly}{\bgroup\BODY\egroup}   % showOnline, en we zijn  online, dus toon de tekst
        \else
          \RenewEnviron{onlineOnly}{\bgroup\color{red!50!black}\BODY\egroup}   % showOnline, maar we zijn toch niet online: kleur de tekst rood 
        \fi
    }{%
      \RenewEnviron{onlineOnly}{\setbox0\vbox\bgroup\BODY\egroup}% geen showOnline
    }
\fi

% hack om na hoofding van definition/proposition/... als dan niet op een nieuwe lijn te starten
% soms is dat goed en mooi, en soms niet; en in HTML is het nu (2/2020) anders dan in pdf
% vandaar suggestie om 
%     \begin{definition}[Nieuw concept] \nl
% te gebruiken als je zeker een newline wil na de hoofdig en titel
% (in het bijzonder itemize zonder \nl is 'lelijk' ...)
\ifdefined\HCode
\newcommand{\nl}{}
\else
\newcommand{\nl}{\ \par} % newline (achter heading van definition etc.)
\fi


% \nl enkel in handoutmode (ihb voor \wordChoice, die dan typisch veeeel langer wordt)
\ifdefined\HCode
\providecommand{\handoutnl}{}
\else
\providecommand{\handoutnl}{%
\ifhandout%
  \nl%
\fi%
}
\fi

% Could potentially replace \pdfOnline/\begin{onlineOnly} : 
% Usage= \ifonline{Hallo surfer}{Hallo PDFlezer}
\providecommand{\ifonline}[2]%
{
\begin{onlineOnly}#1\end{onlineOnly}%
\pdfOnly{#2}
}%


%
% Maak optionele 'basic' en 'extended' versies van een activity
%  met environment basicOnly, basicSkip en extendedOnly
%
%  (
%   Dit werkt ENKEL in de PDF; de online versies tonen (minstens voorklopig) steeds 
%   het default geval met printbasicversion en printextendversion beide FALSE
%  )
%
\providebool{printbasicversion}
\providebool{printextendedversion}   % not properly implemented
\providebool{printfullversion}       % presumably print everything (debug/auteur)
%
% only set these in xourses, and BEFORE loading this preamble
%
%\newif\ifshowbasic     \showbasictrue        % use this line in xourse to show 'basic' sections
%\newif\ifshowextended  \showextendedtrue     % use this line in xourse to show 'extended' sections
%
%
%\ifbool{showbasic}
%      { \NewEnviron{basicOnly}{\BODY} }    % if yes: just print contents
%      { \NewEnviron{basicOnly}{}      }    % if no:  completely ignore contents
%
%\ifbool{showbasic}
%      { \NewEnviron{basicSkip}{}      }
%      { \NewEnviron{basicSkip}{\BODY} }
%

\ifbool{printextendedversion}
      { \NewEnviron{extendedOnly}{\BODY} }
      { \NewEnviron{extendedOnly}{}      }
      


\ifdefined\HCode    % in html: always print
      \newenvironment*{basicOnly}{}{}    % if yes: just print contents
      \newenvironment*{basicSkip}{}{}    % if yes: just print contents
\else

\ifbool{printbasicversion}
      {\newenvironment*{basicOnly}{}{}}    % if yes: just print contents
      {\NewEnviron{basicOnly}{}      }    % if no:  completely ignore contents

\ifbool{printbasicversion}
      {\NewEnviron{basicSkip}{}      }
      {\newenvironment*{basicSkip}{}{}}

\fi

\usepackage{float}
\usepackage[rightbars,color]{changebar}

% Full versie
\ifbool{printfullversion}{
    \hintstrue
    \handoutfalse
    \toggletrue{showonline}
    \printbasicversionfalse
    \cbcolor{red}
    \renewenvironment*{basicOnly}{\cbstart}{\cbend}
    \renewenvironment*{basicSkip}{\cbstart}{\cbend}
    \def\xmtoonprintopties{FULL}   % will be printed in footer
}
{}
      
%
% Evalueer \ifhints IN de environment
%  
%
%\RenewEnviron{hint}
%{
%\ifhandout
%\ifhints\else\setbox0\vbox\fi%everything in een emty box
%\bgroup 
%\stepcounter{hintLevel}
%\BODY
%\egroup\ignorespacesafterend
%\addtocounter{hintLevel}{-1}
%\else
%\ifhints
%\begin{trivlist}\item[\hskip \labelsep\small\slshape\bfseries Hint:\hspace{2ex}]
%\small\slshape
%\stepcounter{hintLevel}
%\BODY
%\end{trivlist}
%\addtocounter{hintLevel}{-1}
%\fi
%\fi
%}

% Onafhankelijk van \ifhandout ...? TO BE VERIFIED
\RenewEnviron{hint}
{
\ifhints
\begin{trivlist}\item[\hskip \labelsep\small\bfseries Hint:\hspace{2ex}]
\small%\slshape
\stepcounter{hintLevel}
\BODY
\end{trivlist}
\addtocounter{hintLevel}{-1}
\else
\iftikzexport   % anders worden de tikz tekeningen in hints niet gegenereerd ?
\setbox0\vbox\bgroup
\stepcounter{hintLevel}
\BODY
\egroup\ignorespacesafterend
\addtocounter{hintLevel}{-1}
\fi % ifhandout
\fi %ifhints
}

%
% \tab sets typewriter-tabs (e.g. to format questions)
% (Has no effect in HTML :-( ))
%
\usepackage{tabto}
\ifdefined\HCode
  \renewcommand{\tab}{\quad}    % otherwise dummy .png's are generated ...?
\fi


% Also redefined in  preamble to get correct styling 
% for tikz images for (\tikzexport)
%

\theoremstyle{definition} % Bold titels
\makeatletter
\let\proposition\relax
\let\c@proposition\relax
\let\endproposition\relax
\makeatother
\newtheorem{proposition}{Eigenschap}


%\instructornotesfalse

% logic with \ifhandoutin ximera.cls unclear;so overwrite ...
\makeatletter
\@ifundefined{ifinstructornotes}{%
  \newif\ifinstructornotes
  \instructornotesfalse
  \newenvironment{instructorNotes}{}{}
}{}
\makeatother
\ifinstructornotes
\else
\renewenvironment{instructorNotes}%
{%
    \setbox0\vbox\bgroup
}
{%
    \egroup
}
\fi

% \RedeclareMathOperator
% from https://tex.stackexchange.com/questions/175251/how-to-redefine-a-command-using-declaremathoperator
\makeatletter
\newcommand\RedeclareMathOperator{%
    \@ifstar{\def\rmo@s{m}\rmo@redeclare}{\def\rmo@s{o}\rmo@redeclare}%
}
% this is taken from \renew@command
\newcommand\rmo@redeclare[2]{%
    \begingroup \escapechar\m@ne\xdef\@gtempa{{\string#1}}\endgroup
    \expandafter\@ifundefined\@gtempa
    {\@latex@error{\noexpand#1undefined}\@ehc}%
    \relax
    \expandafter\rmo@declmathop\rmo@s{#1}{#2}}
% This is just \@declmathop without \@ifdefinable
\newcommand\rmo@declmathop[3]{%
    \DeclareRobustCommand{#2}{\qopname\newmcodes@#1{#3}}%
}
\@onlypreamble\RedeclareMathOperator
\makeatother


%
% Engelse vertaling, vooral in mathmode
%
% 1. Algemene procedure
%
\ifdefined\isEn
 \newcommand{\nlen}[2]{#2}
 \newcommand{\nlentext}[2]{\text{#2}}
 \newcommand{\nlentextbf}[2]{\textbf{#2}}
\else
 \newcommand{\nlen}[2]{#1}
 \newcommand{\nlentext}[2]{\text{#1}}
 \newcommand{\nlentextbf}[2]{\textbf{#1}}
\fi

%
% 2. Lijst van erg veel gebruikte uitdrukkingen
%

% Ja/Nee/Fout/Juits etc
%\newcommand{\TJa}{\nlentext{ Ja }{ and }}
%\newcommand{\TNee}{\nlentext{ Nee }{ No }}
%\newcommand{\TJuist}{\nlentext{ Juist }{ Correct }
%\newcommand{\TFout}{\nlentext{ Fout }{ Wrong }
\newcommand{\TWaar}{\nlentext{ Waar }{ True }}
\newcommand{\TOnwaar}{\nlentext{ Vals }{ False }}
% Korte bindwoorden en, of, dus, ...
\newcommand{\Ten}{\nlentext{ en }{ and }}
\newcommand{\Tof}{\nlentext{ of }{ or }}
\newcommand{\Tdus}{\nlentext{ dus }{ so }}
\newcommand{\Tendus}{\nlentext{ en dus }{ and thus }}
\newcommand{\Tvooralle}{\nlentext{ voor alle }{ for all }}
\newcommand{\Took}{\nlentext{ ook }{ also }}
\newcommand{\Tals}{\nlentext{ als }{ when }} %of if?
\newcommand{\Twant}{\nlentext{ want }{ as }}
\newcommand{\Tmaal}{\nlentext{ maal }{ times }}
\newcommand{\Toptellen}{\nlentext{ optellen }{ add }}
\newcommand{\Tde}{\nlentext{ de }{ the }}
\newcommand{\Thet}{\nlentext{ het }{ the }}
\newcommand{\Tis}{\nlentext{ is }{ is }} %zodat is in text staat in mathmode (geen italics)
\newcommand{\Tmet}{\nlentext{ met }{ where }} % in situaties e.g met p < n --> where p < n
\newcommand{\Tnooit}{\nlentext{ nooit }{ never }}
\newcommand{\Tmaar}{\nlentext{ maar }{ but }}
\newcommand{\Tniet}{\nlentext{ niet }{ not }}
\newcommand{\Tuit}{\nlentext{ uit }{ from }}
\newcommand{\Ttov}{\nlentext{ t.o.v. }{ w.r.t. }}
\newcommand{\Tzodat}{\nlentext{ zodat }{ such that }}
\newcommand{\Tdeth}{\nlentext{de }{th }}
\newcommand{\Tomdat}{\nlentext{omdat }{because }} 


%
% Overschrijf addhoc commando's
%
\ifdefined\isEn
\renewcommand{\pernot}{\overset{\mathrm{notation}}{=}}
\RedeclareMathOperator{\bld}{im}     % beeld
\RedeclareMathOperator{\graf}{graph}   % grafiek
\RedeclareMathOperator{\rico}{slope}   % richtingcoëfficient
\RedeclareMathOperator{\co}{co}       % coordinaat
\RedeclareMathOperator{\gr}{deg}       % graad

% Operators
\RedeclareMathOperator{\bgsin}{arcsin}
\RedeclareMathOperator{\bgcos}{arccos}
\RedeclareMathOperator{\bgtan}{arctan}
\RedeclareMathOperator{\bgcot}{arccot}
\RedeclareMathOperator{\bgsinh}{arcsinh}
\RedeclareMathOperator{\bgcosh}{arccosh}
\RedeclareMathOperator{\bgtanh}{arctanh}
\RedeclareMathOperator{\bgcoth}{arccoth}

\fi

\renewcommand{\Im}[1]{\text{Im}#1}
\renewcommand{\Re}[1]{\text{Re}#1}


% Problem-inside-div  (for css styling ...)
\newcommand{\xmdivEnvironmentStart}[3]{%
\ifdefined\HCode%
   \HCode{\Hnewline<div class="#2">}%
\fi%
\problemEnvironmentStart{#1}{#3}%
}


\newcommand{\xmdivEnvironmentEnd}{%
\problemEnvironmentEnd%
\ifdefined\HCode%
    \HCode{\Hnewline</div>}%
\fi%
}


\newenvironment{quickquestion*}[1][2in]%
{%Env start code
\xmdivEnvironmentStart{#1}{quickquestion}{Quick Question}%
}
{%Env end code
\xmdivEnvironmentEnd%
}
\newenvironment{quickquestion}[1][2in]%
{%Env start code
\xmdivEnvironmentStart{#1}{quickquestion}{Quick Question}%
}
{%Env end code
\xmdivEnvironmentEnd%
}

\newenvironment{denkvraag*}[1][2in]%
{%Env start code
\xmdivEnvironmentStart{#1}{denkvraag}{Denkvraag}%
}
{%Env end code
\xmdivEnvironmentEnd
}

\newenvironment{denkvraag}[1][2in]%
{%Env start code
\xmdivEnvironmentStart{#1}{denkvraag}{Denkvraag}%
}
{%Env end code
\xmdivEnvironmentEnd
}

\addPrintStyle{../..}
\begin{document}
    \author{Zomercursus KU Leuven}
    \xmtitle{Oefeningen goniometrische schrijfwijze van complexe getallen}
    % Start inhoud ximera 


\begin{exercise}
	Schrijf onderstaande getallen in goniometrische vorm $r(\cos\theta+i\sin \theta)$. Zorg er steeds voor dat $\theta$ tussen 0 en $2\pi$ ligt, dus niet $-\frac{\pi}{2}$ invullen maar $\frac{3\pi}{2}$.
	\\
\begin{question} Voor $z=1$ is $r=\answer[onlineshowanswerbutton]{1}$ en $\theta=\answer[onlineshowanswerbutton]{0}$.
\begin{oplossing} 
Omdat $z = 1 = 1 + 0i$ en dus $a = 0$ en $b = 1$, vinden we $|z| = r = 1$. In het complexe vlak zien we dadelijk dat $\theta = 0$.
%Bovendien voldoet $\theta$ aan $\cos \theta = 1$ en $\sin \theta = 0$ zodat $\theta = 0$. 
De goniometrische schrijfwijze van $z$ is dan $z = 1 =  \cos 0 + i\sin 0$.
\begin{image}[0.5\textwidth]
\begin{tikzpicture}[scale = 1, trig format = rad]
	%	grid setup
	
	\def\xmin{-2}
	\def\xmax{2}
	
	\def\ymin{-2}
	\def\ymax{2}				
	
	%	grid
	\draw[step = 1, black, thin, dashed] (\xmin, \ymin) grid (\xmax, \ymax);
	\draw[->] (\xmin, 0) -- (\xmax, 0) node[right]{Re$(z)$};
	\draw[->] (0, \ymin) -- (0, \ymax) node[above]{Im$(z)$};;	
	
	%	vector
	\draw[->, blue, very thick] (0, 0) -- node[below]{$1$}(1, 0);
	
	%	boogje - niet handig nu
	\def\r{0.2}
	%\draw[->, blue] (\r, 0) arc (0:0:\r) node[right]{$0$};
\end{tikzpicture}
\end{image}
\end{oplossing}
\end{question}
    
\begin{question} 
Voor $z=-1$ is $r=\answer[onlineshowanswerbutton]{1}$ en $\theta=\answer[onlineshowanswerbutton]{\pi}$.
\begin{oplossing} 
$z=-1 = -1 + 0i$ en dus $a = -1$ en $b = 0$, waaruit we halen dat $|z| = r = 1$. In het complexe vlak zien we dadelijk dat $\theta = \pi$. 
%Het argument van $z$ moet voldoen aan $\cos \theta = -1$ en $\sin \theta = 0$ zodat $\theta = \pi$. 
De goniometrische schrijfwijze van $z$ is dan $z  = \cos \pi + i\sin \pi$.
\begin{image}[0.5\textwidth]
\begin{tikzpicture}[scale = 1]
	%	grid setup
	
	\def\xmin{-2}
	\def\xmax{2}
	
	\def\ymin{-2}
	\def\ymax{2}				
	
	%	grid
	\draw[step = 1, black, thin, dashed] (\xmin, \ymin) grid (\xmax, \ymax);
	\draw[->] (\xmin, 0) -- (\xmax, 0) node[right]{Re$(z)$};
	\draw[->] (0, \ymin) -- (0, \ymax) node[above]{Im$(z)$};;	
	
	%	vector
	\draw[->, blue, very thick] (0, 0) -- node[below]{$-1$}(-1, 0);
	
	%	boogje - niet handig nu
	\def\r{0.2}
	\draw[->, blue] (\r, 0) arc (0:180:\r) node[midway, above]{$\pi$};
\end{tikzpicture}
\end{image}
Merk op dat de modules positief moet zijn, en met $r=-1$ en $\theta = 0$ is $-(\cos 0 + i\sin0)$ dus \textit{geen} goniometrische schrijfwijze van het getal $-1$.
\end{oplossing}
\end{question}
    
\begin{question} 
Voor $z = i$ is $r = \answer[onlineshowanswerbutton]{1}$ en $\theta = \answer[onlineshowanswerbutton]{\frac{\pi}{2}}$.
\begin{oplossing} 
Omdat $z = i = 0 + i$, en dus $a = 0$, $b = 1$, zodat $|z| = r = 1$. In het complexe vlak zien we dadelijk dat $\theta=\frac{\pi}{2}$.
% en $\theta$ voldoet aan $\cos\theta = 0$ en $\sin \theta = 1$ (of je kan het ook afleiden uit onderstaande schets). 
 De goniometrische schrijfwijze van $z$ is dus $z = \cos \frac{\pi}{2} +i\sin \frac{\pi}{2}$.
\begin{image}[0.5\textwidth]
\begin{tikzpicture}[scale = 1]
	%	grid setup
	
	\def\xmin{-2}
	\def\xmax{2}
	
	\def\ymin{-2}
	\def\ymax{2}				
	
	%	grid
	\draw[step = 1, black, thin, dashed] (\xmin, \ymin) grid (\xmax, \ymax);
	\draw[->] (\xmin, 0) -- (\xmax, 0) node[right]{Re$(z)$};
	\draw[->] (0, \ymin) -- (0, \ymax) node[above]{Im$(z)$};;	
	
	%	vector
	\draw[->, blue, very thick] (0, 0) -- node[left]{$i$}(0, 1);
	
	%	boogje
	\def\r{0.2}
	\draw[->, blue] (\r, 0) arc (0:90:\r) node[midway, right]{$\pi/2$};
\end{tikzpicture}
\end{image}

\end{oplossing}
\end{question}

\begin{question} 
Voor $z = -i$ is $r=\answer[onlineshowanswerbutton]{1}$ en $\theta = \answer[onlineshowanswerbutton]{3\frac{\pi}{2}}$.
\begin{oplossing}
Voor $z = -i$ is $a = 0$ en $b = -1$, dus $|z| = r = 1$. In het complexe vlak zien we dadelijk dat $\theta=\frac{3\pi}{2}$.
% en $\theta$ voldoet aan $\cos \theta = 0$ en $\sin \theta = -1$, dus $\theta = - \frac{\pi}{2} = \frac{3\pi}{2}$. 
 De goniometrische schrijfwijze van $z$ is dus $z  = \cos\left(\frac{3\pi}{2}\right) +i \sin\left(\frac{3\pi}{2}\right)$.
\begin{image}[0.5\textwidth]
\begin{tikzpicture}[scale = 1]
	%	grid setup
	
	\def\xmin{-2}
	\def\xmax{2}
	
	\def\ymin{-2}
	\def\ymax{2}				
	
	%	grid
	\draw[step = 1, black, thin, dashed] (\xmin, \ymin) grid (\xmax, \ymax);
	\draw[->] (\xmin, 0) -- (\xmax, 0) node[right]{Re$(z)$};
	\draw[->] (0, \ymin) -- (0, \ymax) node[above]{Im$(z)$};;	
	
	%	vector
	\draw[->, blue, very thick] (0, 0) -- node[right]{$-i$}(0, -1);
	
	%	boogje
	\def\r{0.2}
	\draw[->, blue] (\r, 0) arc (0:270:\r) node[midway, above]{$\frac{3\pi}{2}$};
\end{tikzpicture}
\end{image}
\end{oplossing}
\end{question}

\begin{question} 
Voor $z = \sqrt{3} + i$ is $r = \answer[onlineshowanswerbutton]{2}$ en $\theta = \answer[onlineshowanswerbutton]{\frac{\pi}{6}}$.
\begin{oplossing} 
Voor $z = \sqrt{3} + i$ geldt $a = \sqrt{3}$ en $b = 1$, dus $|z| = r = \sqrt{a^2 + b^2} = 2$. Het argument van $z$ voldoet aan $\cos \theta = \frac{\sqrt{3}}{2}$ en $\sin \theta = \frac{1}{2}$. Dit zijn de goniometrische getallen van de standaardhoek $\pi/6$. De goniometrische schrijfwijze van $z$ is dus 
$$
z = 2\left(\cos \left(\frac{\pi}{6}\right) + i\sin\left(\frac{\pi}{6}\right)\right) \, .
$$
\begin{image}[0.5\textwidth]
\begin{tikzpicture}[scale = 1.5]
	%	grid setup
	
	\def\xmin{-2.1}
	\def\xmax{2.1}
	
	\def\ymin{-2.1}
	\def\ymax{2.1}				
	
	%	grid
	\draw[step = 1, black, thin, dashed] (\xmin, \ymin) grid (\xmax, \ymax);
	\draw[->] (\xmin, 0) -- (\xmax, 0) node[right]{Re$(z)$};
	\draw[->] (0, \ymin) -- (0, \ymax) node[above]{Im$(z)$};;	
	
	%	vector
	\draw[->, blue, very thick] (0, 0) -- ({sqrt(3)}, 1)node[below]{$\sqrt{3} + i$};
	
	%	boogje
	\def\r{0.5}
	\draw[->, blue] (\r, 0) arc (0:30:\r) node[midway, right]{$\frac{\pi}{6}$};
\end{tikzpicture}
\end{image}
\end{oplossing}
\end{question}
     
\begin{question} 
Voor $z = -\sqrt{3} + i$ is $r = \answer[onlineshowanswerbutton]{2}$ en $\theta = \answer[onlineshowanswerbutton]{\frac{5\pi}{6}}$.
\begin{oplossing} 
Voor $z = -\sqrt{3} + i$ geldt $a = -\sqrt{3}$ en $b = 1$. Dus geldt $|z| = r = 2$. Voor het argument van $z$ geldt dat $\cos \theta = -\frac{\sqrt{3}}{2}$ en $\sin \theta = \frac{1}{2}$. Dan is $\theta =  \frac{5\pi}{6} =\pi -  \frac{\pi}{6}$: dit kan je afleiden uit de goniometrische getallen voor $\pi/6$ en de rekenregels voor supplementaire hoeken te gebruiken.

De goniometrische schrijfwijze van $z$ is dus 
$$
z = 2\left(\cos \left(\frac{5\pi}{6}\right) + i\sin\left(\frac{5\pi}{6}\right)\right) \, .
$$

\begin{image}[0.5\textwidth]
\begin{tikzpicture}[scale = 1.5]
	%	grid setup
	
	\def\xmin{-2.1}
	\def\xmax{2.1}
	
	\def\ymin{-2.1}
	\def\ymax{2.1}				
	
	%	grid
	\draw[step = 1, black, thin, dashed] (\xmin, \ymin) grid (\xmax, \ymax);
	\draw[->] (\xmin, 0) -- (\xmax, 0) node[right]{Re$(z)$};
	\draw[->] (0, \ymin) -- (0, \ymax) node[above]{Im$(z)$};;	
	
	%	vector
	\draw[->, blue, very thick] (0, 0) -- ({-sqrt(3)}, 1)node[below]{$-\sqrt{3} + i$};
	
	%	boogje
	\def\r{0.5}
	\draw[->, blue] (\r, 0) arc (0:150:\r) node[midway, above]{$5\pi/6$};
\end{tikzpicture}
\end{image}
\end{oplossing}
\end{question}
    
\begin{question} 
Voor $z = \cos\frac{\pi}{4}+i\sin\frac{3\pi}{4}$ is $r = \answer[onlineshowanswerbutton]{1}$ en $\theta = \answer[onlineshowanswerbutton]{\frac{\pi}{4}}$.
\begin{oplossing} 
Door supplementaire hoeken te gebruiken vinden we dat 
$$
\sin \left( \frac{3\pi}{4}\right) = \sin \left( \pi-\frac{\pi}{4} \right) = \sin \left(\frac{\pi}{4} \right) \, .
$$
En dus is $z = \cos\frac{\pi}{4}+i\sin\frac{\pi}{4}$: dit is reeds de goniometrische schrijfwijze, met $r = 1$ en $\theta = \frac{\pi}{4}$.
\begin{image}[0.5\textwidth]
\begin{tikzpicture}[scale = 1.8]
	%	grid setup
	
	\def\xmin{-1.5}
	\def\xmax{1.5}
	
	\def\ymin{-1.5}
	\def\ymax{1.5}				
	
	%	grid
	\draw[step = 1, black, thin, dashed] (\xmin, \ymin) grid (\xmax, \ymax);
	\draw[->] (\xmin, 0) -- (\xmax, 0) node[right]{Re$(z)$};
	\draw[->] (0, \ymin) -- (0, \ymax) node[above]{Im$(z)$};;	
	
	%	vector
	\draw[->, blue, very thick] (0, 0) -- ({cos(45)}, {sin(45)}) node[above, fill = white]{$\cos\left(\frac{\pi}{4}\right) + \sin\left(\frac{3\pi}{4}\right) i$};
	
	%	boogje
	\def\r{0.5}
	\draw[->, blue] (\r, 0) arc (0:45:\r) node[midway, right]{$\pi/4$};
\end{tikzpicture}
\end{image}
\end{oplossing}
\end{question}
 
\begin{question} 
Voor $z = \cos\frac{\pi}{3} - i\sin\frac{\pi}{3}$ is $r = \answer[onlineshowanswerbutton]{1}$ en $\theta = \answer[onlineshowanswerbutton]{\frac{5\pi}{3}}$
\begin{oplossing}
Door tegengestelde hoeken te gebruiken vinden we dat $ - \sin\left( \frac{\pi}{3} \right) = \sin \left(-\frac{\pi}{3}\right)$ en $\cos \left(\frac{\pi}{3}\right) = \cos \left(- \frac{\pi}{3} \right)$. Dus $ z = \cos \left(-\frac{\pi}{3}\right) + i\sin \left(-\frac{\pi}{3} \right)$. Als we eisen dat $\theta \in [0, 2\pi]$, dan is de goniometrische vorm van $z$: $z = \cos\frac{5\pi}{3} - i\sin\frac{5\pi}{3}$.

\begin{image}[0.5\textwidth]
\begin{tikzpicture}[scale = 1.5]
	%	grid setup
	
	\def\xmin{-2}
	\def\xmax{2}
	
	\def\ymin{-2}
	\def\ymax{2}				
	
	%	grid
	\draw[step = 1, black, thin, dashed] (\xmin, \ymin) grid (\xmax, \ymax);
	\draw[->] (\xmin, 0) -- (\xmax, 0) node[right]{Re$(z)$};
	\draw[->] (0, \ymin) -- (0, \ymax) node[above]{Im$(z)$};;	
	
	%	vector
	\draw[->, blue, very thick] (0, 0) -- ({cos(-60)}, {sin(-60)}) node[below, fill = white]{$\cos\left(\frac{\pi}{3}\right) - i\sin\left(\frac{\pi}{3}\right)$};
	
	%	boogje
	\def\r{0.5}
	\draw[->, blue] (\r, 0) arc (0:300:\r) node[midway, above]{$5\pi/3$};
\end{tikzpicture}
\end{image}
\end{oplossing}
\end{question}
\end{exercise}

\begin{exercise}
	Bepaal de modulus $|z|$ en het argument $\theta$ van volgende complexe getallen $z$. Geef de hoek $\theta$ in tussen 0 en $2\pi$ radialen.
	
	\begin{question} 
		Als $z=-7i$, dan is $|z|=\answer[onlineshowanswerbutton]{7}$ en $\theta=\answer[onlineshowanswerbutton]{3\pi/2}$.
		\begin{oplossing}
			$z = a + bi = 0 - 7i$, dus is $|z| = \sqrt{a^2 + b^2} = 7$. In het complxe vlak zien we dadelijk dat $\theta = 3 \pi/2$.  Je kan dit ook berekenen:  $\cos\theta=0/7=0$ en $\sin\theta=-7/7=-1$, en dus is $\theta = 3 \pi/2$ (op een veelvoud van $2\pi$ na). 
			
			\begin{image}[0.5\textwidth]
				\begin{tikzpicture}[scale = 0.5]
				%	grid setup
				
				\def\xmin{-7}
				\def\xmax{7}
				
				\def\ymin{-8}
				\def\ymax{2}				
				
				%	grid
				\draw[step = 1.0, gray, thin, dashed] (\xmin, \ymin) grid (\xmax, \ymax);
				\draw[->] (\xmin, 0) -- (\xmax, 0) node[right]{Re$(z)$};
				\draw[->] (0, \ymin) -- (0, \ymax) node[above]{Im$(z)$};;	
				
				%	vector
				\draw[->, blue, very thick] (0, 0) -- node[right]{$-7i$}  (0, -7);
				
				%	boogje
				\def\r{1}
				\draw[->, blue] (\r, 0) arc (0:270:\r) node[midway, left]{$3\pi/2$};
				\end{tikzpicture}
			\end{image}
		\end{oplossing}
	\end{question}
	
	
	\begin{question} 
		Als $z=(7i)^{-1}$, dan is $|z|=\answer[onlineshowanswerbutton]{1/7}$ en $\theta=\answer[onlineshowanswerbutton]{3\pi/2}$.
		\begin{hint} Schrijf $z$ eerst in cartesische vorm $a+bi$. \end{hint}
		\begin{oplossing}
			Eerst herschrijven we het complex getal $z$ in cartesische vorm: $z = \frac{1}{7i} = - \frac17 i$ (vermenigvuldig teller en noemer met $i$). 
			\\
			Een tip voor berekeningen met complexe getallen: \textit{delen door $i$ is hetzelfde als vermenigvuldigen met $-i$}.
			
			Omdat $z = a + bi = 0 - \frac17i$ is $|z| = \sqrt{a^2 + b^2} = \frac17$, en $\theta = 3 \pi/2$ (op een veelvoud van $2\pi$ na), zoals we kunnen afleiden uit een schets (de ijk op beide assen is nu 0.1 in plaats van 1):
			
			\begin{image}[0.5\textwidth]
				\begin{tikzpicture}[scale = 5]
				%	grid setup
				
				\def\xmin{-0.4}
				\def\xmax{0.4}
				
				\def\ymin{-0.4}
				\def\ymax{0.4}				
				
				%	grid
				\draw[step = 0.1, gray, thin, dashed] (\xmin, \ymin) grid (\xmax, \ymax);
				\draw[->] (\xmin, 0) -- (\xmax, 0) node[right]{Re$(z)$};
				\draw[->] (0, \ymin) -- (0, \ymax) node[above]{Im$(z)$};;	
				
				%	vector
				\draw[->, blue, very thick] (0, 0) -- (0, -1/7) node[below right, fill = white]{$-\frac17i$};
				
				%	boogje
				\def\r{0.05}
				\draw[->, blue] (\r, 0) arc (0:270:\r) node[midway, above left, fill = white]{$3\pi/2$};
				\end{tikzpicture}
			\end{image}
		\end{oplossing}
	\end{question}
	
	
	\begin{question} 
		Als $z = \overline{1+i}$, dan is $|z| = \answer[onlineshowanswerbutton]{\sqrt{2}}$ en $\theta = \answer[onlineshowanswerbutton]{7\pi/4}$.
		\begin{hint} Schrijf $z$ eerst in cartesische vorm $a+bi$. \end{hint}
		\begin{oplossing}
			Eerst herschrijven we het complex getal $z$: $z = \overline{1+i} = 1 - i$.
			
			Omdat $z = a + bi = 1 - i$, is $|z| = \sqrt{a^2 + b^2} = \sqrt{2}$, en $\theta =- \pi / 4$: dit kan je afleiden uit de tekening of bepaal je uit de vergelijkingen 
			%$a = r\cos \theta$ en $b = r\sin\theta$. Deze kan je omvormen naar
			\begin{align*}
			\cos \theta &= \frac{a}{r} = \frac{1}{\sqrt{2}} \\
			\sin \theta &= \frac{b}{r} = -\frac{1}{\sqrt{2}}\\
			\end{align*} 
			We weten dat de cosinus en sinus van de standaardhoek $\pi/4$ beide gelijk zijn aan $\frac{1}{\sqrt{2}}=\frac{\sqrt2}{2}$, en de tegengestelde hoek heeft dezelfde cosinus, maar een tegengestelde sinus. Dus $\theta = - \pi/4$ of $\theta= 7 \pi/4$.
			\begin{image}[0.5\textwidth]
				\begin{tikzpicture}[scale = 1]
				%	grid setup
				
				\def\xmin{-2}
				\def\xmax{2}
				
				\def\ymin{-2}
				\def\ymax{2}				
				
				%	grid
				\draw[step = 1, black, thin, dashed] (\xmin, \ymin) grid (\xmax, \ymax);
				\draw[->] (\xmin, 0) -- (\xmax, 0) node[right]{Re$(z)$};
				\draw[->] (0, \ymin) -- (0, \ymax) node[above]{Im$(z)$};;	
				
				%	vector
				\draw[->, blue, very thick] (0, 0) --  (1, -1) node[above right]{$1-i$};
				
				%	boogje
				\def\r{0.5}
				\draw[->, blue] (\r, 0) arc (0:315:\r) node[midway, above left]{$7\pi/4$};
				\end{tikzpicture}
			\end{image}
		\end{oplossing}
	\end{question}
	
	
	\begin{question} 
		Als $z = \dfrac{1-i}{1+i}$, dan is $|z| = \answer[onlineshowanswerbutton]{1}$ en $\theta=\answer[onlineshowanswerbutton]{3\pi/2}$.
		
		\begin{hint} Schrijf $z$ eerst in cartesische vorm $a+bi$. \end{hint}
		\begin{oplossing}
			Eerst herschrijven we het complex getal $z$ naar de vorm $a + bi$:
			$$
			\frac{1-i}{1+i} = \frac{1-i}{1+i} \frac{1-i}{1-i} = \frac{(1-i)^2}{2}= \frac{-2i}{2} = - i 
			$$
			Dan is $|z| = 1$, en in het comlexe vlak zien we dadelijk dat $\theta = 3 \pi/2$.
			\begin{image}[0.5\textwidth]
				\begin{tikzpicture}[scale = 0.7]
				%	grid setup
				
				\def\xmin{-2}
				\def\xmax{2}
				
				\def\ymin{-2}
				\def\ymax{2}				
				
				%	grid
				\draw[step = 1, black, thin, dashed] (\xmin, \ymin) grid (\xmax, \ymax);
				\draw[->] (\xmin, 0) -- (\xmax, 0) node[right]{Re$(z)$};
				\draw[->] (0, \ymin) -- (0, \ymax) node[above]{Im$(z)$};;	
				
				%	vector
				\draw[->, blue, very thick] (0, 0) -- node[right]{$-i$}  (0, -1);
				
				%	boogje
				\def\r{0.2}
				\draw[->, blue] (\r, 0) arc (0:270:\r) node[midway, left]{$3\pi/2$};
				\end{tikzpicture}
			\end{image}
		\end{oplossing}
	\end{question}    
	
	%%% Volgende oefening in comment: geen standaardhoek, is minder interessant.
	
	
	%\begin{question} 
	%Als $z = \dfrac{1}{1+i}+\dfrac{1+i}{4}$, dan is $|z| = \answer[onlineshowanswerbutton]{\sqrt{10}/4}$ en $\theta=\answer[onlineshowanswerbutton]{-0.32}$.
	%\begin{hint}
	%Om $\theta$ te berekenen, gebruik je best een rekenmachine.
	%\end{hint}
	%\begin{oplossing}
	%Eerst herschrijven we het complex getal $z$ naar de vorm $a + bi$. We doen dit eerst voor de eerste term:
	%$$
	%\frac{1}{1+i} = \frac{1}{1+i} \frac{1-i}{1-i} = \frac{1-i}{2}
	%$$
	%Dan is 
	%$$
	%z = \frac12 - \frac{i}{2} + \frac14  + \frac{i}{4} = \frac34 - \frac14 i \, .
	%$$
	%Dan vinden we dat $|z| = \sqrt{a^2 + b^2} = \sqrt{\frac{10}{16}} = \frac{\sqrt{10}}{4}$. Uit $a = r\cos\theta$ halen we dan dat $\theta = \text{bgcos}\left(\frac{3}{\sqrt{10}}\right)$. Het resultaat kan je nu berekenen met een rekenmachine. 
	%\begin{image}[0.5\textwidth]
	%\begin{tikzpicture}[scale = 1, trig format = rad]
	%	%	grid setup
	%	
	%	\def\xmin{-2}
	%	\def\xmax{2}
	%	
	%	\def\ymin{-2}
	%	\def\ymax{2}				
	%	
	%	%	grid
	%	\draw[step = 1, black, thin, dashed] (\xmin, \ymin) grid (\xmax, \ymax);
	%	\draw[->] (\xmin, 0) -- (\xmax, 0) node[right]{Re$(z)$};
	%	\draw[->] (0, \ymin) -- (0, \ymax) node[above]{Im$(z)$};;	
	%	
	%	%	vector
	%	\draw[->, blue, very thick] (0, 0) -- node[below]{$\frac34 - \frac14 i$}(3/4, -1/4);
	%	
	%	%	boogje
	%	\def\r{0.5}
	%	\draw[->, blue] (\r, 0) arc (0:0.23:\r) node[midway, left]{$0.23$};
	%\end{tikzpicture}
	%\end{image}
	%\end{oplossing}
	%\end{question}    
\end{exercise}



%
%\begin{oefening2}
%	Los volgende vergelijkingen in $z\in \C$ op. Schrijf het resultaat in de vorm $a+bi$.
%	\begin{enumerate}
%		\item $5i \;z+2=3i$
%		\item $(3+i)(z+1)=z$
%		\item $\ds{\frac{7}{z+i}=1-i}$
%		\item $3 z +2 \overline z=10-i$
%		\item $(2+i) \; \overline z =z+4$
%	\end{enumerate}
%	\begin{opl}
%		\begin{enumerate}
%			\item $\frac{3}{5} +\frac{2i}{5}$
%			\item $\frac{-7}{5} +\frac{i}{5}$
%			\item $\frac{7}{2} +\frac{5i}{2}$
%			\item $2-i$
%			\item $3+i$
%		\end{enumerate}
%	\end{opl}
%\end{oefening2}
%
%\begin{oefening2}
%	%An Vanfroyenhoven
%	%\id{2015wis01}
%	
%	Veronderstel dat $x$ en $y$ complexe getallen zijn die voldoen aan het stelsel 
%	\begin{equation*}
%	\begin{cases}
%	(-1-i)x-2y & = 4 \\
%	x+(2-i)y   & = i,
%	\end{cases}
%	\end{equation*}
%	waarbij $i^2=-1$.
%	
%	Bepaal $x+y$.
%	\begin{enumerate}
%		\item [(A)] $x+y=-1+4i$
%		\item [(B)] $x+y=-1+2i$
%		\item [(C)] $x+y=-1$
%		\item [(D)] $x+y=1$
%		\item [(E)] $x+y$ kan oneindig veel waarden aannemen.
%	\end{enumerate}
%	
%	\begin{opl}
%		A 
%	\end{opl}
%	
%	\bron{ijkingstoets juli 2015}
%\end{oefening2}
%
%\begin{oefening2}
%	Bereken
%	\[  \left| \frac{(3+4i)(-1+2i)}{(-1-i)(3-i)}\right|\]
%	\begin{opl}
%		$\ds\frac{5}{2}$
%	\end{opl}
%\end{oefening2}
%
%\newpage
%\begin{oefening2}
%	%\id{2013hds05}
%	%\id{201307beg14**}
%	
%	Als het complex getal $z$ voldoet aan
%	$$
%	z^2 = \frac{(2+i)(-1+2i)}{2(3+4i)}
%	$$
%	dan is de modulus van $z$:
%	\begin{enumerate}
%		\item [(A)]$|z| = \frac{1}{2}$
%		\item [(B)] $|z| = \frac{\sqrt{2}}{2}$
%		\item [(C)]$|z| = \sqrt{2}$
%		\item [(D)]$|z| = - \frac{\sqrt{2}}{2}$
%		\item [(E)]$|z| = \frac{1}{4}$
%	\end{enumerate}
%	
%	
%	\begin{opl}
%		B
%	\end{opl}
%	\bron{ijkingstoets juli 2013}
%	
%\end{oefening2}
%
%\begin{oefening2}
%	%\id{SCvraag1}
%	
%	De getallen $\alpha$ en $\beta$ zijn re\"ele getallen, $i^2=-1$. Als $z_1=1-2i$ een nulpunt is van de complexe
%	veelterm $z^2-(\alpha+i)z+(7+i\beta)$,
%	wat is dan het tweede nulpunt $z_2$?
%	
%	\begin{enumerate}
%		\item [(A)] $z_2=1+2i$
%		\item [(B)] $z_2=1+i$
%		\item [(C)] $z_2=3$
%		\item [(D)] $z_2=1+3i$
%		\item [(E)] $z_2=1-3i$
%	\end{enumerate}
%	
%	
%	\begin{opl}
%		D
%	\end{opl}
%	
%	\bron{ijkingstoets juli 2015}
%\end{oefening2}
%
%%%%%%%%%%%%
%
%%%%%%%%%%%
%
%
%
%
%
%
%
%%%%%%%%%%%%%%
%\begin{oefening2}
%	%\id{2015wis09}
%	
%	Het complexe getal $z$ is de oplossing van de vergelijking $(2+i)(z+i)=z-1$, waarbij $i^2=-1$. Bereken de modulus van $z$.
%	
%	\vspace{4mm}
%	
%	(A) $|z|=1$\hspace{1cm}
%	(B) $|z|=\sqrt 2$ \hspace{1cm}
%	(C) $|z|=2$ \newline
%	(D) $|z|=\sqrt 5$  \hspace{1cm}
%	(E) $|z|=2\sqrt 5$
%	
%	
%	\begin{opl}
%		B
%	\end{opl}
%	%
%	%\cat{vaardigheden}
%	%\class{**}
%	%\info{ijkingstoets september 2015 - 313 deelnemers} 
%	%\juist{62} 
%	%\blanco{23} 
%	%\ul{85/36}
%	\bron{ijkingstoets september 2015}
%\end{oefening2}
%%%%%%%%%
%\newpage
%
%\begin{oefening2}
%	Bereken
%	\begin{enumerate} \item  $(-1-i)^{20}$
%		\item $(1+i)^{21}$
%		\item $(-\sqrt{3}+i)^5$
%	\end{enumerate}
%	\begin{opl}\mbox{}
%		\begin{enumerate}
%			\item $-1024$
%			\item $-1024-1024i$
%			\item $16\sqrt{3}+16i$
%		\end{enumerate}
%	\end{opl}
%\end{oefening2}
%
%\begin{oefening2}
%	Los op in $\C$ en geef de ontbinding in factoren.
%	\begin{enumerate}
%		\item $-x^2+x-2=0\qquad$
%		\item  $ 2x^2-10x+13=0\qquad$
%		\item  $ 3x^2-5x+7=0\qquad$
%	\end{enumerate}
%	
%	\begin{opl}
%		\begin{enumerate}
%			\item $x_{1,2}=\frac{1\pm i \sqrt{7} }{2}$, de ontbinding is
%			$-x^2+x-2=-1(x-\frac{1+ i \sqrt{7}}{2})(x-\frac{1- i
%				\sqrt{7}}{2})$
%			\item $x_{1,2}=\frac{5\pm i}{2}$, de ontbinding is
%			$2x^2-10x+13=2(x-\frac{5+ i}{2})(x-\frac{5-i}{2})$
%			\item $x_{1,2}=\frac{5\pm i \sqrt{59} }{6}$, de ontbinding is
%			$3x^2-5x+7=3(x-\frac{5+ i \sqrt{59} }{6})(x-\frac{5- i \sqrt{59}
%			}{6})$
%		\end{enumerate}
%	\end{opl}
%\end{oefening2}
%
%\begin{oefening2}
%	Zoek een vierkantsvergelijking met re\"ele co\"effici\"enten
%	waarvan $3+2i$  een wortel is.
%	\begin{opl}
%		$x^2-6x+13=0$
%	\end{opl}
%\end{oefening2}
%
%\begin{oefening2}
%	Zoek alle complexe wortels van volgende vergelijking
%	$(x^2+5)(x^3+x-2)=0.$
%	\begin{opl}
%		$\pm\sqrt{5}\;i, 1,\ds \frac{-1\pm\sqrt{7}\;i}{2}$
%	\end{opl}
%\end{oefening2}
%
%
%\begin{oefening2}
%	Los de binomiaalvergelijking $z^3=1+i$ op. Stel de
%	oplossingen voor in het complex vlak.
%\end{oefening2}
%
%\begin{oefening2}
%	Beschouw een $z\in\C$ met $|z|=1$.
%	\begin{enumerate}
%		\item Toon aan dat $\left(\ds\frac{1+z}{|1+z|}\right)^2=z.$
%		\item Kan je dat resultaat ook meetkundig verklaren? Teken daartoe
%		$z$ (ergens op de eenheidscirkel), waar ligt dan $1+z$ en $\ds
%		\ds\frac{1+z}{|1+z|}\;\dots$?
%	\end{enumerate}
%\end{oefening2}
%
%




\end{document}
