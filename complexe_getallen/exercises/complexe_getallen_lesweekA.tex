\documentclass{ximera}
%\handouttrue
%%% Begin Laad packages

\makeatletter
\@ifclassloaded{xourse}{%
    \typeout{Start loading preamble.tex (in a XOURSE)}%
    \def\isXourse{true}   % automatically defined; pre 112022 it had to be set 'manually' in a xourse
}{%
    \typeout{Start loading preamble.tex (NOT in a XOURSE)}%
}
\makeatother

\pgfplotsset{compat=1.16}

\usepackage{currfile}

% 201908/202301: PAS OP: babel en doclicense lijken problemen te veroorzaken in .jax bestand
% (wegens syntax error met toegevoegde \newcommands ...)
\pdfOnly{
    \usepackage[type={CC},modifier={by-nc-sa},version={4.0}]{doclicense}
    \usepackage[dutch]{babel}
}



\usepackage[utf8]{inputenc}
\usepackage{morewrites}   % nav zomercursus (answer...?)
\usepackage{multirow}
\usepackage{multicol}
\usepackage{tikzsymbols}
\usepackage{ifthen}
%\usepackage{animate} BREAKS HTML STRUCTURE USED BY XIMERA
\usepackage{relsize}

\usepackage{eurosym}    % \euro  (€ werkt niet in xake ...?)

% Nuttig als ook interactieve beamer slides worden voorzien:
\providecommand{\p}{} % default nothing ; potentially usefull for slides: redefine as \pause
%providecommand{\p}{\pause}

\usepackage{caption} % captionof
%\usepackage{pdflscape}    % landscape environment

% Met "\newcommand\showtodonotes{}" kan je todonotes tonen (in pdf/online)
% 201908: online werkt het niet (goed)
\providecommand\showtodonotes{disable}
\providecommand\todo[1]{\typeout{TODO #1}}
%\usepackage[\showtodonotes]{todonotes}
%\usepackage{todonotes}

%
% Poging tot aanpassen layout
%
\usepackage{tcolorbox}
\tcbuselibrary{theorems}

%%% Einde laad packages

%%% Begin Ximera specifieke zaken

\graphicspath{
	{../../}
	{../}
	{./}
  	{../../pictures/}
   	{../pictures/}
   	{./pictures/}
	{./explog/}    % M05 in groeimodellen       
}

%%% Einde Ximera specifieke zaken

%
% define softer blue/red/green, use KU Leuven base colors for blue (and dark orange for red ?)
%
% todo: rather redefine blue/red/green ...?
%\definecolor{xmblue}{rgb}{0.01, 0.31, 0.59}
%\definecolor{xmred}{rgb}{0.89, 0.02, 0.17}
\definecolor{xmdarkblue}{rgb}{0.122, 0.671, 0.835}   % KU Leuven Blauw
\definecolor{xmblue}{rgb}{0.114, 0.553, 0.69}        % KU Leuven Blauw
\definecolor{xmgreen}{rgb}{0.13, 0.55, 0.13}         % No KULeuven variant for green found ...

\definecolor{xmaccent}{rgb}{0.867, 0.541, 0.18}      % KU Leuven Accent (orange ...)
\definecolor{kuaccent}{rgb}{0.867, 0.541, 0.18}      % KU Leuven Accent (orange ...)

\colorlet{xmred}{xmaccent!50!black}                  % Darker version of KU Leuven Accent

\providecommand{\blue}[1]{{\color{blue}#1}}    
\providecommand{\red}[1]{{\color{red}#1}}

\renewcommand\CancelColor{\color{xmaccent!50!black}}

% werkt in math en text mode om MATH met oranje (of grijze...)  achtergond te tonen (ook \important{\text{blabla}} lijkt te werken)
%\newcommand{\important}[1]{\ensuremath{\colorbox{xmaccent!50!white}{$#1$}}}   % werkt niet in Mathjax
%\newcommand{\important}[1]{\ensuremath{\colorbox{lightgray}{$#1$}}}
\newcommand{\important}[1]{\ensuremath{\colorbox{orange}{$#1$}}}   % TODO: kleur aanpassen voor mathjax; wordt overschreven infra!


% Uitzonderlijk kan met \pdfnl in de PDF een newline worden geforceerd, die online niet nodig/nuttig is omdat daar de regellengte hoe dan ook niet gekend is.
\ifdefined\HCode%
\providecommand{\pdfnl}{}%
\else%
\providecommand{\pdfnl}{%
  \\%
}%
\fi

% Uitzonderlijk kan met \handoutnl in de handout-PDF een newline worden geforceerd, die noch online noch in de PDF-met-antwoorden nuttig is.
\ifdefined\HCode
\providecommand{\handoutnl}{}
\else
\providecommand{\handoutnl}{%
\ifhandout%
  \nl%
\fi%
}
\fi



% \cellcolor IGNORED by tex4ht ?
% \begin{center} seems not to wordk
    % (missing margin-left: auto;   on tabular-inside-center ???)
%\newcommand{\importantcell}[1]{\ensuremath{\cellcolor{lightgray}#1}}  %  in tabular; usablility to be checked
\providecommand{\importantcell}[1]{\ensuremath{#1}}     % no mathjax2 support for colloring array cells

\pdfOnly{
  \renewcommand{\important}[1]{\ensuremath{\colorbox{kuaccent!50!white}{$#1$}}}
  \renewcommand{\importantcell}[1]{\ensuremath{\cellcolor{kuaccent!40!white}#1}}   
}

%%% Tikz styles


\pgfplotsset{compat=1.16}

\usetikzlibrary{trees,positioning,arrows,fit,shapes,math,calc,decorations.markings,through,intersections,patterns,matrix}

\usetikzlibrary{decorations.pathreplacing,backgrounds}    % 5/2023: from experimental


\usetikzlibrary{angles,quotes}

\usepgfplotslibrary{fillbetween} % bepaalde_integraal
\usepgfplotslibrary{polar}    % oa voor poolcoordinaten.tex

\pgfplotsset{ownstyle/.style={axis lines = center, axis equal image, xlabel = $x$, ylabel = $y$, enlargelimits}} 

\pgfplotsset{
	plot/.style={no marks,samples=50}
}

\newcommand{\xmPlotsColor}{
	\pgfplotsset{
		plot1/.style={darkgray,no marks,samples=100},
		plot2/.style={lightgray,no marks,samples=100},
		plotresult/.style={blue,no marks,samples=100},
		plotblue/.style={blue,no marks,samples=100},
		plotred/.style={red,no marks,samples=100},
		plotgreen/.style={green,no marks,samples=100},
		plotpurple/.style={purple,no marks,samples=100}
	}
}
\newcommand{\xmPlotsBlackWhite}{
	\pgfplotsset{
		plot1/.style={black,loosely dashed,no marks,samples=100},
		plot2/.style={black,loosely dotted,no marks,samples=100},
		plotresult/.style={black,no marks,samples=100},
		plotblue/.style={black,no marks,samples=100},
		plotred/.style={black,dotted,no marks,samples=100},
		plotgreen/.style={black,dashed,no marks,samples=100},
		plotpurple/.style={black,dashdotted,no marks,samples=100}
	}
}


\newcommand{\xmPlotsColorAndStyle}{
	\pgfplotsset{
		plot1/.style={darkgray,no marks,samples=100},
		plot2/.style={lightgray,no marks,samples=100},
		plotresult/.style={blue,no marks,samples=100},
		plotblue/.style={xmblue,no marks,samples=100},
		plotred/.style={xmred,dashed,thick,no marks,samples=100},
		plotgreen/.style={xmgreen,dotted,very thick,no marks,samples=100},
		plotpurple/.style={purple,no marks,samples=100}
	}
}


%\iftikzexport
\xmPlotsColorAndStyle
%\else
%\xmPlotsBlackWhite
%\fi
%%%


%
% Om venndiagrammen te arceren ...
%
\makeatletter
\pgfdeclarepatternformonly[\hatchdistance,\hatchthickness]{north east hatch}% name
{\pgfqpoint{-1pt}{-1pt}}% below left
{\pgfqpoint{\hatchdistance}{\hatchdistance}}% above right
{\pgfpoint{\hatchdistance-1pt}{\hatchdistance-1pt}}%
{
	\pgfsetcolor{\tikz@pattern@color}
	\pgfsetlinewidth{\hatchthickness}
	\pgfpathmoveto{\pgfqpoint{0pt}{0pt}}
	\pgfpathlineto{\pgfqpoint{\hatchdistance}{\hatchdistance}}
	\pgfusepath{stroke}
}
\pgfdeclarepatternformonly[\hatchdistance,\hatchthickness]{north west hatch}% name
{\pgfqpoint{-\hatchthickness}{-\hatchthickness}}% below left
{\pgfqpoint{\hatchdistance+\hatchthickness}{\hatchdistance+\hatchthickness}}% above right
{\pgfpoint{\hatchdistance}{\hatchdistance}}%
{
	\pgfsetcolor{\tikz@pattern@color}
	\pgfsetlinewidth{\hatchthickness}
	\pgfpathmoveto{\pgfqpoint{\hatchdistance+\hatchthickness}{-\hatchthickness}}
	\pgfpathlineto{\pgfqpoint{-\hatchthickness}{\hatchdistance+\hatchthickness}}
	\pgfusepath{stroke}
}
%\makeatother

\tikzset{
    hatch distance/.store in=\hatchdistance,
    hatch distance=10pt,
    hatch thickness/.store in=\hatchthickness,
   	hatch thickness=2pt
}

\colorlet{circle edge}{black}
\colorlet{circle area}{blue!20}


\tikzset{
    filled/.style={fill=green!30, draw=circle edge, thick},
    arceerl/.style={pattern=north east hatch, pattern color=blue!50, draw=circle edge},
    arceerr/.style={pattern=north west hatch, pattern color=yellow!50, draw=circle edge},
    outline/.style={draw=circle edge, thick}
}




%%% Updaten commando's
\def\hoofding #1#2#3{\maketitle}     % OBSOLETE ??

% we willen (bijna) altijd \geqslant ipv \geq ...!
\newcommand{\geqnoslant}{\geq}
\renewcommand{\geq}{\geqslant}
\newcommand{\leqnoslant}{\leq}
\renewcommand{\leq}{\leqslant}

% Todo: (201908) waarom komt er (soms) underlined voor emph ...?
\renewcommand{\emph}[1]{\textit{#1}}

% API commando's

\newcommand{\ds}{\displaystyle}
\newcommand{\ts}{\textstyle}  % tegenhanger van \ds   (Ximera zet PER  DEFAULT \ds!)

% uit Zomercursus-macro's: 
\newcommand{\bron}[1]{\begin{scriptsize} \emph{#1} \end{scriptsize}}     % deprecated ...?


%definities nieuwe commando's - afkortingen veel gebruikte symbolen
\newcommand{\R}{\ensuremath{\mathbb{R}}}
\newcommand{\Rnul}{\ensuremath{\mathbb{R}_0}}
\newcommand{\Reen}{\ensuremath{\mathbb{R}\setminus\{1\}}}
\newcommand{\Rnuleen}{\ensuremath{\mathbb{R}\setminus\{0,1\}}}
\newcommand{\Rplus}{\ensuremath{\mathbb{R}^+}}
\newcommand{\Rmin}{\ensuremath{\mathbb{R}^-}}
\newcommand{\Rnulplus}{\ensuremath{\mathbb{R}_0^+}}
\newcommand{\Rnulmin}{\ensuremath{\mathbb{R}_0^-}}
\newcommand{\Rnuleenplus}{\ensuremath{\mathbb{R}^+\setminus\{0,1\}}}
\newcommand{\N}{\ensuremath{\mathbb{N}}}
\newcommand{\Nnul}{\ensuremath{\mathbb{N}_0}}
\newcommand{\Z}{\ensuremath{\mathbb{Z}}}
\newcommand{\Znul}{\ensuremath{\mathbb{Z}_0}}
\newcommand{\Zplus}{\ensuremath{\mathbb{Z}^+}}
\newcommand{\Zmin}{\ensuremath{\mathbb{Z}^-}}
\newcommand{\Znulplus}{\ensuremath{\mathbb{Z}_0^+}}
\newcommand{\Znulmin}{\ensuremath{\mathbb{Z}_0^-}}
\newcommand{\C}{\ensuremath{\mathbb{C}}}
\newcommand{\Cnul}{\ensuremath{\mathbb{C}_0}}
\newcommand{\Cplus}{\ensuremath{\mathbb{C}^+}}
\newcommand{\Cmin}{\ensuremath{\mathbb{C}^-}}
\newcommand{\Cnulplus}{\ensuremath{\mathbb{C}_0^+}}
\newcommand{\Cnulmin}{\ensuremath{\mathbb{C}_0^-}}
\newcommand{\Q}{\ensuremath{\mathbb{Q}}}
\newcommand{\Qnul}{\ensuremath{\mathbb{Q}_0}}
\newcommand{\Qplus}{\ensuremath{\mathbb{Q}^+}}
\newcommand{\Qmin}{\ensuremath{\mathbb{Q}^-}}
\newcommand{\Qnulplus}{\ensuremath{\mathbb{Q}_0^+}}
\newcommand{\Qnulmin}{\ensuremath{\mathbb{Q}_0^-}}

\newcommand{\perdef}{\overset{\mathrm{def}}{=}}
\newcommand{\pernot}{\overset{\mathrm{notatie}}{=}}
\newcommand\perinderdaad{\overset{!}{=}}     % voorlopig gebruikt in limietenrekenregels
\newcommand\perhaps{\overset{?}{=}}          % voorlopig gebruikt in limietenrekenregels

\newcommand{\degree}{^\circ}


\DeclareMathOperator{\dom}{dom}     % domein
\DeclareMathOperator{\codom}{codom} % codomein
\DeclareMathOperator{\bld}{bld}     % beeld
\DeclareMathOperator{\graf}{graf}   % grafiek
\DeclareMathOperator{\rico}{rico}   % richtingcoëfficient
\DeclareMathOperator{\co}{co}       % coordinaat
\DeclareMathOperator{\gr}{gr}       % graad

\newcommand{\func}[5]{\ensuremath{#1: #2 \rightarrow #3: #4 \mapsto #5}} % Easy to write a function


% Operators
\DeclareMathOperator{\bgsin}{bgsin}
\DeclareMathOperator{\bgcos}{bgcos}
\DeclareMathOperator{\bgtan}{bgtan}
\DeclareMathOperator{\bgcot}{bgcot}
\DeclareMathOperator{\bgsinh}{bgsinh}
\DeclareMathOperator{\bgcosh}{bgcosh}
\DeclareMathOperator{\bgtanh}{bgtanh}
\DeclareMathOperator{\bgcoth}{bgcoth}

% Oude \Bgsin etc deprecated: gebruik \bgsin, en herdefinieer dat als je Bgsin wil!
%\DeclareMathOperator{\cosec}{cosec}    % not used? gebruik \csc en herdefinieer

% operatoren voor differentialen: to be verified; 1/2020: inconsequent gebruik bij afgeleiden/integralen
\renewcommand{\d}{\mathrm{d}}
\newcommand{\dx}{\d x}
\newcommand{\dd}[1]{\frac{\mathrm{d}}{\mathrm{d}#1}}
\newcommand{\ddx}{\dd{x}}

% om in voorbeelden/oefeningen de notatie voor afgeleiden te kunnen kiezen
% Usage: \afg{(2\sin(x))}  (en wordt d/dx, of accent, of D )
\newcommand{\afg}[1]{{#1}'}
%\renewcommand{\afg}[1]{\frac{\mathrm{d}#1}{\mathrm{d}x}}   % include in relevant exercises ...
%\renewcommand{\afg}[1]{D{#1}}

%
% \xmxxx commands: Extra KU Leuven functionaliteit van, boven of naast Ximera
%   ( Conventie 8/2019: xm+nederlandse omschrijving, maar is niet consequent gevolgd, en misschien ook niet erg handig !)
%
% (Met een minimale ximera.cls en preamble.tex zou een bruikbare .pdf moeten kunnen worden gemaakt van eender welke ximera)
%
% Usage: \xmtitle[Mijn korte abstract]{Mijn titel}{Mijn abstract}
% Eerste command na \begin{document}:
%  -> definieert de \title
%  -> definieert de abstract
%  -> doet \maketitle ( dus: print de hoofding als 'chapter' of 'sectie')
% Optionele parameter geeft eenn kort abstract (die met de globale setting \xmshortabstract{} al dan niet kan worden geprint.
% De optionele korte abstract kan worden gebruikt voor pseudo-grappige abtsarts, dus dus globaal al dan niet kunnen worden gebuikt...
% Globale settings:
%  de (optionele) 'korte abstract' wordt enkele getoond als \xmshortabstract is gezet
\providecommand\xmshortabstract{} % default: print (only!) short abstract if present
\providecommand\theabstract{} % otherwise complaint Undefined control sequence.  <recently read> \theabstract  ????
\newcommand{\xmtitle}[3][]{
	\title{#2}
	% \begin{abstract}
	% 			\ifdefined\xmshortabstract
	% 			\ifstrempty{#1}{%
	% 						#3
	% 			}{%
	% 						#1
	% 			}%
	% 			\else
	% 			#3
	% 			\fi
	% \end{abstract}
	\maketitle
}

% 
% Kleine grapjes: moeten zonder verder gevolg kunnen worden verwijderd
%
%\newcommand{\xmopje}[1]{{\small#1{\reversemarginpar\marginpar{\Smiley}}}}   % probleem in floats!!
\newtoggle{showxmopje}
\toggletrue{showxmopje}

\newcommand{\xmopje}[1]{%
   \iftoggle{showxmopje}{#1}{}%
}


% -> geef een abstracte-formule-met-rechts-een-concreet-voorbeeld
% VB:  \formulevb{a^2+b^2=c^2}{3^2+4^2=5^2}
%
\ifdefined\HCode
\NewEnviron{xmdiv}[1]{\HCode{\Hnewline<div class="#1">\Hnewline}\BODY{\HCode{\Hnewline</div>\Hnewline}}}
\else
\NewEnviron{xmdiv}[1]{\BODY}
\fi

\providecommand{\formulevb}[2]{
	{\centering

    \begin{xmdiv}{xmformulevb}    % zie css voor online layout !!!
	\begin{tabular}{lcl}
		\important{#1}
		&  &
		Vb: $#2$
		\end{tabular}
	\end{xmdiv}

	}
}

\ifdefined\HCode
\providecommand{\xmcolorbox}[2]{
	\HCode{\Hnewline<div class="xmcolorbox">\Hnewline}#2\HCode{\Hnewline</div>\Hnewline}
}
\else
\providecommand{\xmcolorbox}[2]{
  \cellcolor{#1}#2
}
\fi


\ifdefined\HCode
\providecommand{\xmopmerking}[1]{
 \HCode{\Hnewline<div class="xmopmerking">\Hnewline}#1\HCode{\Hnewline</div>\Hnewline}
}
\else
\providecommand{\xmopmerking}[1]{
	{\footnotesize #1}
}
\fi
% \providecommand{\voorbeeld}[1]{
% 	\colorbox{blue!10}{$#1$}
% }



% Hernoem Proof naar Bewijs, nodig voor HTML versie
\renewcommand*{\proofname}{Bewijs}

% Om opgave van oefening (wordt niet geprint bij oplossingenblad)
% (to be tested test)
\NewEnviron{statement}{\BODY}

% Environment 'oplossing' en 'uitkomst'
% voor resp. volledige 'uitwerking' dan wel 'enkel eindresultaat'
% geimplementeerd via feedback, omdat er in de ximera-server adhoc feedback-code is toegevoegd
%% Niet tonen indien handout
%% Te gebruiken om volledige oplossingen/uitwerkingen van oefeningen te tonen
%% \begin{oplossing}        De optelling is commutatief \end{oplossing}  : verschijnt online enkel 'op vraag'
%% \begin{oplossing}[toon]  De optelling is commutatief \end{oplossing}  : verschijnt steeds onmiddellijk online (bv te gebruiken bij voorbeelden) 

\ifhandout%
    \NewEnviron{oplossing}[1][onzichtbaar]%
    {%
    \ifthenelse{\equal{\detokenize{#1}}{\detokenize{toon}}}
    {
    \def\PH@Command{#1}% Use PH@Command to hold the content and be a target for "\expandafter" to expand once.

    \begin{trivlist}% Begin the trivlist to use formating of the "Feedback" label.
    \item[\hskip \labelsep\small\slshape\bfseries Oplossing% Format the "Feedback" label. Don't forget the space.
    %(\texttt{\detokenize\expandafter{\PH@Command}}):% Format (and detokenize) the condition for feedback to trigger
    \hspace{2ex}]\small%\slshape% Insert some space before the actual feedback given.
    \BODY
    \end{trivlist}
    }
    {  % \begin{feedback}[solution]   \BODY     \end{feedback}  }
    }
    }    
\else
% ONLY for HTML; xmoplossing is styled with css, and is not, and need not be a LaTeX environment
% THUS: it does NOT use feedback anymore ...
%    \NewEnviron{oplossing}{\begin{expandable}{xmoplossing}{\nlen{Toon uitwerking}{Show solution}}{\BODY}\end{expandable}}
    \newenvironment{oplossing}[1][onzichtbaar]
   {%
       \begin{expandable}{xmoplossing}{}
   }
   {%
   	   \end{expandable}
   } 
%     \newenvironment{oplossing}[1][onzichtbaar]
%    {%
%        \begin{feedback}[solution]   	
%    }
%    {%
%    	   \end{feedback}
%    } 
\fi

\ifhandout%
    \NewEnviron{uitkomst}[1][onzichtbaar]%
    {%
    \ifthenelse{\equal{\detokenize{#1}}{\detokenize{toon}}}
    {
    \def\PH@Command{#1}% Use PH@Command to hold the content and be a target for "\expandafter" to expand once.

    \begin{trivlist}% Begin the trivlist to use formating of the "Feedback" label.
    \item[\hskip \labelsep\small\slshape\bfseries Uitkomst:% Format the "Feedback" label. Don't forget the space.
    %(\texttt{\detokenize\expandafter{\PH@Command}}):% Format (and detokenize) the condition for feedback to trigger
    \hspace{2ex}]\small%\slshape% Insert some space before the actual feedback given.
    \BODY
    \end{trivlist}
    }
    {  % \begin{feedback}[solution]   \BODY     \end{feedback}  }
    }
    }    
\else
\ifdefined\HCode
   \newenvironment{uitkomst}[1][onzichtbaar]
    {%
        \begin{expandable}{xmuitkomst}{}%
    }
    {%
    	\end{expandable}%
    } 
\else
  % Do NOT print 'uitkomst' in non-handout
  %  (presumably, there is also an 'oplossing' ??)
  \newenvironment{uitkomst}[1][onzichtbaar]{}{}
\fi
\fi

%
% Uitweidingen zijn extra's die niet redelijkerwijze tot de leerstof behoren
% Uitbreidingen zijn extra's die wel redelijkerwijze tot de leerstof van bv meer geavanceerde versies kunnen behoren (B-programma/Wiskundestudenten/...?)
% Nog niet voorzien: design voor verschillende versies (A/B programma, BIO, voorkennis/ ...)
% Voor 'uitweidingen' is er een environment die online per default is ingeklapt, en in pdf al dan niet kan worden geincluded  (via \xmnouitweiding) 
%
% in een xourse, per default GEEN uitweidingen, tenzij \xmuitweiding expliciet ergens is gezet ...
\ifdefined\isXourse
   \ifdefined\xmuitweiding
   \else
       \def\xmnouitweiding{true}
   \fi
\fi

\ifdefined\xmnouitweiding
\newcounter{xmuitweiding}  % anders error undefined ...  
\excludecomment{xmuitweiding}
\else
\newtheoremstyle{dotless}{}{}{}{}{}{}{ }{}
\theoremstyle{dotless}
\newtheorem*{xmuitweidingnofrills}{}   % nofrills = no accordion; gebruikt dus de dotless theoremstyle!

\newcounter{xmuitweiding}
\newenvironment{xmuitweiding}[1][ ]%
{% 
	\refstepcounter{xmuitweiding}%
    \begin{expandable}{xmuitweiding}{Uitweiding \arabic{xmuitweiding}: #1}%
	\begin{xmuitweidingnofrills}%
}
{%
    \end{xmuitweidingnofrills}%
    \end{expandable}%
}   
% \newenvironment{xmuitweiding}[1][ ]%
% {% 
% 	\refstepcounter{xmuitweiding}
% 	\begin{accordion}\begin{accordion-item}[Uitweiding \arabic{xmuitweiding}: #1]%
% 			\begin{xmuitweidingnofrills}%
% 			}
% 			{\end{xmuitweidingnofrills}\end{accordion-item}\end{accordion}}   
\fi


\newenvironment{xmexpandable}[1][]{
	\begin{accordion}\begin{accordion-item}[#1]%
		}{\end{accordion-item}\end{accordion}}


% Command that gives a selection box online, but just prints the right answer in pdf
\newcommand{\xmonlineChoice}[1]{\pdfOnly{\wordchoicegiventrue}\wordChoice{#1}\pdfOnly{\wordchoicegivenfalse}}   % deprecated, gebruik onlineChoice ...
\newcommand{\onlineChoice}[1]{\pdfOnly{\wordchoicegiventrue}\wordChoice{#1}\pdfOnly{\wordchoicegivenfalse}}

\newcommand{\TJa}{\nlentext{ Ja }{ Yes }}
\newcommand{\TNee}{\nlentext{ Nee }{ No }}
\newcommand{\TJuist}{\nlentext{ Juist }{ True }}
\newcommand{\TFout}{\nlentext{ Fout }{ False }}

\newcommand{\choiceTrue}{{\wordChoice{\choice[correct]{\TJuist}\choice{\TFout}}}}
\newcommand{\choiceFalse}{{\wordChoice{\choice{\TJuist}\choice[correct]{\TFout}}}}

\newcommand{\choiceYes}{{\wordChoice{\choice[correct]{\TJa}\choice{\TNee}}}}
\newcommand{\choiceNo}{{\wordChoice{\choice{\TJa}\choice[correct]{\TNee}}}}

\newcommand{\choiceEen}{{\wordChoice{\choice[correct]{een }\choice{geen }}}}
\newcommand{\choiceGeen}{{\wordChoice{\choice{een }\choice[correct]{geen }}}}

% Optional nicer formatting for wordChoice in PDF

\let\inlinechoiceorig\inlinechoice

%\makeatletter
%\providecommand{\choiceminimumverticalsize}{\vphantom{$\frac{\sqrt{2}}{2}$}}   % minimum height of boxes (cfr infra)
\providecommand{\choiceminimumverticalsize}{\vphantom{$\tfrac{2}{2}$}}   % minimum height of boxes (cfr infra)

\newcommand{\inlinechoicesquares}[2][]{%
		\setkeys{choice}{#1}%
		\ifthenelse{\boolean{\choice@correct}}%
		{%
            \ifhandout%
               \fbox{\choiceminimumverticalsize #2}\allowbreak\ignorespaces%
            \else%
               \fcolorbox{blue}{blue!20}{\choiceminimumverticalsize #2\checkmark}\allowbreak\ignorespaces\setkeys{choice}{correct=false}\ignorespaces%
            \fi%
		}%
		{% else
			\fbox{\choiceminimumverticalsize #2}\allowbreak\ignorespaces%  HACK: wat kleiner, zodat fits on line ... 	
		}%
}

\newcommand{\inlinechoicesquareX}[2][]{%
		\setkeys{choice}{#1}%
		\ifthenelse{\boolean{\choice@correct}}%
		{%
            \ifhandout%
               \fbox{\choiceminimumverticalsize #2}\allowbreak\ignorespaces\setkeys{choice}{correct=false}\ignorespaces%
            \else%
               \fcolorbox{blue}{blue!20}{\choiceminimumverticalsize #2\checkmark}\allowbreak\ignorespaces\setkeys{choice}{correct=false}\ignorespaces%
            \fi%
		}%
		{% else
        \ifhandout%
			\fbox{\choiceminimumverticalsize #2}\allowbreak\ignorespaces%  HACK: wat kleiner, zodat fits on line ... 	
        \fi
		}%
}


\newcommand{\inlinechoicepuntjes}[2][]{%
		\setkeys{choice}{#1}%
		\ifthenelse{\boolean{\choice@correct}}%
		{%
            \ifhandout%
               \dots\ldots\ignorespaces\setkeys{choice}{correct=false}\ignorespaces
            \else%
               \fcolorbox{blue}{blue!20}{\choiceminimumverticalsize #2}\allowbreak\ignorespaces\setkeys{choice}{correct=false}\ignorespaces%
            \fi%
		}%
		{% else
			%\fbox{\choiceminimumverticalsize #2}\allowbreak\ignorespaces%  HACK: wat kleiner, zodat fits on line ... 	
		}%
}

% print niets, maar definieer globale variable \myanswer
%  (gebruikt om oplossingsbladen te printen) 
\newcommand{\inlinechoicedefanswer}[2][]{%
		\setkeys{choice}{#1}%
		\ifthenelse{\boolean{\choice@correct}}%
		{%
               \gdef\myanswer{#2}\setkeys{choice}{correct=false}

		}%
		{% else
			%\fbox{\choiceminimumverticalsize #2}\allowbreak\ignorespaces%  HACK: wat kleiner, zodat fits on line ... 	
		}%
}



%\makeatother

\newcommand{\setchoicedefanswer}{
\ifdefined\HCode
\else
%    \renewenvironment{multipleChoice@}[1][]{}{} % remove trailing ')'
    \let\inlinechoice\inlinechoicedefanswer
\fi
}

\newcommand{\setchoicepuntjes}{
\ifdefined\HCode
\else
    \renewenvironment{multipleChoice@}[1][]{}{} % remove trailing ')'
    \let\inlinechoice\inlinechoicepuntjes
\fi
}
\newcommand{\setchoicesquares}{
\ifdefined\HCode
\else
    \renewenvironment{multipleChoice@}[1][]{}{} % remove trailing ')'
    \let\inlinechoice\inlinechoicesquares
\fi
}
%
\newcommand{\setchoicesquareX}{
\ifdefined\HCode
\else
    \renewenvironment{multipleChoice@}[1][]{}{} % remove trailing ')'
    \let\inlinechoice\inlinechoicesquareX
\fi
}
%
\newcommand{\setchoicelist}{
\ifdefined\HCode
\else
    \renewenvironment{multipleChoice@}[1][]{}{)}% re-add trailing ')'
    \let\inlinechoice\inlinechoiceorig
\fi
}

\setchoicesquareX  % by default list-of-squares with onlineChoice in PDF

% Omdat multicols niet werkt in html: enkel in pdf  (in html zijn langere pagina's misschien ook minder storend)
\newenvironment{xmmulticols}[1][2]{
 \pdfOnly{\begin{multicols}{#1}}%
}{ \pdfOnly{\end{multicols}}}

%
% Te gebruiken in plaats van \section\subsection
%  (in een printstyle kan dan het level worden aangepast
%    naargelang \chapter vs \section style )
% 3/2021: DO NOT USE \xmsubsection !
\newcommand\xmsection\subsection
\newcommand\xmsubsection\subsubsection

% Aanpassen printversie
%  (hier gedefinieerd, zodat ze in xourse kunnen worden gezet/overschreven)
\providebool{parttoc}
\providebool{printpartfrontpage}
\providebool{printactivitytitle}
\providebool{printactivityqrcode}
\providebool{printactivityurl}
\providebool{printcontinuouspagenumbers}

% The following three commands are hardcoded in xake, you can't create other commands like these, without adding them to xake as well
%  ( gebruikt in xourses om juiste soort titelpagina te krijgen voor verschillende ximera's )
\newcommand{\activitychapter}[1]{
	\typeout{ACTIVITYCHAPTER #1}   % logging
	\chapterstyle
	\activity{#1}
}
\newcommand{\activitysection}[1]{
	\typeout{ACTIVITYSECTION #1}   % logging
	\sectionstyle
	\activity{#1}
}
% Partices worden als activity getoond om de grote blokken te krijgen online
\newcommand{\practicesection}[1]{
	\typeout{PRACTICESECTION #1}   % logging
	\sectionstyle
	\activity{#1}
}


% Commando om de printstyle toe te voegen in ximera's. Zorgt ervoor dat er geen problemen zijn als je de xourses compileert
% hack om onhandige relative paden in TeX te omzeilen
% should work both in xourse and ximera (pre-112022 only in ximera; thus obsoletes adhoc setup in xourses)
% loads global.sty if present (cfr global.css for online settings!)
% use global.sty to overwrite settings in printstyle.sty ...
\newcommand{\addPrintStyle}[1]{
\iftikzexport\else   % only in PDF
  \makeatletter
  \ifx\@onlypreamble\@notprerr\else   % ONLY if in tex-preamble   (and e.g. not when included from xourse)
    \typeout{Loading printstyle}   % logging
    \usepackage{#1/printstyle} % mag enkel geinclude worden als je die apart compileert
    \IfFileExists{#1/global.sty}{
        \typeout{Loading printstyle-folder #1/global.sty}   % logging
        \usepackage{#1/global}
        }{
        \typeout{Info: No extra #1/global.sty}   % logging
    }   % load global.sty if present
    \IfFileExists{global.sty}{
        \typeout{Loading local-folder global.sty (or TEXINPUTPATH..)}   % logging
        \usepackage{global}
    }{
        \typeout{Info: No folder/global.sty}   % logging
    }   % load global.sty if present
    \IfFileExists{\currfilebase.sty}
    {
        \typeout{Loading \currfilebase.sty}
        \input{\currfilebase.sty}
    }{
        \typeout{Info: No local \currfilebase.sty}
    }
    \fi
  \makeatother
\fi
}

%
%  
% references: Ximera heeft adhoc logica	 om online labels te doen werken over verschillende files heen
% met \hyperref kan de getoonde tekst toch worden opgegeven, in plaats van af te hangen van de label-text
\ifdefined\HCode
% Link to standard \labels, but give your own description
% Usage:  Volg \hyperref[my_very_verbose_label]{deze link} voor wat tijdverlies
%   (01/2020: Ximera-server aangepast om bij class reference-keeptext de link-text NIET te vervangen door de label-text !!!) 
\renewcommand{\hyperref}[2][]{\HCode{<a class="reference reference-keeptext" href="\##1">}#2\HCode{</a>}}
%
%  Link to specific targets  (not tested ?)
\renewcommand{\hypertarget}[1]{\HCode{<a class="ximera-label" id="#1"></a>}}
\renewcommand{\hyperlink}[2]{\HCode{<a class="reference reference-keeptext" href="\##1">}#2\HCode{</a>}}
\fi


\renewcommand{\figurename}{Figuur}
\renewcommand{\tablename}{Tabel}

%
% Gedoe om verschillende versies van xourse/ximera te maken afhankelijk van settings
%
% default: versie met antwoorden
% handout: versie voor de studenten, zonder antwoorden/oplossingen
% full: met alles erop en eraan, dus geschikt voor auteurs en/of lesgevers  (bevat in de pdf ook de 'online-only' stukken!)
%
%
% verder kunnen ook opties/variabele worden gezet voor hints/auteurs/uitweidingen/ etc
%
% 'Full' versie
\newtoggle{showonline}
\ifdefined\HCode   % zet default showOnline
    \toggletrue{showonline} 
\else
    \togglefalse{showonline}
\fi

% Full versie   % deprecated: see infra
\newcommand{\printFull}{
    \hintstrue
    \handoutfalse
    \toggletrue{showonline} 
}

\ifdefined\shouldPrintFull   % deprecated: see infra
    \printFull
\fi

%% \onlineOnly kan jammer genoeg niet, omdat het al betsaat als neveneffect van \begin{onlineOnly} ...
\newcommand{\onlyOnline}[1]{\ifdefined\HCode#1\fi}

% Overschrijf onlineOnly  (zoals gedefinieerd in ximera.cls)
\ifhandout   % in handout: gebruik de oorspronkelijke ximera.cls implementatie  (is dit wel nodig/nuttig?)
\else
    \iftoggle{showonline}{%
        \ifdefined\HCode
          \RenewEnviron{onlineOnly}{\bgroup\BODY\egroup}   % showOnline, en we zijn  online, dus toon de tekst
        \else
          \RenewEnviron{onlineOnly}{\bgroup\color{red!50!black}\BODY\egroup}   % showOnline, maar we zijn toch niet online: kleur de tekst rood 
        \fi
    }{%
      \RenewEnviron{onlineOnly}{\setbox0\vbox\bgroup\BODY\egroup}% geen showOnline
    }
\fi

% hack om na hoofding van definition/proposition/... als dan niet op een nieuwe lijn te starten
% soms is dat goed en mooi, en soms niet; en in HTML is het nu (2/2020) anders dan in pdf
% vandaar suggestie om 
%     \begin{definition}[Nieuw concept] \nl
% te gebruiken als je zeker een newline wil na de hoofdig en titel
% (in het bijzonder itemize zonder \nl is 'lelijk' ...)
\ifdefined\HCode
\newcommand{\nl}{}
\else
\newcommand{\nl}{\ \par} % newline (achter heading van definition etc.)
\fi


% \nl enkel in handoutmode (ihb voor \wordChoice, die dan typisch veeeel langer wordt)
\ifdefined\HCode
\providecommand{\handoutnl}{}
\else
\providecommand{\handoutnl}{%
\ifhandout%
  \nl%
\fi%
}
\fi

% Could potentially replace \pdfOnline/\begin{onlineOnly} : 
% Usage= \ifonline{Hallo surfer}{Hallo PDFlezer}
\providecommand{\ifonline}[2]%
{
\begin{onlineOnly}#1\end{onlineOnly}%
\pdfOnly{#2}
}%


%
% Maak optionele 'basic' en 'extended' versies van een activity
%  met environment basicOnly, basicSkip en extendedOnly
%
%  (
%   Dit werkt ENKEL in de PDF; de online versies tonen (minstens voorklopig) steeds 
%   het default geval met printbasicversion en printextendversion beide FALSE
%  )
%
\providebool{printbasicversion}
\providebool{printextendedversion}   % not properly implemented
\providebool{printfullversion}       % presumably print everything (debug/auteur)
%
% only set these in xourses, and BEFORE loading this preamble
%
%\newif\ifshowbasic     \showbasictrue        % use this line in xourse to show 'basic' sections
%\newif\ifshowextended  \showextendedtrue     % use this line in xourse to show 'extended' sections
%
%
%\ifbool{showbasic}
%      { \NewEnviron{basicOnly}{\BODY} }    % if yes: just print contents
%      { \NewEnviron{basicOnly}{}      }    % if no:  completely ignore contents
%
%\ifbool{showbasic}
%      { \NewEnviron{basicSkip}{}      }
%      { \NewEnviron{basicSkip}{\BODY} }
%

\ifbool{printextendedversion}
      { \NewEnviron{extendedOnly}{\BODY} }
      { \NewEnviron{extendedOnly}{}      }
      


\ifdefined\HCode    % in html: always print
      \newenvironment*{basicOnly}{}{}    % if yes: just print contents
      \newenvironment*{basicSkip}{}{}    % if yes: just print contents
\else

\ifbool{printbasicversion}
      {\newenvironment*{basicOnly}{}{}}    % if yes: just print contents
      {\NewEnviron{basicOnly}{}      }    % if no:  completely ignore contents

\ifbool{printbasicversion}
      {\NewEnviron{basicSkip}{}      }
      {\newenvironment*{basicSkip}{}{}}

\fi

\usepackage{float}
\usepackage[rightbars,color]{changebar}

% Full versie
\ifbool{printfullversion}{
    \hintstrue
    \handoutfalse
    \toggletrue{showonline}
    \printbasicversionfalse
    \cbcolor{red}
    \renewenvironment*{basicOnly}{\cbstart}{\cbend}
    \renewenvironment*{basicSkip}{\cbstart}{\cbend}
    \def\xmtoonprintopties{FULL}   % will be printed in footer
}
{}
      
%
% Evalueer \ifhints IN de environment
%  
%
%\RenewEnviron{hint}
%{
%\ifhandout
%\ifhints\else\setbox0\vbox\fi%everything in een emty box
%\bgroup 
%\stepcounter{hintLevel}
%\BODY
%\egroup\ignorespacesafterend
%\addtocounter{hintLevel}{-1}
%\else
%\ifhints
%\begin{trivlist}\item[\hskip \labelsep\small\slshape\bfseries Hint:\hspace{2ex}]
%\small\slshape
%\stepcounter{hintLevel}
%\BODY
%\end{trivlist}
%\addtocounter{hintLevel}{-1}
%\fi
%\fi
%}

% Onafhankelijk van \ifhandout ...? TO BE VERIFIED
\RenewEnviron{hint}
{
\ifhints
\begin{trivlist}\item[\hskip \labelsep\small\bfseries Hint:\hspace{2ex}]
\small%\slshape
\stepcounter{hintLevel}
\BODY
\end{trivlist}
\addtocounter{hintLevel}{-1}
\else
\iftikzexport   % anders worden de tikz tekeningen in hints niet gegenereerd ?
\setbox0\vbox\bgroup
\stepcounter{hintLevel}
\BODY
\egroup\ignorespacesafterend
\addtocounter{hintLevel}{-1}
\fi % ifhandout
\fi %ifhints
}

%
% \tab sets typewriter-tabs (e.g. to format questions)
% (Has no effect in HTML :-( ))
%
\usepackage{tabto}
\ifdefined\HCode
  \renewcommand{\tab}{\quad}    % otherwise dummy .png's are generated ...?
\fi


% Also redefined in  preamble to get correct styling 
% for tikz images for (\tikzexport)
%

\theoremstyle{definition} % Bold titels
\makeatletter
\let\proposition\relax
\let\c@proposition\relax
\let\endproposition\relax
\makeatother
\newtheorem{proposition}{Eigenschap}


%\instructornotesfalse

% logic with \ifhandoutin ximera.cls unclear;so overwrite ...
\makeatletter
\@ifundefined{ifinstructornotes}{%
  \newif\ifinstructornotes
  \instructornotesfalse
  \newenvironment{instructorNotes}{}{}
}{}
\makeatother
\ifinstructornotes
\else
\renewenvironment{instructorNotes}%
{%
    \setbox0\vbox\bgroup
}
{%
    \egroup
}
\fi

% \RedeclareMathOperator
% from https://tex.stackexchange.com/questions/175251/how-to-redefine-a-command-using-declaremathoperator
\makeatletter
\newcommand\RedeclareMathOperator{%
    \@ifstar{\def\rmo@s{m}\rmo@redeclare}{\def\rmo@s{o}\rmo@redeclare}%
}
% this is taken from \renew@command
\newcommand\rmo@redeclare[2]{%
    \begingroup \escapechar\m@ne\xdef\@gtempa{{\string#1}}\endgroup
    \expandafter\@ifundefined\@gtempa
    {\@latex@error{\noexpand#1undefined}\@ehc}%
    \relax
    \expandafter\rmo@declmathop\rmo@s{#1}{#2}}
% This is just \@declmathop without \@ifdefinable
\newcommand\rmo@declmathop[3]{%
    \DeclareRobustCommand{#2}{\qopname\newmcodes@#1{#3}}%
}
\@onlypreamble\RedeclareMathOperator
\makeatother


%
% Engelse vertaling, vooral in mathmode
%
% 1. Algemene procedure
%
\ifdefined\isEn
 \newcommand{\nlen}[2]{#2}
 \newcommand{\nlentext}[2]{\text{#2}}
 \newcommand{\nlentextbf}[2]{\textbf{#2}}
\else
 \newcommand{\nlen}[2]{#1}
 \newcommand{\nlentext}[2]{\text{#1}}
 \newcommand{\nlentextbf}[2]{\textbf{#1}}
\fi

%
% 2. Lijst van erg veel gebruikte uitdrukkingen
%

% Ja/Nee/Fout/Juits etc
%\newcommand{\TJa}{\nlentext{ Ja }{ and }}
%\newcommand{\TNee}{\nlentext{ Nee }{ No }}
%\newcommand{\TJuist}{\nlentext{ Juist }{ Correct }
%\newcommand{\TFout}{\nlentext{ Fout }{ Wrong }
\newcommand{\TWaar}{\nlentext{ Waar }{ True }}
\newcommand{\TOnwaar}{\nlentext{ Vals }{ False }}
% Korte bindwoorden en, of, dus, ...
\newcommand{\Ten}{\nlentext{ en }{ and }}
\newcommand{\Tof}{\nlentext{ of }{ or }}
\newcommand{\Tdus}{\nlentext{ dus }{ so }}
\newcommand{\Tendus}{\nlentext{ en dus }{ and thus }}
\newcommand{\Tvooralle}{\nlentext{ voor alle }{ for all }}
\newcommand{\Took}{\nlentext{ ook }{ also }}
\newcommand{\Tals}{\nlentext{ als }{ when }} %of if?
\newcommand{\Twant}{\nlentext{ want }{ as }}
\newcommand{\Tmaal}{\nlentext{ maal }{ times }}
\newcommand{\Toptellen}{\nlentext{ optellen }{ add }}
\newcommand{\Tde}{\nlentext{ de }{ the }}
\newcommand{\Thet}{\nlentext{ het }{ the }}
\newcommand{\Tis}{\nlentext{ is }{ is }} %zodat is in text staat in mathmode (geen italics)
\newcommand{\Tmet}{\nlentext{ met }{ where }} % in situaties e.g met p < n --> where p < n
\newcommand{\Tnooit}{\nlentext{ nooit }{ never }}
\newcommand{\Tmaar}{\nlentext{ maar }{ but }}
\newcommand{\Tniet}{\nlentext{ niet }{ not }}
\newcommand{\Tuit}{\nlentext{ uit }{ from }}
\newcommand{\Ttov}{\nlentext{ t.o.v. }{ w.r.t. }}
\newcommand{\Tzodat}{\nlentext{ zodat }{ such that }}
\newcommand{\Tdeth}{\nlentext{de }{th }}
\newcommand{\Tomdat}{\nlentext{omdat }{because }} 


%
% Overschrijf addhoc commando's
%
\ifdefined\isEn
\renewcommand{\pernot}{\overset{\mathrm{notation}}{=}}
\RedeclareMathOperator{\bld}{im}     % beeld
\RedeclareMathOperator{\graf}{graph}   % grafiek
\RedeclareMathOperator{\rico}{slope}   % richtingcoëfficient
\RedeclareMathOperator{\co}{co}       % coordinaat
\RedeclareMathOperator{\gr}{deg}       % graad

% Operators
\RedeclareMathOperator{\bgsin}{arcsin}
\RedeclareMathOperator{\bgcos}{arccos}
\RedeclareMathOperator{\bgtan}{arctan}
\RedeclareMathOperator{\bgcot}{arccot}
\RedeclareMathOperator{\bgsinh}{arcsinh}
\RedeclareMathOperator{\bgcosh}{arccosh}
\RedeclareMathOperator{\bgtanh}{arctanh}
\RedeclareMathOperator{\bgcoth}{arccoth}

\fi

\renewcommand{\Im}[1]{\text{Im}#1}
\renewcommand{\Re}[1]{\text{Re}#1}


% Problem-inside-div  (for css styling ...)
\newcommand{\xmdivEnvironmentStart}[3]{%
\ifdefined\HCode%
   \HCode{\Hnewline<div class="#2">}%
\fi%
\problemEnvironmentStart{#1}{#3}%
}


\newcommand{\xmdivEnvironmentEnd}{%
\problemEnvironmentEnd%
\ifdefined\HCode%
    \HCode{\Hnewline</div>}%
\fi%
}


\newenvironment{quickquestion*}[1][2in]%
{%Env start code
\xmdivEnvironmentStart{#1}{quickquestion}{Quick Question}%
}
{%Env end code
\xmdivEnvironmentEnd%
}
\newenvironment{quickquestion}[1][2in]%
{%Env start code
\xmdivEnvironmentStart{#1}{quickquestion}{Quick Question}%
}
{%Env end code
\xmdivEnvironmentEnd%
}

\newenvironment{denkvraag*}[1][2in]%
{%Env start code
\xmdivEnvironmentStart{#1}{denkvraag}{Denkvraag}%
}
{%Env end code
\xmdivEnvironmentEnd
}

\newenvironment{denkvraag}[1][2in]%
{%Env start code
\xmdivEnvironmentStart{#1}{denkvraag}{Denkvraag}%
}
{%Env end code
\xmdivEnvironmentEnd
}


\addPrintStyle{../..}
\begin{document}
    \author{Zomercursus KU Leuven}
    \xmtitle{Oefeningen  Complexe getallen}{}

\providecommand{\shortanswerscols}{3}  % to be removed  when \shortanswercols in preambe.tex ...!

\xmsection{Niveau 1} 

\begin{exercise} 
	\begin{statement}
		Bereken volgende uitdrukkingen. Schrijf de complexe getallen in de vorm $a+bi$.
	\end{statement}	
	\begin{question} $(4+i) + (1-2i) = \answer[onlineshowanswerbutton]{5-i}$
		\begin{oplossing}
			Je kan ofwel reëel en imaginair deel van beide complexe getallen bij elkaar optellen om het antwoord te vinden, ofwel in het complexe vlak de vectorsom maken.
			\begin{image}[0.5\textwidth]
				\begin{tikzpicture}[scale=1]
				% grid setup
				
				\def\xmin{-1}
				\def\xmax{6}
				
				\def\ymin{-2}
				\def\ymax{2}				
				
				% grid
				\draw[step = 1.0, black, thin, dashed] (\xmin, \ymin) grid (\xmax, \ymax);
				\draw[->, thick] (\xmin, 0) -- (\xmax, 0) node[right]{Re$(z)$};
				\draw[->, thick] (0, \ymin) -- (0, \ymax) node[above]{Im$(z)$};;	
				
				% tekening
				\draw[->, red]  (0, 0) -- node[above]{$4+i$}  (4,  1);
				\draw[->, red]  (4, 1) -- node[right]{$1-2i$} (5, -1);
				\draw[->, blue] (0, 0) -- node[below]{$5-i$}  (5, -1);	
				\end{tikzpicture}
			\end{image}
		\end{oplossing}
	\end{question}
	

	\begin{question} $(2+3i)(-5+i) = \answer[onlineshowanswerbutton]{-13-13i}$
		\begin{oplossing}
			$$
			(2+3i)(-5+i) = -5 + 2i + 3i\cdot(-5) 3i\cdot i = -13 - 13i
			$$
			\begin{image}[0.5\textwidth]
				\begin{tikzpicture}[scale=0.5]
				% grid setup
				
				\def\xmin{-14}
				\def\xmax{1}
				
				\def\ymin{-14}
				\def\ymax{1}				
				
				% grid
				\draw[step = 1.0, black, thin, dashed] (\xmin, \ymin) grid (\xmax, \ymax);
				\draw[->, thick] (\xmin, 0) -- (\xmax, 0) node[right]{Re$(z)$};
				\draw[->, thick] (0, \ymin) -- (0, \ymax) node[above]{Im$(z)$};;	
				
				% tekening
				\draw[->, blue] (0, 0) -- node[left]{$-13-13i$}    (-13, -13);	
				\end{tikzpicture}
			\end{image}
		\end{oplossing}
	\end{question}
	
	
	
	\begin{question} $\overline{(3-2i)} = \answer[onlineshowanswerbutton]{3+2i}$
		
	\end{question}
	
	
	
	\begin{question} $|3-2i| = \answer[onlineshowanswerbutton]{\sqrt{13}}$
		\begin{oplossing}
			$|3-2i| = \sqrt{3^2 + 2^2} = \sqrt{13}$ 
		\end{oplossing}
	\end{question}
	
	
	
\end{exercise}

\xmsection{Niveau 2}
\begin{exercise} 
	\begin{statement}
		Bereken volgende uitdrukkingen. Schrijf de complexe getallen in de vorm $a+bi$.
	\end{statement}	
		
		\begin{question} $ (2+4i) - (6-7i) = \answer[onlineshowanswerbutton]{ -4+11i}$
			\begin{oplossing}
				Je kan ofwel reëel en imaginair deel van beide complexe getallen bij elkaar optellen om het antwoord te vinden, ofwel in het complexe vlak de vectorsom maken.
				\begin{image}[0.5\textwidth]
					\begin{tikzpicture}[scale=1]
					% grid setup
					
					\def\xmin{-5}
					\def\xmax{5}
					
					\def\ymin{-1}
					\def\ymax{12}				
					
					% grid
					\draw[step = 1.0, black, thin, dashed] (\xmin, \ymin) grid (\xmax, \ymax);
					\draw[->, thick] (\xmin, 0) -- (\xmax, 0) node[right]{Re$(z)$};
					\draw[->, thick] (0, \ymin) -- (0, \ymax) node[above]{Im$(z)$};;	
					
					% tekening
					\draw[->, red]  (0, 0) -- node[right]{$2+4i$}    (2,  4);
					\draw[->, red]  (2, 4) -- node[right]{$-(6-7i)$} (-4, 11);
					\draw[->, blue] (0, 0) -- node[left]{$-4-3i$}   (-4, 11);	
					\end{tikzpicture}
				\end{image}
			\end{oplossing}
		\end{question}
		
		
		
		\begin{question} $(2+i)^2 = \answer[onlineshowanswerbutton]{3+4i}$
			\begin{hint}
				$(a+b)^2 = a^2 + 2ab + b^2$
			\end{hint}
			\begin{oplossing}
				$$
				(2+i)^2  = 4 + 4i + i^2 = 3 + 4i
				$$
				\begin{image}[0.5\textwidth]
					\begin{tikzpicture}[scale=1.3]
					% grid setup
					
					\def\xmin{0}
					\def\xmax{4}
					
					\def\ymin{0}
					\def\ymax{5}				
					
					% grid
					\draw[step = 1.0, black, thin, dashed] (\xmin, \ymin) grid (\xmax, \ymax);
					\draw[->, thick] (\xmin, 0) -- (\xmax, 0) node[right]{Re$(z)$};
					\draw[->, thick] (0, \ymin) -- (0, \ymax) node[above]{Im$(z)$};;	
					
					% tekening
					\draw[->, blue] (0, 0) -- (3, 4) node[above]{$3+4i$};	
					\end{tikzpicture}
				\end{image}
			\end{oplossing}
		\end{question}
		
		\begin{question} $(2+i)+\overline{(3+2i)} = \answer[onlineshowanswerbutton]{5-i}$
			\begin{oplossing}
				Merk op dat $\overline{3+2i} = 3-2i$. Dus moeten we de som $(2+i)+ (3-2i)$ bepalen. Je kan ofwel reëel en imaginair deel van beide complexe getallen bij elkaar optellen om het antwoord te vinden, ofwel in het complexe vlak de vectorsom maken.
				\begin{image}[0.5\textwidth]
					\begin{tikzpicture}[scale=1.2]
					% grid setup
					
					\def\xmin{0}
					\def\xmax{5}
					
					\def\ymin{-2}
					\def\ymax{2}				
					
					% grid
					\draw[step = 1.0, black, thin, dashed] (\xmin, \ymin) grid (\xmax, \ymax);
					\draw[->, thick] (\xmin, 0) -- (\xmax, 0) node[right]{Re$(z)$};
					\draw[->, thick] (0, \ymin) -- (0, \ymax) node[above]{Im$(z)$};;	
					
					% tekening
					\draw[->, red]  (0, 0) -- node[left]{$2+i$}   (2,  1);	
					\draw[->, red]  (2, 1) -- node[above]{$3-2i$} (5, -1);			
					\draw[->, blue] (0, 0) -- node[below]{$3+4i$} (5, -1);	
					\end{tikzpicture}
				\end{image}
			\end{oplossing}
		\end{question}
		
		\begin{question} $(5-6i)(5+6i) = \answer[onlineshowanswerbutton]{61}$
			\begin{oplossing}
				Je kan de rekenregel $z\cdot \overline{z} = |z|^2$ gebruiken. Dan is $(5-6i)(5+6i) = 25 + 36 = 61$.
			\end{oplossing}
		\end{question}
		
				
		\begin{question} $|3-2i + \overline {4-2i}| = \answer[onlineshowanswerbutton]{7}$
			\begin{oplossing}
				Merk eerst op dat $\overline{4-2i} = 4 + 2i$. Je kan ofwel reëel en imaginair deel van beide complexe getallen bij elkaar optellen om de som van twee complexe getallen te vinden, ofwel in het complexe vlak de vectorsom maken.
				\begin{image}[0.5\textwidth]
					\begin{tikzpicture}[scale=0.7]
					% grid setup
					
					\def\xmin{0}
					\def\xmax{8}
					
					\def\ymin{-3}
					\def\ymax{3}				
					
					% grid
					\draw[step = 1.0, black, thin, dashed] (\xmin, \ymin) grid (\xmax, \ymax);
					\draw[->, thick] (\xmin, 0) -- (\xmax, 0) node[right]{Re$(z)$};
					\draw[->, thick] (0, \ymin) -- (0, \ymax) node[above]{Im$(z)$};;	
					
					% tekening
					\draw[->, red]  (0, 0)  -- node[left]{$3-2i$}  (3,  -2);	
					\draw[->, red]  (3, -2) -- node[right]{$4+2i$}  (7, 0);			
					\draw[->, blue] (0, 0)  -- node[above]{$7+0i$}  (7, 0);	
					\end{tikzpicture}
				\end{image}
				% Aangezien het resultaat een reëel getal is, is de norm van dit getal hetzelfde als de absolute waarde van dit getal, en $|7|= 7$.
			\end{oplossing}
		\end{question}
		
		\begin{question} $|3+4i+4+3i|     = \answer[onlineshowanswerbutton]{7\sqrt 2}$
			\begin{oplossing}
				$|3+4i+4+3i| = |7 + 7i| = \sqrt{49 + 49} = 7 \sqrt{2}$.
			\end{oplossing}
		\end{question}
		
		\begin{question} $|3+4i|+|4+3i| = \answer[onlineshowanswerbutton]{10}$
			\begin{oplossing}
				$|3 + 4i| = \sqrt{3^2 + 4^2} = 5$\\
				$|4+3i| = \sqrt{4^2 + 3^2} = 5$
			\end{oplossing}
		\end{question}
		
		\begin{question} $\left| \frac{(3+4i)(-1+2i)}{(-1-i)(3-i)} \right| =  \answer[onlineshowanswerbutton]{\frac52}$
		\end{question}
		
	\end{exercise}
	
\renewcommand{\shortanswerscols}{1}	
	\begin{exercise}
		%\id{201206wis08*}
		\begin{statement}
		Een complex getal $z$ kunnen we schrijven als $z=a+bi$ met $a$ en $b$ reële getallen en $i^2=-1$.Hoeveel complexe getallen in deze lijst
		\[ 
		\frac{(1+i)^4}{4}, \quad \frac{(1-i)^4}{4}, \quad\frac{(1+i)^2}{2},\quad \frac{(1-i)^2}{2}, \quad i^2 
		\] 
	zijn gelijk zijn aan -1? 
	\end{statement}
 \quad $\answer{3}$
		
		\begin{oplossing}
			Er is al minstens één van deze complexe getallen gelijk aan -1: de laatste, $i^2$, per definitie van de imaginaire eenheid. 
			
			Je kan best te werk gaan door de derde en vierde uitdrukking te bepalen: de eerste twee complexe getallen zijn hun kwadraten. Hiervoor maken we de volgende berekeningen:
			\begin{align*}
			(1+i)^2 &= 1 + 2i + i^2 = 2i  \\
			(1-i)^2 &= 1 - 2i + i^2 = -2i  \\	
			\end{align*}
			en dus vinden we dat
			\begin{align*}
			\frac{(1+i)^2}{2} &= i  \\
			\frac{(1-i)^2}{2} &= -i  \\	
			\end{align*}
			Het derde en vierde complexe getal zijn dus niet gelijk aan -1. Hun kwadraten zijn echter wel gelijk aan -1: drie van de complexe getallen zijn dus gelijk aan -1. 
		\end{oplossing}	
	\end{exercise}

\renewcommand{\shortanswerscols}{3}	
	\begin{exercise}
		\begin{statement}
		Schets in het complexe vlak de gebieden omschreven door volgende vergelijkingen:
		\end{statement}
		\begin{question} 
			$|z| < 1$
			\begin{oplossing} $|z|$ is de afstand van $z$ tot de oorsprong. Alle complexe getallen waarvan de afstand tot de oorsprong kleiner is dan 1 liggen \textit{binnen} een cirkel met straal 1, de rand zelf wordt dus niet meegerekend:
				
				\begin{image}[0.4\textwidth]
					\begin{tikzpicture}[scale=1.4, baseline=(current bounding box.north)]
					
					\draw[->] (-1.2,0) -- (1.2,0) node[right] {Re$(z)$};
					\draw[->] (0,-1.2) -- (0,1.2) node[above] {Im$(z)$};
					
					\draw[white, pattern=north west lines, pattern color=blue] (0,0) circle (1cm);
					\end{tikzpicture}
				\end{image}           
			\end{oplossing}
		\end{question}
		
%		\begin{question} 
%			$|z| \leq 1$
%			\begin{oplossing} $|z|$ is de afstand van $z$ tot de oorsprong. Alle complexe getallen waarvan de afstand tot de oorsprong kleiner is dan of gelijk is aan 1 liggen \textit{binnen} een cirkel met straal 1 waarbij de rand zelf wordt meegerekend:
%				
%				\begin{image}[0.4\textwidth]
%					\begin{tikzpicture}[scale=1.4, baseline=(current bounding box.north)]
%					
%					\draw[->] (-1.2,0) -- (1.2,0) node[right] {Re$(z)$};
%					\draw[->] (0,-1.2) -- (0,1.2) node[above] {Im$(z)$};
%					
%					\draw[blue, thick, pattern=north west lines, pattern color=blue] (0,0) circle (1cm);
%					\end{tikzpicture}
%				\end{image}           
%			\end{oplossing}
%		\end{question}
		
		\begin{question} 
			$|z| = 1$
			\begin{oplossing} $|z|$ is de afstand van $z$ tot de oorsprong. Alle complexe getallen waarvan de afstand tot de oorsprong gelijk is aan 1 liggen dus op een cirkel met straal 1:
				
				\begin{image}[0.4\textwidth]
					\begin{tikzpicture}[scale=1.4, baseline=(current bounding box.north)]
					
					\draw[dashed, thin] (-1.2,-1.2) grid (1.2, 1.2);
					\draw[->] (-1.2,0) -- (1.2,0) node[right] {Re$(z)$};
					\draw[->] (0,-1.2) -- (0,1.2) node[above] {Im$(z)$};
					
					\draw[blue, thick] (0,0) circle (1cm);
					\end{tikzpicture}
				\end{image}           
			\end{oplossing}
		\end{question}
		
%		\begin{question} 
%			$|z-i| < 1$
%			\begin{oplossing} 
%				Alle complexe getallen waarvoor $|z-i| < 1$ zijn alle complexe getallen $z$ waarvoor de afstand tot $i$ kleiner is dan 1:
%				
%				\begin{image}[0.4\textwidth]
%					\begin{tikzpicture}[scale=1.4, baseline=(current bounding box.north)]
%					
%					\draw[dashed, thin] (-1.2,-1.2) grid (1.2, 2.2);
%					\draw[->] (-1.2, 0) -- (1.2, 0) node[right] {Re$(z)$};
%					\draw[->] (0, -1.2) -- (0, 2.2) node[above] {Im$(z)$};
%					
%					\draw[white, pattern=north west lines, pattern color=blue] (0,1) circle (1cm);	
%					\draw (0,1) node[name=Zi,circle, fill=black, radius=1pt,scale=0.5] {} node [fill=white,xshift=-2pt,left] {$i$};   	
%					\end{tikzpicture}
%				\end{image}  
%			\end{oplossing}
%		\end{question}
		
		\begin{question} 
			$|z-i| \geq 1$
			\begin{oplossing} Alle complexe getallen waarvoor $|z-i| \geq 1$ zijn alle complexe getallen $z$ waarvoor de afstand tot $i$ groter of gelijk is aan 1: 
				\begin{image}[0.4\textwidth]
					\begin{tikzpicture}[scale=1.2, baseline=(current bounding box.north)]
					
					\draw[->] (-2.2, 0) -- (2.2, 0) node[right] {Re$(z)$};
					\draw[->] (0, -1.2) -- (0, 2.2) node[above] {Im$(z)$};
					\draw[pattern=north west lines, pattern color=blue] (-2.2, -1.2) rectangle (2.2, 2.2);
					
					\fill[white] (0,1) circle (1cm);
					\draw[blue, thick] (0,1) circle (1cm);		    
					\draw (0,1) node[name=Zi,circle, fill=black, radius=1pt,scale=0.5] {} node [fill=white,xshift=-2pt,left] {$i$};
					\end{tikzpicture}
				\end{image}            
			\end{oplossing}
		\end{question}
		
		
		
		
		\begin{question} 
			$-1 \leq \text{Im}(z) \leq 2$
			\begin{oplossing} Deze ongelijkheden bepalen een \textit{horizontale} strook, aangezien het imaginaire deel van het complex getal in een interval moet liggen. De randen van de strook behoren tot het geschetste domein.
				
				\begin{image}[0.5\textwidth]
					\begin{tikzpicture}[scale=0.8, baseline=(current bounding box.north)]
					\draw[dashed] (-5, -2) grid (5, 3);
					
					\draw[->, thick] (-5, 0) -- (5, 0) node[right] {Re$(z)$};
					\draw[->, thick] (0, -2) -- (0, 3) node[above] {Im$(z)$};
					
					\draw[white, pattern=north west lines, pattern color=blue] (-5,2) rectangle (5, -1);
					
					% randen toevoegen voor duidelijkheid 
					\draw[blue, thick] (-5, -1) -- (5, -1);
					\draw[blue, thick] (-5, 2) -- (5, 2);
					\end{tikzpicture}
				\end{image}
			\end{oplossing}            
		\end{question}  
		
		
		
		
		\begin{question} 
			$\text{Re}(z)> \text{Im}(z)>0$
			\begin{oplossing} 
				Als $z=a+bi$, dan betekent $\text{Re}(z) > \text{Im}(z) > 0$ dat $a > b > 0$. Dit is het gebied onder de rechte $y = x$, (daar is $a > b$), en boven de $x$-as (daar is $b > 0$):
				
				\begin{image}[0.4\textwidth]
					\begin{tikzpicture}[scale=1.1, baseline=(current bounding box.north)]
					\draw[dashed] (-0.5, -0.5) grid (2.9, 2.9);
					
					\draw[->, thick] (-0.5, 0) -- (2.9, 0) node[right] {Re$(z)$};
					\draw[->, thick] (0, -0.5) -- (0, 2.9) node[above] {Im$(z)$};
					
					\draw[white, pattern=north west lines, pattern color=blue] (0,0.02) -- (2.9, 2.9) -- (2.9, 0.02) -- cycle;
					\end{tikzpicture}
				\end{image}       
			\end{oplossing}
		\end{question}
		
		
	\end{exercise}

\renewcommand{\shortanswerscols}{2}	
	\begin{exercise}
		\begin{statement}
		Voer volgende delingen uit door de opgave te schrijven in de vorm $a+bi$.
		\end{statement}
		\begin{question}$ \frac{7}{i} = \answer[onlineshowanswerbutton]{-7i}$
			\begin{oplossing}
				We vermenigvuldigen teller en noemer met $\overline{i} = -i$. In de noemer krijgen we dan $i\cdot (-i) = 1$. Het resultaat is $-7i$. Voor oefeningen kan het handig zijn om te onthouden dat \textit{delen door $i$ hetzelfde is als vermenigvuldigen met $-i$}. % IDEE: dit laatste misschien ook eens bespreken en duidelijk maken in theorie blokje? 
			\end{oplossing}
		\end{question}
		
		\begin{question}$ \frac{1}{5+2i} = \answer[onlineshowanswerbutton]{\frac{5}{29}-\frac{2}{29}i}$
			\begin{oplossing}
				We vermenigvuldigen teller en noemer met $\overline{5+2i} = 5-2i$. In de noemer krijgen we dan $(5+2i)(5-2i) = 29$.
				$$
				\frac{1}{5+2i} = \frac{1}{5+2i} \frac{5-2i}{5-2i} = \frac{5-2i}{25}=\frac{5}{29}-\frac{2}{29}i
				$$
			\end{oplossing}
		\end{question}
		
		\begin{question} $ \frac{1+i}{2+3i}	= \answer[onlineshowanswerbutton]{\frac{5}{13}-\frac{1}{13}i}$
			\begin{oplossing}
				We vermenigvuldigen teller en noemer met $\overline{2+3i} = 2-3i$. In de noemer krijgen we dan $(2+3i)(2-3i) = 13$.
				$$
				\frac{1+i}{2+3i} = \frac{1+i}{2+3i} \cdot \frac{2-3i}{2-3i} = \frac{5-i}{13}=\frac{5}{13}-\frac{1}{13}i
				$$
			\end{oplossing}
		\end{question}
		
		
		\begin{question} $ \frac{1+2i}{3-4i} + \frac{2-i}{5i} = \answer[onlineshowanswerbutton]{-\frac{2}{5}}$
			\begin{oplossing}
				We berekenen de beide termen apart, en tellen de resultaten vervolgens bij elkaar op.
				
				Voor de eerste term: we vermenigvuldigen teller en noemer met $\overline{3-4i} = 3+4i$. In de noemer krijgen we dan $(3-4i)(3+4i) = 25$. De eerste term kunnen we dus vereenvoudigen tot
				$$
				\frac{1+2i}{3-4i} = \frac{1+2i}{3-4i} \cdot \frac{3+4i}{3+4i} = \frac{-5+10i}{25} = \frac{-1+2i}{5}
				$$
				
				Voor de tweede term: we vermenigvuldigen teller en noemer met $\overline{5i} = -5i$. In de noemer krijgen we dan $(5i)\cdot(-5i) = 25$. De tweede term kunnen we dus vereenvoudigen tot
				$$
				\frac{2-i}{5i} = \frac{2-i}{5i} \cdot \frac{-5i}{-5i} = \frac{-1-2i}{5}
				$$
				Een andere mogelijkheid is om te gebruiken dat delen door $i$ hetzelfde is als vermenigvuldigen met $-i$, zodat we meteen bekomen dat
				$$
				\frac{2-i}{5i} = \frac{(2-i)\cdot (-i)}{5} = \frac{-1-2i}{5}
				$$
				
				Wanneer we deze twee termen bij elkaar optellen, valt het imaginaire deel weg. Het resultaat is $\frac{-2}{5}$. 
			\end{oplossing}
		\end{question}
	\end{exercise}
	
	
	\begin{exercise}
		\begin{statement}
		Bepaal alle oplossingen van volgende veeltermvergelijkingen.
		\end{statement}
		
		
		\begin{question}
			$ x^2 + x = 0$
			\begin{uitkomst} $x_1 = 0$ en $x_2 = -1$
				\end{uitkomst}
			\begin{oplossing}
				De oplossingen zijn $x_1 = 0$ en $x_2 = -1$.
				
				We kunnen een $x$ afzonderen in deze vergelijking en vinden dan de equivalente vergelijking $x(x+1) = 0$. Deze heeft als oplossingen $x=0$ en $x=-1$.
			\end{oplossing}
		\end{question}
		
		\begin{question}
			$ x^3 + 16x = 0$
			\begin{uitkomst} $x_1 = 0$, $x_2 = 4i$ en $x_3 = -4i$
				\end{uitkomst}
			\begin{oplossing}
				De oplossingen zijn $x_1 = 0$, $x_2 = 4i$ en $x_3 = -4i$.
				
				We kunnen een $x$ afzonderen, en vinden dan de equivalente vergelijking $x(x^2+16)=0$. Deze heeft als oplossingen ofwel $x=0$ ofwel $x^2 + 16 = 0$. De laatste vergelijking heeft als oplossingen $x=4i$ en $x=-4i$.
			\end{oplossing}
		\end{question}
		
		
		\begin{question}
			$ x^2 + 5x + 2 = 0$
			\begin{uitkomst} $x_1 = \frac{-5 + \sqrt{17}}{2}$ en $x_2 = \frac{-5 - \sqrt{17}}{2}$
			\end{uitkomst}
			\begin{oplossing}
				De oplossingen zijn $x_1 = \frac{-5 + \sqrt{17}}{2}$ en $x_2 = \frac{-5 - \sqrt{17}}{2}$.
				
				De vergelijking is van de vorm $ax^2 + bx + c = 0$, met $a=1$, $b=5$ en $c=2$. Dan is $D = b^2 - 4ac = 25-8 = 17$. Omdat $D > 0$, zijn er twee reële oplossingen:
				$$
				x_1 = \frac{-b + \sqrt{D}}{2a} = \frac{-5 + \sqrt{17}}{2}, \ \ x_2 = \frac{-b - \sqrt{D}}{2a} = \frac{-5 - \sqrt{17}}{2} \, .
				$$
			\end{oplossing}
		\end{question}
		
		\begin{question}
			$ x^2 + 4x + 5 = 0$
			\begin{uitkomst} $x_1 = -2 - i$ en $x_2 = -2 + i$
			\end{uitkomst}
			\begin{oplossing}
				De oplossingen zijn $x_1 = -2 - i$ en $x_2 = -2 + i$.
				
				De vergelijking is van de vorm $ax^2 + bx + c = 0$, met $a=1$, $b=4$ en $c=5$. Dan is $D = b^2 - 4ac = 16 - 20 = -4$. Omdat $D < 0$, zijn er twee complexe oplossingen. De oplossingen zijn
				$$
				x_1 = \frac{-b + i\sqrt{-D}}{2a} = \frac{-4 + 2i}{2} = -2 + i, \ \ x_2 = \frac{-b - i\sqrt{-D}}{2a} = \frac{-4 - 2i}{2} = -2 - i \, .
				$$
			\end{oplossing}
		\end{question}
		
		
		
		
%		\begin{question}
%			$ 9x^2 - 14 = 4x$
%			\begin{uitkomst} $x_1 = \frac{2 + \sqrt{130}}{9}$ en $x_2 = \frac{2 - \sqrt{130}}{9}$
%			\end{uitkomst}
%			\begin{oplossing}
%				De oplossingen zijn $x_1 = \frac{2 + \sqrt{130}}{9}$ en $x_2 = \frac{2 - \sqrt{130}}{9}$.
%				
%				De vergelijking is van de vorm $ax^2 + bx + c = 0$, met $a=9$, $b=-4$ en $c=-14$. Dan is $D = b^2 - 4ac = 16 + 504 = 520$. Omdat $D > 0$, zijn er twee verschillende reële oplossingen. Merk op dat $\sqrt{D} = \sqrt{520} = \sqrt{4 \cdot 130} = 2 \sqrt{130}$. De oplossingen zijn
%				$$
%				x_1 = \frac{-b + \sqrt{D}}{2a} = \frac{4 + 2 \sqrt{130}}{18} = \frac{2 + \sqrt{130}}{9}, \ \ x_2 = \frac{-b - \sqrt{D}}{2a} = \frac{4 - 2 \sqrt{130}}{18} = \frac{2 - \sqrt{130}}{9}\, .
%				$$
%			\end{oplossing}
%		\end{question}
		
		\begin{question}
			$ x = 2x^2 + 5$
			\begin{uitkomst} $x_1 = \frac{1}{4} + \frac{\sqrt{39}}{4} i$ en $x_2 = \frac{1}{4} - \frac{\sqrt{39}}{4} i$
			\end{uitkomst}
			\begin{oplossing}
				De oplossingen zijn $x_1 = \frac{1}{4} + \frac{\sqrt{39}}{4} i$ en $x_2 = \frac{1}{4} - \frac{\sqrt{39}}{4} i$.
				
				De vergelijking is van de vorm $ax^2 + bx + c = 0$, met $a=2$, $b=-1$ en $c=5$. Dan is $D = b^2 - 4ac = 1 - 40 = -39$. Omdat $D < 0$, zijn er twee complexe oplossingen. De oplossingen zijn
				$$
				x_1 = \frac{-b + i\sqrt{-D}}{2a} = \frac{1 + \sqrt{39}i}{4} = \frac{1}{4} + \frac{\sqrt{39}}{4} i, \ \ x_2 = \frac{-b - i\sqrt{-D}i}{2a} = \frac{1 - \sqrt{39}i}{4} = \frac{1}{4} - \frac{\sqrt{39}}{4} i \, .
				$$
			\end{oplossing}
		\end{question}
		
	\end{exercise}
	

\begin{exercise}
	\begin{statement}
	Schrijf onderstaande getallen in goniometrische vorm $r(\cos\theta+i\sin \theta)$. Zorg er steeds voor dat $\theta$ tussen 0 en $2\pi$ ligt, dus niet $-\frac{\pi}{2}$ antwoorden maar $\frac{3\pi}{2}$.
	\end{statement}
%	\begin{question}  $z=1$ 
%		\begin{uitkomst} $z =  \cos 0 + i\sin 0$
%			\end{uitkomst}
%		\begin{oplossing} 
%			Omdat $z = 1 = 1 + 0i$ en dus $a = 0$ en $b = 1$, vinden we $|z| = r = 1$. In het complexe vlak zien we dadelijk dat $\theta = 0$.
%			%Bovendien voldoet $\theta$ aan $\cos \theta = 1$ en $\sin \theta = 0$ zodat $\theta = 0$. 
%			De goniometrische schrijfwijze van $z$ is dan $z = 1 =  \cos 0 + i\sin 0$.
%			\begin{image}[0.5\textwidth]
%				\begin{tikzpicture}[scale = 1, trig format = rad]
%				%	grid setup
%				
%				\def\xmin{-2}
%				\def\xmax{2}
%				
%				\def\ymin{-2}
%				\def\ymax{2}				
%				
%				%	grid
%				\draw[step = 1, black, thin, dashed] (\xmin, \ymin) grid (\xmax, \ymax);
%				\draw[->] (\xmin, 0) -- (\xmax, 0) node[right]{Re$(z)$};
%				\draw[->] (0, \ymin) -- (0, \ymax) node[above]{Im$(z)$};;	
%				
%				%	vector
%				\draw[->, blue, very thick] (0, 0) -- node[below]{$1$}(1, 0);
%				
%				%	boogje - niet handig nu
%				\def\r{0.2}
%				%\draw[->, blue] (\r, 0) arc (0:0:\r) node[right]{$0$};
%				\end{tikzpicture}
%			\end{image}
%		\end{oplossing}
%	\end{question}
	
	\begin{question} 
		 $z=-1$ 
		\begin{uitkomst} $z  = \cos \pi + i\sin \pi$
		\end{uitkomst}
		\begin{oplossing} 
			$z=-1 = -1 + 0i$ en dus $a = -1$ en $b = 0$, waaruit we halen dat $|z| = r = 1$. In het complexe vlak zien we dadelijk dat $\theta = \pi$. 
			%Het argument van $z$ moet voldoen aan $\cos \theta = -1$ en $\sin \theta = 0$ zodat $\theta = \pi$. 
			De goniometrische schrijfwijze van $z$ is dan $z  = \cos \pi + i\sin \pi$.
			\begin{image}[0.5\textwidth]
				\begin{tikzpicture}[scale = 1]
				%	grid setup
				
				\def\xmin{-2}
				\def\xmax{2}
				
				\def\ymin{-2}
				\def\ymax{2}				
				
				%	grid
				\draw[step = 1, black, thin, dashed] (\xmin, \ymin) grid (\xmax, \ymax);
				\draw[->] (\xmin, 0) -- (\xmax, 0) node[right]{Re$(z)$};
				\draw[->] (0, \ymin) -- (0, \ymax) node[above]{Im$(z)$};;	
				
				%	vector
				\draw[->, blue, very thick] (0, 0) -- node[below]{$-1$}(-1, 0);
				
				%	boogje - niet handig nu
				\def\r{0.2}
				\draw[->, blue] (\r, 0) arc (0:180:\r) node[midway, above]{$\pi$};
				\end{tikzpicture}
			\end{image}
			Merk op dat de modules positief moet zijn, en met $r=-1$ en $\theta = 0$ is $-(\cos 0 + i\sin0)$ dus \textit{geen} goniometrische schrijfwijze van het getal $-1$.
		\end{oplossing}
	\end{question}
	
%	\begin{question} 
%		 $z = i$ 
%		\begin{uitkomst} $z = \cos \frac{\pi}{2} +i\sin \frac{\pi}{2}$
%		\end{uitkomst}
%		\begin{oplossing} 
%			Omdat $z = i = 0 + i$, en dus $a = 0$, $b = 1$, zodat $|z| = r = 1$. In het complexe vlak zien we dadelijk dat $\theta=\frac{\pi}{2}$.
%			% en $\theta$ voldoet aan $\cos\theta = 0$ en $\sin \theta = 1$ (of je kan het ook afleiden uit onderstaande schets). 
%			De goniometrische schrijfwijze van $z$ is dus $z = \cos \frac{\pi}{2} +i\sin \frac{\pi}{2}$.
%			\begin{image}[0.5\textwidth]
%				\begin{tikzpicture}[scale = 1]
%				%	grid setup
%				
%				\def\xmin{-2}
%				\def\xmax{2}
%				
%				\def\ymin{-2}
%				\def\ymax{2}				
%				
%				%	grid
%				\draw[step = 1, black, thin, dashed] (\xmin, \ymin) grid (\xmax, \ymax);
%				\draw[->] (\xmin, 0) -- (\xmax, 0) node[right]{Re$(z)$};
%				\draw[->] (0, \ymin) -- (0, \ymax) node[above]{Im$(z)$};;	
%				
%				%	vector
%				\draw[->, blue, very thick] (0, 0) -- node[left]{$i$}(0, 1);
%				
%				%	boogje
%				\def\r{0.2}
%				\draw[->, blue] (\r, 0) arc (0:90:\r) node[midway, right]{$\pi/2$};
%				\end{tikzpicture}
%			\end{image}
%			
%		\end{oplossing}
%	\end{question}
	
	\begin{question} 
		 $z = -i$ 
		\begin{uitkomst} $z  = \cos\left(\frac{3\pi}{2}\right) +i \sin\left(\frac{3\pi}{2}\right)$
		\end{uitkomst}
		\begin{oplossing}
			Voor $z = -i$ is $a = 0$ en $b = -1$, dus $|z| = r = 1$. In het complexe vlak zien we dadelijk dat $\theta=\frac{3\pi}{2}$.
			% en $\theta$ voldoet aan $\cos \theta = 0$ en $\sin \theta = -1$, dus $\theta = - \frac{\pi}{2} = \frac{3\pi}{2}$. 
			De goniometrische schrijfwijze van $z$ is dus $z  = \cos\left(\frac{3\pi}{2}\right) +i \sin\left(\frac{3\pi}{2}\right)$.
			\begin{image}[0.5\textwidth]
				\begin{tikzpicture}[scale = 1]
				%	grid setup
				
				\def\xmin{-2}
				\def\xmax{2}
				
				\def\ymin{-2}
				\def\ymax{2}				
				
				%	grid
				\draw[step = 1, black, thin, dashed] (\xmin, \ymin) grid (\xmax, \ymax);
				\draw[->] (\xmin, 0) -- (\xmax, 0) node[right]{Re$(z)$};
				\draw[->] (0, \ymin) -- (0, \ymax) node[above]{Im$(z)$};;	
				
				%	vector
				\draw[->, blue, very thick] (0, 0) -- node[right]{$-i$}(0, -1);
				
				%	boogje
				\def\r{0.2}
				\draw[->, blue] (\r, 0) arc (0:270:\r) node[midway, above]{$\frac{3\pi}{2}$};
				\end{tikzpicture}
			\end{image}
		\end{oplossing}
	\end{question}
	
	\begin{question} 
		$z = \sqrt{3} + i$ 
		\begin{uitkomst} $
			z = 2\left(\cos \left(\frac{\pi}{6}\right) + i\sin\left(\frac{\pi}{6}\right)\right) \, .
			$
		\end{uitkomst}
		\begin{oplossing} 
			Voor $z = \sqrt{3} + i$ geldt $a = \sqrt{3}$ en $b = 1$, dus $|z| = r = \sqrt{a^2 + b^2} = 2$. Het argument van $z$ voldoet aan $\cos \theta = \frac{\sqrt{3}}{2}$ en $\sin \theta = \frac{1}{2}$. Dit zijn de goniometrische getallen van de standaardhoek $\pi/6$. De goniometrische schrijfwijze van $z$ is dus 
			$$
			z = 2\left(\cos \left(\frac{\pi}{6}\right) + i\sin\left(\frac{\pi}{6}\right)\right) \, .
			$$
			\begin{image}[0.5\textwidth]
				\begin{tikzpicture}[scale = 1.5]
				%	grid setup
				
				\def\xmin{-2.1}
				\def\xmax{2.1}
				
				\def\ymin{-2.1}
				\def\ymax{2.1}				
				
				%	grid
				\draw[step = 1, black, thin, dashed] (\xmin, \ymin) grid (\xmax, \ymax);
				\draw[->] (\xmin, 0) -- (\xmax, 0) node[right]{Re$(z)$};
				\draw[->] (0, \ymin) -- (0, \ymax) node[above]{Im$(z)$};;	
				
				%	vector
				\draw[->, blue, very thick] (0, 0) -- ({sqrt(3)}, 1)node[below]{$\sqrt{3} + i$};
				
				%	boogje
				\def\r{0.5}
				\draw[->, blue] (\r, 0) arc (0:30:\r) node[midway, right]{$\frac{\pi}{6}$};
				\end{tikzpicture}
			\end{image}
		\end{oplossing}
	\end{question}
	
	\begin{question} 
		 $z = -\sqrt{3} + i$ 
		\begin{uitkomst} $
			z = 2\left(\cos \left(\frac{5\pi}{6}\right) + i\sin\left(\frac{5\pi}{6}\right)\right) \, .
			$
		\end{uitkomst}
		\begin{oplossing} 
			Voor $z = -\sqrt{3} + i$ geldt $a = -\sqrt{3}$ en $b = 1$. Dus geldt $|z| = r = 2$. Voor het argument van $z$ geldt dat $\cos \theta = -\frac{\sqrt{3}}{2}$ en $\sin \theta = \frac{1}{2}$. Dan is $\theta =  \frac{5\pi}{6} =\pi -  \frac{\pi}{6}$: dit kan je afleiden uit de goniometrische getallen voor $\pi/6$ en de rekenregels voor supplementaire hoeken te gebruiken.
			
			De goniometrische schrijfwijze van $z$ is dus 
			$$
			z = 2\left(\cos \left(\frac{5\pi}{6}\right) + i\sin\left(\frac{5\pi}{6}\right)\right) \, .
			$$
			
			\begin{image}[0.5\textwidth]
				\begin{tikzpicture}[scale = 1.5]
				%	grid setup
				
				\def\xmin{-2.1}
				\def\xmax{2.1}
				
				\def\ymin{-2.1}
				\def\ymax{2.1}				
				
				%	grid
				\draw[step = 1, black, thin, dashed] (\xmin, \ymin) grid (\xmax, \ymax);
				\draw[->] (\xmin, 0) -- (\xmax, 0) node[right]{Re$(z)$};
				\draw[->] (0, \ymin) -- (0, \ymax) node[above]{Im$(z)$};;	
				
				%	vector
				\draw[->, blue, very thick] (0, 0) -- ({-sqrt(3)}, 1)node[below]{$-\sqrt{3} + i$};
				
				%	boogje
				\def\r{0.5}
				\draw[->, blue] (\r, 0) arc (0:150:\r) node[midway, above]{$5\pi/6$};
				\end{tikzpicture}
			\end{image}
		\end{oplossing}
	\end{question}
	
	
\end{exercise}

\begin{exercise}
	\begin{statement}
	Bepaal de modulus $|z|$ en het argument $\theta$ van volgende complexe getallen $z$. Geef de hoek $\theta$ tussen 0 en $2\pi$ radialen.
	\end{statement}
	\begin{question} 
		 $z=-7i$
		\begin{uitkomst}  $|z|=7$ en $\theta=3\pi/2$.
			\end{uitkomst}
		\begin{oplossing}
			$z = a + bi = 0 - 7i$, dus is $|z| = \sqrt{a^2 + b^2} = 7$. In het complexe vlak zien we dadelijk dat $\theta = 3 \pi/2$.  Je kan dit ook berekenen:  $\cos\theta=0/7=0$ en $\sin\theta=-7/7=-1$, en dus is $\theta = 3 \pi/2$ (op een veelvoud van $2\pi$ na). 
			
			\begin{image}[0.5\textwidth]
				\begin{tikzpicture}[scale = 0.5]
				%	grid setup
				
				\def\xmin{-7}
				\def\xmax{7}
				
				\def\ymin{-8}
				\def\ymax{2}				
				
				%	grid
				\draw[step = 1.0, gray, thin, dashed] (\xmin, \ymin) grid (\xmax, \ymax);
				\draw[->] (\xmin, 0) -- (\xmax, 0) node[right]{Re$(z)$};
				\draw[->] (0, \ymin) -- (0, \ymax) node[above]{Im$(z)$};;	
				
				%	vector
				\draw[->, blue, very thick] (0, 0) -- node[right]{$-7i$}  (0, -7);
				
				%	boogje
				\def\r{1}
				\draw[->, blue] (\r, 0) arc (0:270:\r) node[midway, left]{$3\pi/2$};
				\end{tikzpicture}
			\end{image}
		\end{oplossing}
	\end{question}
	
	
	\begin{question} 
		 $z=(7i)^{-1}$
		\begin{uitkomst}  $|z|=1/7$ en $\theta=3\pi/2$.
		\end{uitkomst}
		\begin{hint} Schrijf $z$ eerst in cartesische vorm $a+bi$. \end{hint}
		\begin{oplossing}
			Eerst herschrijven we het complex getal $z$ in cartesische vorm: $z = \frac{1}{7i} = - \frac17 i$ (vermenigvuldig teller en noemer met $i$). 
			\\
			Een tip voor berekeningen met complexe getallen: \textit{delen door $i$ is hetzelfde als vermenigvuldigen met $-i$}.
			
			Omdat $z = a + bi = 0 - \frac17i$ is $|z| = \sqrt{a^2 + b^2} = \frac17$, en $\theta = 3 \pi/2$ (op een veelvoud van $2\pi$ na), zoals we kunnen afleiden uit een schets (de ijk op beide assen is nu 0.1 in plaats van 1):
			
			\begin{image}[0.5\textwidth]
				\begin{tikzpicture}[scale = 5]
				%	grid setup
				
				\def\xmin{-0.4}
				\def\xmax{0.4}
				
				\def\ymin{-0.4}
				\def\ymax{0.4}				
				
				%	grid
				\draw[step = 0.1, gray, thin, dashed] (\xmin, \ymin) grid (\xmax, \ymax);
				\draw[->] (\xmin, 0) -- (\xmax, 0) node[right]{Re$(z)$};
				\draw[->] (0, \ymin) -- (0, \ymax) node[above]{Im$(z)$};;	
				
				%	vector
				\draw[->, blue, very thick] (0, 0) -- (0, -1/7) node[below right, fill = white]{$-\frac17i$};
				
				%	boogje
				\def\r{0.05}
				\draw[->, blue] (\r, 0) arc (0:270:\r) node[midway, above left, fill = white]{$3\pi/2$};
				\end{tikzpicture}
			\end{image}
		\end{oplossing}
	\end{question}
	
	
	\begin{question} 
		 $z = \overline{1+i}$
		\begin{uitkomst} $|z| = \sqrt{2}$ en $\theta = 7\pi/4$.
		\end{uitkomst}
		\begin{hint} Schrijf $z$ eerst in cartesische vorm $a+bi$. \end{hint}
		\begin{oplossing}
			Eerst herschrijven we het complex getal $z$: $z = \overline{1+i} = 1 - i$.
			
			Omdat $z = a + bi = 1 - i$, is $|z| = \sqrt{a^2 + b^2} = \sqrt{2}$, en $\theta =- \pi / 4$: dit kan je afleiden uit de tekening of bepaal je uit de vergelijkingen 
			%$a = r\cos \theta$ en $b = r\sin\theta$. Deze kan je omvormen naar
			\begin{align*}
			\cos \theta &= \frac{a}{r} = \frac{1}{\sqrt{2}} \\
			\sin \theta &= \frac{b}{r} = -\frac{1}{\sqrt{2}}\\
			\end{align*} 
			We weten dat de cosinus en sinus van de standaardhoek $\pi/4$ beide gelijk zijn aan $\frac{1}{\sqrt{2}}=\frac{\sqrt2}{2}$, en de tegengestelde hoek heeft dezelfde cosinus, maar een tegengestelde sinus. Dus $\theta = - \pi/4$ of $\theta= 7 \pi/4$.
			\begin{image}[0.5\textwidth]
				\begin{tikzpicture}[scale = 1]
				%	grid setup
				
				\def\xmin{-2}
				\def\xmax{2}
				
				\def\ymin{-2}
				\def\ymax{2}				
				
				%	grid
				\draw[step = 1, black, thin, dashed] (\xmin, \ymin) grid (\xmax, \ymax);
				\draw[->] (\xmin, 0) -- (\xmax, 0) node[right]{Re$(z)$};
				\draw[->] (0, \ymin) -- (0, \ymax) node[above]{Im$(z)$};;	
				
				%	vector
				\draw[->, blue, very thick] (0, 0) --  (1, -1) node[above right]{$1-i$};
				
				%	boogje
				\def\r{0.5}
				\draw[->, blue] (\r, 0) arc (0:315:\r) node[midway, above left]{$7\pi/4$};
				\end{tikzpicture}
			\end{image}
		\end{oplossing}
	\end{question}
	
	
	\begin{question} 
		 $z = \dfrac{1-i}{1+i}$
		\begin{uitkomst}  $|z| = 1$ en $\theta=3\pi/2$.
		\end{uitkomst}
		\begin{hint} Schrijf $z$ eerst in cartesische vorm $a+bi$. \end{hint}
		\begin{oplossing}
			Eerst herschrijven we het complex getal $z$ naar de vorm $a + bi$:
			$$
			\frac{1-i}{1+i} = \frac{1-i}{1+i} \frac{1-i}{1-i} = \frac{(1-i)^2}{2}= \frac{-2i}{2} = - i 
			$$
			Dan is $|z| = 1$, en in het comlexe vlak zien we dadelijk dat $\theta = 3 \pi/2$.
			\begin{image}[0.5\textwidth]
				\begin{tikzpicture}[scale = 0.7]
				%	grid setup
				
				\def\xmin{-2}
				\def\xmax{2}
				
				\def\ymin{-2}
				\def\ymax{2}				
				
				%	grid
				\draw[step = 1, black, thin, dashed] (\xmin, \ymin) grid (\xmax, \ymax);
				\draw[->] (\xmin, 0) -- (\xmax, 0) node[right]{Re$(z)$};
				\draw[->] (0, \ymin) -- (0, \ymax) node[above]{Im$(z)$};;	
				
				%	vector
				\draw[->, blue, very thick] (0, 0) -- node[right]{$-i$}  (0, -1);
				
				%	boogje
				\def\r{0.2}
				\draw[->, blue] (\r, 0) arc (0:270:\r) node[midway, left]{$3\pi/2$};
				\end{tikzpicture}
			\end{image}
		\end{oplossing}
	\end{question}    
	
	%%% Volgende oefening in comment: geen standaardhoek, is minder interessant.
	
	
	%\begin{question} 
	%Als $z = \dfrac{1}{1+i}+\dfrac{1+i}{4}$, dan is $|z| = \answer[onlineshowanswerbutton]{\sqrt{10}/4}$ en $\theta=\answer[onlineshowanswerbutton]{-0.32}$.
	%\begin{hint}
	%Om $\theta$ te berekenen, gebruik je best een rekenmachine.
	%\end{hint}
	%\begin{oplossing}
	%Eerst herschrijven we het complex getal $z$ naar de vorm $a + bi$. We doen dit eerst voor de eerste term:
	%$$
	%\frac{1}{1+i} = \frac{1}{1+i} \frac{1-i}{1-i} = \frac{1-i}{2}
	%$$
	%Dan is 
	%$$
	%z = \frac12 - \frac{i}{2} + \frac14  + \frac{i}{4} = \frac34 - \frac14 i \, .
	%$$
	%Dan vinden we dat $|z| = \sqrt{a^2 + b^2} = \sqrt{\frac{10}{16}} = \frac{\sqrt{10}}{4}$. Uit $a = r\cos\theta$ halen we dan dat $\theta = \text{bgcos}\left(\frac{3}{\sqrt{10}}\right)$. Het resultaat kan je nu berekenen met een rekenmachine. 
	%\begin{image}[0.5\textwidth]
	%\begin{tikzpicture}[scale = 1, trig format = rad]
	%	%	grid setup
	%	
	%	\def\xmin{-2}
	%	\def\xmax{2}
	%	
	%	\def\ymin{-2}
	%	\def\ymax{2}				
	%	
	%	%	grid
	%	\draw[step = 1, black, thin, dashed] (\xmin, \ymin) grid (\xmax, \ymax);
	%	\draw[->] (\xmin, 0) -- (\xmax, 0) node[right]{Re$(z)$};
	%	\draw[->] (0, \ymin) -- (0, \ymax) node[above]{Im$(z)$};;	
	%	
	%	%	vector
	%	\draw[->, blue, very thick] (0, 0) -- node[below]{$\frac34 - \frac14 i$}(3/4, -1/4);
	%	
	%	%	boogje
	%	\def\r{0.5}
	%	\draw[->, blue] (\r, 0) arc (0:0.23:\r) node[midway, left]{$0.23$};
	%\end{tikzpicture}
	%\end{image}
	%\end{oplossing}
	%\end{question}    
\end{exercise}	

\begin{exercise} 
	\begin{statement}
		Bereken met de formule van De Moivre
		\end{statement}
	\begin{question} $(\sqrt 3 + i)^3 = \answer[onlineshowanswerbutton]{8i}$
	\end{question}
	\begin{question} $(-1-i)^{20} = \answer[onlineshowanswerbutton]{-1024}$
	\end{question}
	\begin{question} $(1+ i)^{21} = \answer[onlineshowanswerbutton]{-1024-1024i}$
	\end{question}
	\begin{question} $(-\sqrt 3 + i)^5 = \answer[onlineshowanswerbutton]{16\sqrt3+16i}$
	\end{question}
\end{exercise}

\begin{exercise} 
	\begin{statement}
		Bereken alle oplossingen van $z^3=8i$, met $z\in \C$.
		\end{statement}
	\begin{uitkomst}
		$ z_1 = \sqrt 3 + i$,
		$	z_2 = -\sqrt 3 +i $,
		$	z_3 = -2i $.
		
	\end{uitkomst}
	\begin{oplossing}
		We schrijven eerst beide leden van de opgave in de goniometrische schrijfwijze.
		
		De onbekende $z$ schrijven we als $z = r(\cos\theta+i\sin\theta)$. Het linkerlid van de vergelijking wordt dan volgens de formule van De Moivre $$z^3 = r^3 (\cos 3\theta+i\sin 3\theta).$$
		
		Voor het rechterlid $8i$ geldt
		\[ |8i| = 8, \qquad \arg (8i) =  \frac{\pi}{2}, \]
		zodat
		\[ 8i = 8( \cos(\frac{\pi}{2}) + i \sin(\frac{\pi}{2})). \]
		
		De
		vergelijking $z^3=8i$ wordt:
		$$r^3 (\cos 3\theta+i\sin 3\theta) = 8( \cos(\frac{\pi}{2}) + i \sin(\frac{\pi}{2})).$$
		Twee complexe getallen in goniometrische schrijfwijze zijn gelijk aan elkaar als ze dezelfde modulus hebben, en hun argument gelijk is op een veelvoud van $2 \pi$ na.
		Hieruit volgt:
		\[ r^3 = 8 \quad \text{met} \quad r\in \Rplus \qquad \text{en} \qquad 3 \theta =  \frac{\pi}{2} + 2 k \pi \quad \text{met} \quad k\in \Z\]
		Dus $r= 2$, want $r^3=8$ heeft slecht 1 reële oplossing, en $\theta =  \frac{\pi}{6} + k \frac{2\pi}{3}$, $k \in \Z$. 
		
		Voor $k=0$ is
		$\theta =  \frac{\pi}{6}$, voor $k=1$ is $\theta = 
		\frac{\pi}{6} + \frac{2\pi}{3} = \frac{5\pi}{6}$ en voor $k=2$ is $\theta = 
		\frac{\pi}{6} + \frac{4\pi}{3} = \frac{9\pi}{6} = \frac{3\pi}{2}$. Voor alle andere waarden van $k$ krijgen we één van deze hoeken op een veelvoud van $2 \pi$ na. Er zijn dus drie verschillende oplossingen 
		\\$ z_1 = 2( \cos(\frac{\pi}{6}) + i \sin(\frac{\pi}{6}) = 2(\frac{\sqrt3}{2} + i \frac12)= \sqrt 3 + i$,
		\\$	z_2 = 2( \cos(\frac{5\pi}{6}) + i \sin(\frac{5\pi}{6})) =2(- \frac{\sqrt3}{2} + i \frac12)=-\sqrt 3 +i $,
		\\$	z_3 = 2( \cos(\frac{3\pi}{2}) + i \sin(\frac{3\pi}{2})) =2(0 + i (-1))=-2i $.
	\end{oplossing}
\end{exercise}

\xmsection{Niveau 3} 
\begin{exercise}
	\begin{statement}
		Schets in het complexe vlak de gebieden omschreven door volgende vergelijkingen:
	\end{statement}
\begin{question} 
    $1 < |z+2i-1| \leq 2$
    \begin{oplossing} $1 < |z+2i-1| \leq 2$ kunnen we herschrijven als $1 < |z-(1-2i)| \leq 2$. Alle complexe getallen waarvoor $1 < |z+2i-1| \leq 2$ zijn alle complexe getallen $z$ waarvoor de afstand tot $(1-2i)$ groter is dan 1 en kleiner of gelijk is aan 2:
        
        \begin{image}[0.4\textwidth]
            \begin{tikzpicture}[scale=1.2, baseline=(current bounding box.north)]
            
            \draw[dashed] (-1.2, -4.2) grid (3.2, 1.2);
            \draw[->] (-1.2, 0) -- (3.2, 0) node[right] {Re$(z)$};
            \draw[->] (0, -4.2) -- (0, 1.2) node[above] {Im$(z)$};
            
            \draw[blue, thick ,pattern=north west lines, pattern color=blue] (1,-2) circle (2);	
            \fill[white] (1,-2) circle (1);
            \draw (1,-2) node[name=Zi,circle, fill=black, radius=1pt,scale=0.5] {} node [fill=white,xshift=3pt,right] {$1-2i$};
            \end{tikzpicture}
        \end{image}
    \end{oplossing}
\end{question}

\begin{question} 
	$|\text{Im}(z)-i| < \sqrt{2}$
	\begin{oplossing} Als $z=a+bi$, dan betekent $|\text{Im}(z) - i| < \sqrt{2} $ dat $|b-i| < \sqrt{2}$. Wegens de definitie van modulus is $|b-i|= \sqrt{b^2 + (-1)^2}$. De ongelijkheid wordt dus $\sqrt{b^2 + 1} < \sqrt{2}$, en kwadrateren geeft dan de voorwaarde $b^2 + 1 < 2$ ofwel $b^2 < 1$. Aangezien $b = \text{Im}(z)$, is dit gebied precies de horizontale strook $-1 < \text{Im}(z) < 1$. Merk op dat dit symmetrisch is rond de $x$-as. 
		
		\begin{image}[0.5\textwidth]
			\begin{tikzpicture}[scale=0.7, baseline=(current bounding box.north)]
			\draw[dashed] (-5, -2) grid (5, 2);
			
			\draw[->, thick] (-5, 0) -- (5, 0) node[right] {Re$(z)$};
			\draw[->, thick] (0, -2) -- (0, 2) node[above] {Im$(z)$};
			
			\draw[white, pattern=north west lines, pattern color=blue] (-5,-1) rectangle (5, 1);
			\end{tikzpicture}
		\end{image}
	\end{oplossing}            
\end{question}

\begin{question} 
	$|\text{Re}(z)-i| < \sqrt{2}$
	\begin{oplossing} 
		Als $z=a+bi$, dan betekent $|\text{Re}(z) - i| < \sqrt{2} $ dat $|a-i|<\sqrt{2}$. De berekening van de modulus geeft $\sqrt{a^2 +(-1)^2}< \sqrt{2}$. Na kwadrateren geeft dit $a^2+1 < 2$ of dus $-1 < a < 1$. Aangezien $\text{Re}(z) = a$, is dit gebied precies de verticale strook $-1 < \text{Re}(z) < 1$. Merk op dat dit symmetrisch is rond de $y$-as.
		
		\begin{image}[0.4\textwidth]
			\begin{tikzpicture}[scale=1.1, baseline=(current bounding box.north)]
			\draw[dashed] (-2, -3) grid (2, 3);
			
			\draw[->, thick] (-2, 0) -- (2, 0) node[right] {Re$(z)$};
			\draw[->, thick] (0, -3) -- (0, 3) node[above] {Im$(z)$};
			
			\draw[white, pattern=north west lines, pattern color=blue] (-1,3) rectangle (1, -3);
			\end{tikzpicture}
		\end{image}
	\end{oplossing}
\end{question}
\end{exercise}

	\begin{exercise}
	\begin{statement}
		Geef een vierkantsvergelijking van de vorm $ax^2 + bx + c = 0$ die
	\end{statement}
	\begin{question}
		enkel het getal -5 als oplossing heeft.
		\begin{uitkomst} $x^2 +10 x + 25 = 0$
		\end{uitkomst}
					\begin{oplossing}
						De enige oplossing moet $x = -5$ zijn. Hieruit halen we $x + 5 = 0$, maar dit is geen tweedegraadsvergelijking: $(x+5)^2 = 0$ is dat wel. Voluit geschreven is dit de vergelijking $x^2 +10 x + 25 = 0$. Je kan eventueel met de discriminantmethode nagaan dat $D = 0$ en $\frac{-b}{2a} = -5$.
						
						(Met deze truc kan je zelfs een veeltermvergelijking van eender welke graad geven met enkel -5 als oplossing: $(x-5)^n = 0$: ze voluit schrijven is wel wat lastiger)
					\end{oplossing}
	\end{question}
	
	\begin{question}
		$x_1 = 3-i$ en $x_2 = 3+i$ als oplossingen heeft.
		\begin{uitkomst} $x^2-6x+10$
		\end{uitkomst}
					\begin{hint}De gezochte vierkantsvergelijking is $(x-x_1)(x-x_2)=0$ \end{hint}
					\begin{oplossing}
						$(x-x_1)(x-x_2)=(x-3+i)(x-3-i)=x^2 -(3+i)x+(-3+i)x+(-3+i)(-3-i)=x^2-6x+10$
					\end{oplossing}
	\end{question}
\end{exercise}

\begin{exercise}
	\begin{statement}
		Schrijf onderstaande getallen in goniometrische vorm $r(\cos\theta+i\sin \theta)$. Zorg er steeds voor dat $\theta$ tussen 0 en $2\pi$ ligt.
	\end{statement}
\begin{question} 
	$z = \cos\frac{\pi}{4}+i\sin\frac{3\pi}{4}$ 
	\begin{uitkomst} $z = \cos\frac{\pi}{4}+i\sin\frac{\pi}{4}$
	\end{uitkomst}
			\begin{oplossing} 
				Door supplementaire hoeken te gebruiken vinden we dat 
				$$
				\sin \left( \frac{3\pi}{4}\right) = \sin \left( \pi-\frac{\pi}{4} \right) = \sin \left(\frac{\pi}{4} \right) \, .
				$$
				En dus is $z = \cos\frac{\pi}{4}+i\sin\frac{\pi}{4}$: dit is reeds de goniometrische schrijfwijze, met $r = 1$ en $\theta = \frac{\pi}{4}$.
				\begin{image}[0.5\textwidth]
					\begin{tikzpicture}[scale = 1.8]
					%	grid setup
					
					\def\xmin{-1.5}
					\def\xmax{1.5}
					
					\def\ymin{-1.5}
					\def\ymax{1.5}				
					
					%	grid
					\draw[step = 1, black, thin, dashed] (\xmin, \ymin) grid (\xmax, \ymax);
					\draw[->] (\xmin, 0) -- (\xmax, 0) node[right]{Re$(z)$};
					\draw[->] (0, \ymin) -- (0, \ymax) node[above]{Im$(z)$};;	
					
					%	vector
					\draw[->, blue, very thick] (0, 0) -- ({cos(45)}, {sin(45)}) node[above, fill = white]{$\cos\left(\frac{\pi}{4}\right) + \sin\left(\frac{3\pi}{4}\right) i$};
					
					%	boogje
					\def\r{0.5}
					\draw[->, blue] (\r, 0) arc (0:45:\r) node[midway, right]{$\pi/4$};
					\end{tikzpicture}
				\end{image}
			\end{oplossing}
\end{question}

\begin{question} 
	$z = \cos\frac{\pi}{3} - i\sin\frac{\pi}{3}$ 
	\begin{uitkomst} $z = \cos\frac{5\pi}{3} - i\sin\frac{5\pi}{3}$
	\end{uitkomst}
			\begin{oplossing}
				Door tegengestelde hoeken te gebruiken vinden we dat $ - \sin\left( \frac{\pi}{3} \right) = \sin \left(-\frac{\pi}{3}\right)$ en $\cos \left(\frac{\pi}{3}\right) = \cos \left(- \frac{\pi}{3} \right)$. Dus $ z = \cos \left(-\frac{\pi}{3}\right) + i\sin \left(-\frac{\pi}{3} \right)$. Als we eisen dat $\theta \in [0, 2\pi]$, dan is de goniometrische vorm van $z$: $z = \cos\frac{5\pi}{3} - i\sin\frac{5\pi}{3}$.
				
				\begin{image}[0.5\textwidth]
					\begin{tikzpicture}[scale = 1.5]
					%	grid setup
					
					\def\xmin{-2}
					\def\xmax{2}
					
					\def\ymin{-2}
					\def\ymax{2}				
					
					%	grid
					\draw[step = 1, black, thin, dashed] (\xmin, \ymin) grid (\xmax, \ymax);
					\draw[->] (\xmin, 0) -- (\xmax, 0) node[right]{Re$(z)$};
					\draw[->] (0, \ymin) -- (0, \ymax) node[above]{Im$(z)$};;	
					
					%	vector
					\draw[->, blue, very thick] (0, 0) -- ({cos(-60)}, {sin(-60)}) node[below, fill = white]{$\cos\left(\frac{\pi}{3}\right) - i\sin\left(\frac{\pi}{3}\right)$};
					
					%	boogje
					\def\r{0.5}
					\draw[->, blue] (\r, 0) arc (0:300:\r) node[midway, above]{$5\pi/3$};
					\end{tikzpicture}
				\end{image}
			\end{oplossing}
\end{question}
\end{exercise}

\end{document}