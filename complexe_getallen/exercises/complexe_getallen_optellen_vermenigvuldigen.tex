\documentclass{ximera}
\input{../../preamble.tex}
\addPrintStyle{../..}


\begin{document}
    \author{Ingmar Herreman}
    \date{September 2022}
    \xmtitle{Oefeningen optelling en vermenigvuldigen complexe getallen}{}

\begin{exercise} Bereken.
\begin{question}
 $(2+i)+(3-i)$ $= \answer{5}$
 \end{question}
 \begin{question}
 $(5-2i)+(6+3i)$ $= \answer{11+i}$
 \end{question}
 \begin{question} $(1+i)+(3-2i)+(1+i)$ $= \answer{5}$
 \end{question}
 \begin{question} $1-3i-(2-6i)-(3i-4)$ $= \answer{3}$
 \end{question}
 \begin{question} $-2(1-2i)$ $= \answer{-2+4i}$
 \end{question}
 \begin{question} $2i(1+i)$ $= \answer{-2+2i}$
 \end{question}
 \begin{question} $6i\left(\dfrac{1}{2}+\dfrac{1}{3}i\right)$ $= \answer{-2+3i}$
 \end{question}
 \begin{question} $i(2-i\sqrt{3})$ $= \answer{\sqrt{3}+2i}$
 \end{question}
 \begin{question} $(2+i)(3+i)$ $= \answer{5+5i}$
 \end{question}
 \begin{question} $(2-i)(1+2i)$ $= \answer{4+3i}$
 \end{question}
 \begin{question} $(5-4i)(3+i)$ $= \answer{19-7i}$
 \end{question}
 \begin{question} $(11+2i)(2-5i)$ $= \answer{32-51i}$
 \end{question}
 \begin{question} $(1+2i)(5i-4)$ $= \answer{-14-3i}$
 \end{question}
\end{exercise}

\begin{exercise}
Bereken.


 \begin{question} $i\cdot 2i \cdot (4i-1)$ $= \answer{2-8i}$
 \end{question}
 \begin{question} $(4-5i)(5-4i)(1+i)$ $= \answer{41-41i}$
 \end{question}
 \begin{question} $(1-i)(4+5i)(5-4i)$ $= \answer{49-31i}$
 \end{question}
 \begin{question} $(i-1)(i+5)-3i(2i+3)$ $= \answer{13i}$
 \end{question}
 \begin{question} $(1+i)(1-i)$ $= \answer{2}$
 \end{question}
 \begin{question} $(5-6i)(5+6i)$ $= \answer{61}$
 \end{question}
 \begin{question} $(7+i)(7-i)$ $= \answer{50}$
 \end{question}
 \begin{question} $(\cos(x)+i\cdot \sin(x))\cdot (\cos(x)-i\cdot \sin(x))$ $= \answer{1}$
 \end{question}
 \begin{question} $(7+i)(1+2i)(7-i)(1-2i)$ $= \answer{250}$
 \end{question}
 \begin{question} $(4+i)(4-i)-(3-2i)(3+2i)$ $= \answer{4}$
 \end{question}

\end{exercise}

\end{document}
