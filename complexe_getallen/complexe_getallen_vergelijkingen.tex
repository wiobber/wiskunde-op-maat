\documentclass{ximera}
\input{../preamble}
\addPrintStyle{..}
\begin{document}
	\author{Zomercursus KU Leuven}
	\xmtitle{Vergelijkingen over de complexe getallen}{}
	
	\label{xim:complexe_getallen_vergelijkingen}
    
Een belangrijke motivatie om de verzameling $\C$ van de \hyperref[xim:complexe_getallen]{complexe getallen} in te voeren was het oplossen van de vergelijking $x^2+1=0$. Dat bleek mogelijk door aan de reële getallen een nieuw 'getal' $i$ toe te voegen dat \textit{per definitie} een oplossing is van de vergelijking $x^2+1=0$. Er geldt dus \textit{per definitie} dat $i$ een getal is zodat  $i^2=-1$. We onderzoeken hier of we nu ook andere veeltermvergelijkingen kunnen oplossen die dat niet waren in $\R$. Het resultaat is spectaculair, en is bevat in de zogenaamde Hoofdstelling van de algebra:

\begin{theorem}[Hoofdstelling van de algebra (Stelling van Gauss)]\nl
\textit{Elke} veeltermvergelijking (van graad minstens 1) is oplosbaar over de complexe getallen.
\end{theorem}

\begin{remark}\nl
\begin{itemize}
\item er moet natuurlijk wel een echte vergelijking zijn: volgens de definitie van veeltermen is $5$ immers een veelterm (van graad 0), maar de vergelijking $5=0$ heeft toch geen oplossingen. De Hoofdstelling zegt dat de constante veeltermen de enige zijn die geen nulpunten hebben over de complexe getallen.
\item je kan de stelling ook formuleren met nulpunten van veeltermen in plaats van oplossingen van vergeljkingen: elke niet-contsante veelterm heeft minstens één nulpunt over de complexe getallen.
%\item de \textit{graad} van een veeltermvergelijking is de hoogste graad waarmee de onbekende voorkomt in die vergelijking. Bijvoorbeeld is $x^3+x^2+x+1=0$ van de derde graad, $x^5 - 3x +  2 = 0$ van de vijfde, $x^{2020} = 2020$ van de 2020-ste graad en zo verder. 
\item Als we een veelterm $p(x)$ kunnen schrijven als $p(x) = (x-x_1)q(x)$, met $x_i\in\C$, dan is $x_1$ duidelijk een oplossing van de vergelijking $p(x)=0$, want  $p(x_1) = (x_1-x_1)q(x_1)\perinderdaad 0$). Men kan via het algoritme van de \hyperref[xim:euclidische_deling]{Euclidische deling} aantonen dat ook het omgekeerde geldt: als voor een veelterm $p(x)$ geldt dat $p(a)=0$, dan bestaat er altijd een veelterm $q(x)$ zodat $p(x)=(x-a)q(x)$. Als $p(x)$ graad $n$ heeft, dan heeft  $q(x)$ graad $n-1$. Als $n-1>0$, dan heeft volgens de Hoofdstelling ook $q(x)=0$ een oplossing, zeg $x_2$, en bestaat er dus een veelterm $q_2(x)$ zodat $q(x) = (x-x_2)q_2(x)$. Door deze procedure te herhalen krijgen we:
\begin{proposition}
Elke veelterm  $p(x)=\sum_{i=0}^n a_ix^i$ kan ontbonden worden in lineaire factoren als
$$
p(x)=\sum_{i=0}^n a_ix^i = a_n\prod_{i=1}^n (x-x_i)
$$

Een equivalente formulering is:

Elke veelterm van graad $n$ heeft precies $n$ (niet noodzakelijk verschillende) nulpunten.
\end{proposition}
Elke veelterm kan dus geschreven worden als een \textit{som} van machten in $x$ (per definitie) of als \textit{product} van uitdrukkingen $x-x_1$.
\item de stelling niet zegt dat deze $n$ oplossingen onderling verschillend zijn: de vergelijking $x^2 = 0$ heeft als het ware 'twee keer' de oplossing $x=0$, en $x^2+-2x+1 = (x-1)^2 = 0$ heeft 'twee keer' de oplossing $x=1$. 
\item de stelling geldt niet alleen voor vergelijkingen met reële coëfficiënten, maar ook voor vergelijkingen met complexe coëfficiënten. Er bestaan dus $5$ complexe getallen $x$ waarvoor $ix^5 +2x^3 +2i = 0$. 
\item het bewijs van deze stelling is niet eenvoudig, en gebruikt tamelijk geavanceerde eigenschappen van limieten en  reële getallen.
\item de stelling zegt alleen iets over het \textit{bestaan} van oplossingen. Ze geeft geen methode om de oplossingen ook effectief te berekenen.
\end{itemize}
\end{remark}

We bekijken concreet de vierkantsvergelijkingen, dus de veeltermvergelijkingen van de tweede graad
$$
ax^2 + bx + c = 0 \quad\text{ met } a\neq 0
$$

De Hoofdstelling van de algebra zegt dat vergelijkingen van graad twee altijd twee oplossingen hebben, waarbij we mogelijk één oplossing 'twee keer moeten tellen'.

De zogenaamde discriminant is per definite het getal $D = b^2 - 4ac$. Over de reële getallen beweerden we dat de vergelijking oplosbaar was in het geval $D > 0$ of $D = 0$, en dat er geen oplossingen waren voor $D < 0$. Over de complexe getallen zijn er echter \textit{altijd} oplossingen:

%Dan hadden we de volgende opties:
%\begin{itemize}
%	\item $D > 0$: de vergelijking heeft twee verschillende oplossingen,
%	\item $D = 0$: de vergelijking heeft één unieke oplossing,
%	\item $D < 0$: de vergelijking heeft \textit{geen reële} oplossingen
%\end{itemize}

\begin{proposition}[Oplossen van vierkantsvergelijkingen over de complexe getallen]\nl
De twee oplossingen $x_1$ en $x_2$ van een vierkantsvergelijking $ax^2 + bx + c = 0$ zijn
	$$ 
	x_1 = \frac{-b + \sqrt{b^2 - 4ac}}{2a}, \quad	x_2 = \frac{-b - \sqrt{b^2 - 4ac}}{2a}
	$$
\end{proposition}

\begin{remark} \nl
\begin{itemize}
\item Merk op dat $a\neq 0$ (want als $a=0$ is dit geen vierkantsvergelijking maar een lineaire vergelijking $bx+c=0$), en dat er dus geen probleem is met mogelijk 'delen door nul'.

\item Afhankelijk van de waarde van de discriminant  $D = b^2 - 4ac$  geldt:
\begin{itemize}
	\item $D \neq 0$: de vergelijking heeft twee verschillende oplossingen, namelijk
	$$ 
	x_1 = \frac{-b + \sqrt{D}}{2a}, \quad	x_2 = \frac{-b - \sqrt{D}}{2a}
	$$
	\item $D = 0$: de vergelijking heeft één unieke oplossing, namelijk
	$$
	x_{1,2} = \frac{-b}{2a}
	$$
\end{itemize}
\item Als $a,b,c$ \textit{reële} getallen zijn, dan is ook de discriminant $D$ reëel en heeft het zin om over het \textit{teken} van $D$ te spreken. Als $D$ dan negatief is, is $\sqrt{D} = i\sqrt{|D|}$ zuiver imaginair, en zijn de wortels $x_1$ en $x_2$ complex.
\end{itemize}
\end{remark}


In het voorbeeld $x^2 + 1 = 0$ is $a = 1$, $b=0$ en $c = 1$. Dus is $D = b^2 - 4ac = -4 < 0$, en de oplossingen zijn volgens de discriminantmethode 
$$
x_1 = \frac{-b + \sqrt{D}}{2a} = \frac{0 + 2i}{2} = i, \ \ x_2 = \frac{-b - \sqrt{D}}{2a} = \frac{0 - 2i}{2} = -i \, .
$$
Dit komt overeen met de oplossingen die we wisten vanuit de definitie van $i$. 

%Ga zelf na aan de hand van de discriminantmethode dat $D < 0$ voor de vergelijking $x^2 + 4 = 0$, en dat de oplossingen $x=2i$ en $x=-2i$ zijn. 

\begin{xmuitweiding}[Complexe oplossingen van reële vierkantsvergelijkingen zijn elkaars complex toegevoegde]
Een vierkantsvergelijking met reële coëfficiënten heeft steeds twee oplossingen. Als de oplossingen reëel zijn, dan kunnen ze verschillend zijn van elkaar (geval $D < 0$) of gelijk zijn aan elkaar (geval $D = 0$). Als de oplossingen complex zijn, dan zijn ze altijd verschillend.

Er kan nog meer gezegd worden over deze complexe oplossingen. Kijk eens terug naar de twee voorbeelden van vierkantsvergelijkingen waarbij complexe oplossingen opduiken: $x^2 + 1 = 0$ en $x^2 + 4 = 0$. Beide oplossingen waren niet zomaar verschillend van elkaar: ze waren elkaars complex toegevoegde! Dit blijkt \textit{altijd} het geval te zijn: als in het geval $D < 0$ het getal $z \in \C$ een oplossing is, dan is $\overline{z}$ de andere oplossing.

Dit is natuurlijk geen wonder: de vergelijkingen van de discriminantmethode geven voor het geval $D < 0$ de oplossingen
$$
x_{1,2} = \frac{-b \pm \sqrt{D}}{2a} \, .
$$ 
De twee oplossingen verschillen van teken bij de wortel van de discriminant, en dat is net de enige plek waar een imaginaire eenheid tevoorschijn komt. 
\end{xmuitweiding}
\end{document}

