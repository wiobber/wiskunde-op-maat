\documentclass{xourse}
\input{../preamble.tex}
% 
% to make slides with tcbraster ...
%
\tcbuselibrary{raster}
\tcbuselibrary{breakable,skins}

%\usetikzlibrary{external}
%\tikzexternalize
%\tcbset{shield externalize}

\usepackage{pdflscape}
\usepackage{pdfpages}
   % voor formularia ...
\addPrintStyle{..}


%\printactivitytitlefalse
% Met QQ en denkvragen

\begin{document}
	\setcounter{tocdepth}{1}
    \xmtitle{Complexe getallen (8uA)}  
    

\part{Complexe getallen}
\activitychapter{inleiding_genetisch.tex}
%\activitychapter{cmplx_definitie_algebraisch.tex}
%\activitychapter{cmplx_definitie_complexe_vlak.tex}
\activitychapter{cmplx_definitie_meetkundig.tex}
\activitychapter{cmplx_optellen_en_vermenigvuldigen.tex}

\activitychapter{cmplx_deling_dhz.tex}

\activitychapter{cmplx_norm.tex} 
\activitychapter{cmplx_complex_toegevoegde.tex}
\activitychapter{cmplx_inverse_en_deling.tex}
\activitychapter{cmplx_tweedegraadsvergelijkingen.tex}

\practicesection{exercises/complexe_getallen_basis.tex}

%\part{Complexe getallen (8u)}
%\activitychapter{cmplx_vierkantswortels.tex}
%\activitychapter{cmplx_structuur.tex}

\part{Goniometrische voorstelling}

%\part{Complexe getallen (work-in-progress)}



\part{Formularia}
{\providecommand{\wraptitle}{Complexe getallen (versie Burgerlijk Ingenieur KU Leuven)}
\activitychapter{../formularium/complexe_getallen.tex}
}

\end{document}
