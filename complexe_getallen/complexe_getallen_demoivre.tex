\documentclass{ximera}
\input{../preamble}
\addPrintStyle{..}
\begin{document}
	% Start specifieke settings:    
	\author{Zomercursus KU Leuven}
	\xmtitle{Machten van complexe getallen: formule van De Moivre}{}
	% Start inhoud ximera 
	
	\label{xim:complexe_getallen_demoivre}

Met de formules van het product van complexe getallen kunnen we natuurlijk ook het \textit{kwadraat} van een complex getal berekenen, door eenvoudig twee keer hetzelfde getal in te vullen. In de cartesische schrijfwijze is dat mooi, maar niet spectaculair. In de goniometrische schrijfwijze vinden we iets bijzonder interessants:

Een kwadraat van een complex getal $z=a+bi = r (\cos \theta + i \sin \theta )$ is
$$
z^2 = a^2 - b^2 -2abi = r^2(\cos 2\theta + i \sin 2\theta )
$$

Als we een derde of vierde macht willen berekenen worden de formules met $a$ en $b$ steeds ingewikkelder, maar de formules in $r$ en $\theta$ blijven verrassend eenvoudig:

\begin{proposition}[Formule van De Moivre]\nl
	
Voor een complex getal $z= r (\cos \theta + i \sin \theta )$ en elk natuurlijk getal $n$ geldt
$$
z^n = r^n(\cos n\theta + i \sin n\theta )
$$
\end{proposition}

\begin{basicSkip}
Onderstaande applet illustreert de geometrische betekenis van deze formule: $z=1+i$ is vast gekozen en weergegeven in het complexe vlak. De modulus van $z$ is $|z|=r=\sqrt{2}$. Je kan met het schuivertje een macht $n$ opgeven en daarna $z^n$ op de figuur laten weergeven. Eerst wordt $r^n= (\sqrt{2})^n$ berekend en weergegeven op de reële as. Daarna wordt $\theta$ vermenigvuldigd met $n$ en wordt ook $z^n$ weergegeven.

\geogebra{nglwjJ79}{916}{721}
\end{basicSkip}

\begin{example}
	We kunnen $(1+i)^8$ op 2 manieren berekenen:
			
		\begin{enumerate}
			\item $(1+i)(1+i)=1+2i-1=2i$, dus $(1+i)(1+i)(1+i)(1+i)=(2i)^2=-4$, dus $(1+i)^8=(-4)^2=16$
			\item $1+i=\sqrt{2}(\cos \frac{\pi}{4}+i\sin\frac{\pi}{4})$, dus $(1+i)^8=(\sqrt{2})^8(\cos  \frac{8\pi}{4}+i\sin \frac{8\pi}{4})=2^4(\cos 2\pi +i \sin 2\pi)=16(1+i\cdot 0)=16$
		\end{enumerate}
		
	\end{example}

\begin{basicSkip}
\begin{exercise} Bereken met de formule van De Moivre
	\begin{question} $(\sqrt 3 + i)^3 = \answer[onlineshowanswerbutton]{8i}$
		\end{question}
	\begin{question} $(-1-i)^{20} = \answer[onlineshowanswerbutton]{-1024}$
	\end{question}
\begin{question} $(1+ i)^{21} = \answer[onlineshowanswerbutton]{-1024-1024i}$
\end{question}
\begin{question} $(-\sqrt 3 + i)^5 = \answer[onlineshowanswerbutton]{16\sqrt3+16i}$
\end{question}
	\end{exercise}
\end{basicSkip}

Ook vergelijkingen van de vorm: $z^n=a+bi$ met $z\in \C, n\in \Z$ zijn met de formule van De Moivre op te lossen. Volgens de \hyperref[def:hoofdstelling_algebra]{Hoofdstelling van de Algebra} heeft de vergelijking $z^n=a+bi$ steeds $n$ oplossingen. Het oplossen van de vergelijking $z^n=a+bi$ komt neer op het zoeken van de n-de machtswortels van $a+bi$.

\begin{exercise}
	Bereken alle oplossingen van $z^2 = 1 - i$, met $z \in \C$.
	\begin{oplossing}
		We schrijven eerst beide leden van de opgave in de goniometrische schrijfwijze.
		
		De onbekende $z$ schrijven we als $z = r(\cos\theta+i\sin\theta)$. Het linkerlid van de vergelijking wordt dan volgens de formule van De Moivre $$z^2 = r^2 (\cos 2\theta+i\sin 2\theta).$$
		
		Voor het rechterlid $1-i$ geldt
		\[ |1-i| = \sqrt{2}, \qquad \arg (1-i) = - \frac{\pi}{4}, \]
		zodat
		\[ 1-i = \sqrt{2}( \cos(-\frac{\pi}{4}) + i \sin(-\frac{\pi}{4})). \]
		
		De
		vergelijking $z^2 = 1-i$ in de onbekende $z$ wordt zo een vergelijking in de onbekenden $r$ en $\theta$:
		$$r^2 (\cos 2\theta+i\sin 2\theta) = \sqrt{2} ( \cos(-\frac{\pi}{4}) + i \sin(-\frac{\pi}{4})).$$
		Twee complexe getallen in goniometrische schrijfwijze zijn gelijk aan elkaar als ze dezelfde modulus hebben, en hun argument gelijk is op een veelvoud van $2 \pi$ na.
		Hieruit volgt:
		\[ r^2 = \sqrt{2} \quad \Tmet \quad r\in \Rplus \qquad \Ten \qquad 2 \theta = - \frac{\pi}{4} + 2 k \pi \quad \Tmet \quad k\in \Z\]
		Dus $r= 2^{1/4}$ en $\theta = - \frac{\pi}{8} + k \pi$, $k \in \Z$. We hebben de onbekenden $r$ en $\theta$ dus gevonden.
		
		 Voor $k=0$ is
		$\theta = - \frac{\pi}{8}$ en voor $k=1$ is $\theta = -
		\frac{\pi}{8} + \pi = \frac{7\pi}{8}$. Voor alle andere waarden van $k$ krijgen we één van beide hoeken op een veelvoud van $2 \pi$ na. Er zijn dus slechts twee verschillende oplossingen 
		\[ z_1 = 2^{1/4}( \cos(-\frac{\pi}{8}) + i \sin(-\frac{\pi}{8}))  \qquad \Ten \qquad
		z_2 = 2^{1/4}( \cos(\frac{7\pi}{8}) + i \sin(\frac{7\pi}{8}))  \]
		
	\end{oplossing}
\end{exercise}

\begin{example}
	Zoek alle vierdemachtswortels uit $1$.
	\begin{oplossing}
	We zoeken dus alle oplossingen van de vergelijking $z^4=1.$
	
	We stellen $z = r(\cos\theta+i\sin\theta)$. 
	Het linkerlid van de vergelijking wordt dan volgens de formule van De Moivre $$r^4(\cos 4\theta+i\sin 4\theta).$$
	
	Voor het rechterlid $1$ geldt
	\[ |1| = 1, \qquad \arg (1) = 0, \]
	zodat
	\[ 1 = 1(\cos 0+i\sin 0). \]
	
	De
	vergelijking $z^4=1$ in de onbekende $z$ wordt zo een vergelijking in de onbekenden $r$ en $\theta$:
	
	$$r^4(\cos 4\theta+i\sin 4\theta) =1(\cos 0+i\sin 0)$$
	Gelijkheid van complexe getallen leert ons dat $$\mbox{$r^4=1$ \Tmet $r\in \Rplus$ \Ten
		$4\theta
		=0+2k\pi$ \Tmet $k\in \Z.$}$$ Dus $r=1$, want $r^4=1$ heeft slecht 1 reële oplossing, en $\theta=k\pi/2$ met $k\in
	\Z.$
	\\ Dit levert vier verschillende wortels.
	\\$z_1=1(\cos 0+i\sin 0)=1$,
	\\$z_2=1(\cos \pi/2+i\sin \pi/2) =i$,
	\\$z_3=1 (\cos \pi+i\sin \pi) =-1,$
	\\$z_4=1 (\cos 3\pi/2+i\sin 3\pi/2)=-i.$
	
%	In het complex vlak liggen al deze wortels op de eenheidscirkel
%	rond de oorsprong, ze vormen de hoekpunten van een vierkant met
%	\'e\'en
%	hoekpunt in het punt $z=1$.
	
	Dit voorbeeld kan je veel sneller oplossen door $z^4-1$ te ontbinden in factoren:
	$$ z^4-1=(z^2-1)(z^2+1)$$
	De nulpunten hiervan zijn inderdaad 1,-1,i,-i. Ontbinden in factoren lukt echter niet meer bij bijvoorbeeld $z^5=1$, hiervoor moet je bovenstaande methode gebruiken.
	\end{oplossing}
	\end{example}

\begin{basicSkip}
	\begin{exercise} Bereken alle oplossingen van $z^3=8i$, met $z\in \C$.
		\begin{oplossing}
			We schrijven eerst beide leden van de opgave in de goniometrische schrijfwijze.
			
			De onbekende $z$ schrijven we als $z = r(\cos\theta+i\sin\theta)$. Het linkerlid van de vergelijking wordt dan volgens de formule van De Moivre $$z^3 = r^3 (\cos 3\theta+i\sin 3\theta).$$
			
			Voor het rechterlid $8i$ geldt
			\[ |8i| = 8, \qquad \arg (8i) =  \frac{\pi}{2}, \]
			zodat
			\[ 8i = 8( \cos(\frac{\pi}{2}) + i \sin(\frac{\pi}{2})). \]
			
			De
			vergelijking $z^3=8i$ wordt:
			$$r^3 (\cos 3\theta+i\sin 3\theta) = 8( \cos(\frac{\pi}{2}) + i \sin(\frac{\pi}{2})).$$
			Twee complexe getallen in goniometrische schrijfwijze zijn gelijk aan elkaar als ze dezelfde modulus hebben, en hun argument gelijk is op een veelvoud van $2 \pi$ na.
			Hieruit volgt:
			\[ r^3 = 8 \quad \Tmet \quad r\in \Rplus \qquad \Ten \qquad 3 \theta =  \frac{\pi}{2} + 2 k \pi \quad \Tmet \quad k\in \Z\]
			Dus $r= 2$, want $r^3=8$ heeft slecht 1 reële oplossing, en $\theta =  \frac{\pi}{6} + k \frac{2\pi}{3}$, $k \in \Z$. 
			
			Voor $k=0$ is
			$\theta =  \frac{\pi}{6}$, voor $k=1$ is $\theta = 
			\frac{\pi}{6} + \frac{2\pi}{3} = \frac{5\pi}{6}$ en voor $k=2$ is $\theta = 
			\frac{\pi}{6} + \frac{4\pi}{3} = \frac{9\pi}{6} = \frac{3\pi}{2}$. Voor alle andere waarden van $k$ krijgen we één van deze hoeken op een veelvoud van $2 \pi$ na. Er zijn dus drie verschillende oplossingen 
			\\$ z_1 = 2( \cos(\frac{\pi}{6}) + i \sin(\frac{\pi}{6}) = 2(\frac{\sqrt3}{2} + i \frac12)= \sqrt 3 + i$,
			\\$	z_2 = 2( \cos(\frac{5\pi}{6}) + i \sin(\frac{5\pi}{6})) =2(- \frac{\sqrt3}{2} + i \frac12)=-\sqrt 3 +i $,
			\\$	z_3 = 2( \cos(\frac{3\pi}{2}) + i \sin(\frac{3\pi}{2})) =2(0 + i (-1))=-2i $.
		\end{oplossing}
		\end{exercise}
	\end{basicSkip}

\end{document}

