\documentclass{ximera}
\input{../preamble}
\addPrintStyle{..}
\begin{document}
	\author{Wim Obbels}
	\xmtitle{Complexe getallen: eindtermen}{}
	\label{xim:complexe_getallen_xeindtermen}
    

    \xmsection{Uitgebreide wiskunde i.f.v. wetenschappen}

    \begin{enumerate}
            \item  [06.04.19]  De leerlingen stellen complexe getallen voor in het vlak.	
            \item  [06.04.20]  De leerlingen voeren bewerkingen met complexe getallen in cartesische vorm uit: optelling, aftrekking, vermenigvuldiging en deling.	
            \item  [06.04.21]  De leerlingen lossen tweedegraadsvergelijkingen met reële coëfficiënten in één onbekende op in de verzameling van de complexe getallen.	
            \item  [06.04.22]  De leerlingen zetten complexe getallen in cartesische vorm om naar goniometrische vorm en omgekeerd.	
            \item  [06.04.23]  De leerlingen voeren de vermenigvuldiging van complexe getallen in goniometrische vorm uit.	"'

                    Onderliggend: Goniometrische formules: somformules"

    \end{enumerate}

    \xmsection{Gevorderde wiskunde}

    \begin{enumerate}
        \item 	[06.08.24]	De leerlingen stellen complexe getallen voor in het vlak.
        \item 	[06.08.25]	De leerlingen voeren bewerkingen uit met complexe getallen in cartesische vorm: optelling, aftrekking, vermenigvuldiging en deling.
        \item 	[06.08.26]	De leerlingen lossen tweedegraadsvergelijkingen met reële coëfficiënten in één onbekende op in de verzameling van de complexe getallen.
        \item 	[06.08.27]	De leerlingen zetten complexe getallen in cartesische vorm om naar goniometrische vorm en omgekeerd.
        \item 	[06.08.28]	De leerlingen voeren bewerkingen uit met complexe getallen in goniometrische vorm: vermenigvuldiging, deling, machtsverheffing en n-de machtsworteltrekking.
        
                Onderliggend:  Formule van de Moivre
    \end{enumerate}


    \xmsection{Wiskundeplan (6u-8u; te voorzien aantal lesuren: 15u)}

    \begin{enumerate}
    \item Complexe getallen en hun voorstelling in het complexe vlak
    \item Bewerkingen in cartesiaanse vorm: optelling, vermenigvuldiging, deling
        \begin{itemize}
            \item Tweedegraadsvergelijkingen met reële coëfficiënten
            \item (U6) Tweedegraadsvergelijkingen met complexe coëfficiënten
        \end{itemize}
    \item Bewerkingen in goniometrische vorm: vermenigvuldiging, deling
        \begin{itemize}
            \item Machten met gehele exponent (formule van de Moivre)
            \item N-de machtswortels
        \end{itemize}
    \item (8u) veeltermen in $\C$

    \end{enumerate}
    
    

\end{document}

