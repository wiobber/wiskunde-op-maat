\documentclass{ximera}
\input{../preamble}
\addPrintStyle{..}
\begin{document}
	% Start specifieke settings:    
	\author{Zomercursus KU Leuven}
	\xmtitle{De hoofdstelling van de algebra}{In de complexe getallen heeft elke veelterm nulpunten}
	% Start inhoud ximera 
	
	\label{xim:hoofdstelling_algebra}

% OBSOLETE: zoe complexe_getallen_vergelijkingen

\begin{proposition}[Hoofdstelling van de algebra (Stelling van Gauss)] \label{def:hoofdstelling}
	
	Een complexe veelterm van de $n$-de graad heeft steeds precies $n$ complexe nulpunten 
	\\ (niet noodzakelijk allemaal onderling verschillend).
	
	Dus: voor een veelterm $a_nx^n+a_{n-1}x^{n-1}+ \cdots +a_2x^2+a_1x+a_0$ met coëfficiënten $a_i\in \C$
	\\
	bestaan er $z_1,z_2,\ldots,z_n \in \C,$ zo dat 
	$$
	a_nx^n+a_{n-1}x^{n-1}+ \cdots +a_2x^2+a_1+a_0=a_n(x-z_1)(x-z_2)\cdots (x-z_n)
	$$ 
	Of: er bestaan $z_1,z_2,\ldots,z_n \in\C$ die oplossingen (of wortels) zijn van de vergelijking
	$$
	a_nx^n+a_{n-1}x^{n-1}+ \cdots +a_2x^2+a_1x+a_0=0
	$$
	( dus waarvoor $a_nz_i^n+a_{n-1}z_i^{n-1}+ \cdots +a_2z_i^2+a_1x+a_0=0$)
\end{proposition}


Deze stelling is een boeiend resultaat.  Kinderen die voor het eerst
over getallen leren, werken met natuurlijke (positieve gehele)
getallen: $0,1,2,3,\ldots$ Op een bepaald ogenblik blijkt het elegant
te zijn om ook negatieve getallen in het plaatje toe te laten:
$-1,-2,-3,\ldots$. Hierdoor krijgen bewerkingen als  '$3$ min $7$' ook een oplossing.  Iets gelijkaardigs
gebeurt als we breuken (of rationale getallen) toelaten: ook '$3$
gedeeld door $7$' wordt als getal aanvaard. Over de uitbreiding van
rationale getallen naar re\"ele getallen valt veel te zeggen, waar we
nu niet op ingaan. De hoofdstelling van de algebra zegt dat er
iets strafs gebeurt bij de stap van re\"ele naar complexe getallen.
Bij de re\"ele getallen bestaat de vierkantswortel uit $-1$ niet. Bij
de complexe getallen voeren we die vierkantswortel toch in. D.w.z. we
geven hem een naam '$i$'. Of om preciezer te zijn, eigenlijk zijn er
twee vierkantswortels uit $-1$, namelijk $i$ en $-i$. Of nog anders
gezegd, geven we een naam aan de (binnen de re\"ele getallen niet
bestaande) oplossingen van de veeltermvergelijking $x^2+1=0$. De
veelterm $x^2+1$ kan dan geschreven worden als $(x-i)(x+i)$.

De hoofdstelling van de algebra stelt nu dat door op deze manier $i$ in
te voeren niet alleen de veelterm $x^2+1$ te schrijven is (men zegt, te
'factoriseren' is) als een product van eerstegraadsveeltermen, maar dat
meteen elke $n$-de graadsveelterm, zelfs met complexe co\"effici\"enten, te
schrijven is als een product van eerste graadsveeltermen. 

De gevolgen van deze stelling laten zich voelen tot in heel wat
toepassingen. Bij de studie van lineaire differentiaalvergelijkingen
en differentievergelijkingen, bij de studie van trillingen of golven
of elektrische kringen (We noemen slechts enkele van een hele reeks
aan mekaar verwante toepassingen), blijken dit soort
veeltermvergelijkingen voor te komen, en blijkt de theorie heel
elegant te formuleren als men complexe getallen toelaat. Ook de
exponenti\"ele vorm blijkt hier van nut, omdat deze onmiddellijk duidelijk
maakt hoe sinuso\"idale verschijnselen zoals trillingen, wiskundig heel
verwant zijn met exponenti\"ele verschijnselen.

We besluiten met een opmerking over de soms afschrikkende naamgeving
van complexe getallen. Complex heeft de bijklank van 'ingewikkeld'
maar in oorsprong betekent 'complex getal' gewoon 'samengesteld
getal', d.w.z. te beschrijven met twee re\"ele getallen. (We merken
wel op dat het vaak nuttig is over een complex getal te denken als 1
wiskundig object). Ook de benaming 'imaginair deel' roept soms
vragen op. Maar het bovenstaande zou moeten duidelijk maken dat die
'imaginaire' getallen niet denkbeeldiger of schimmiger zijn dan
bijvoorbeeld negatieve getallen.

\end{document}

